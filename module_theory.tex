\section{Modultheorie}


\begin{question}
  Zeigen Sie, dass es auf jeder abelschen Gruppe genau eine $\Integer$-Modulstruktur gibt.
\end{question}


\begin{solution}
  Es sei $A$ eine abelsche Gruppe.
  Aus der Vorlesung ist die Bijektion
  \begin{align*}
    \{\text{$\Integer$-Modulstrukturen $\Integer \times A \to A$}\}
    &\longleftrightarrow
    \{\text{Ringhomomorphismen $\Integer \to \End(A)$}\},
    \\
                    \mu
    &\longmapsto    (n \mapsto (a \mapsto \mu(n,a))),
    \\
                    ( (n,a) \mapsto \phi(n)(a) )
    &\longmapsfrom  \phi.
  \end{align*}
  bekannt.
  Dabei ist
  \[
      \End(A)
    = \{f \colon A \to A \mid \text{$f$ ist additiv}\}
  \]
  ein Ring unter punktweiser Adddition und Komposition.
  Da es genau einen Ringhomomorphismus $\Integer \to \End(A)$ gibt (siehe Übung~\ref{qst: Z is inital}) folgt die Aussage.
\end{solution}


\begin{question}
  Es sei $R$ ein kommutativer Ring und $M$ ein $R$-Modul.
  Es sei $I \subseteq R$ ein Ideal.
  \begin{enumerate}
    \item
      Zeigen Sie, dass sich die $R$-Modulstruktur auf $M$ genau dann zu einer $R/I$-Modulstruktur fortsetzen lässt, wenn $IM =  0$ (d.h.\ wenn $am = 0$ für alle $a \in I$ und $m \in M$).
    \item
      Es sei $S \subseteq R$ eine multiplikative Teilmenge.
      Zeigen Sie, dass sich die $R$-Modulstruktur auf $M$ genau dann zu einer $R_S$-Modulstruktur fortsetzen lässt, wenn für jedes $s \in S$ die Abbildung $\lambda_s \colon M \to M$, $m \mapsto sm$ bijektiv ist.
  \end{enumerate}
\end{question}


% TODO: Adding a solution.


\begin{question}
  Es sei $M$ ein endlich erzeugter $R$-Modul.
  Zeigen Sie, dass jedes Erzeugendensystem $S \subseteq M$ ein endliches Erzeugendensystem enthält.
\end{question}


\begin{solution}
  Es sei $\{m_1, \dotsc, m_s\} \subseteq M$ ein endliches Erzeugendensystem.
  Da $S$ ein Erzeugendensystem ist, lässt sich jedes $m_i$ als $m_i = r_{i,1} s_{i,1} + \dotsb + r_{i,t_i} s_{i,t_i}$ mit $t_i \geq 0$, $s_{i,1}, \dotsc, s_{i,t_i} \in S$ und $r_{i,1}, \dotsc, r_{i,t_i} \in R$ schreiben.
  Für $S' \coloneqq \{s_{i,j} \mid i = 1, \dotsc, s, j = 1, \dotsc, t_i\}$ gilt dann $m_i \in \generate{S}$ für alle $i = 1, \dotsc, s$ und deshalb
  \[
              M
    =         \generate{m_1, \dotsc, m_s}
    \subseteq \generate{S'}
    \subseteq M.
  \]
  Also ist $\generate{S'} = M$ und somit $S'$ ein endliches Erzeugendensystem von $M$.
\end{solution}


\begin{question}
  Es sei $0 \to N \xrightarrow{f} M \xrightarrow{g} P \to 0$ eine kurze exakte Sequenz von $R$-Moduln.
  \begin{enumerate}
    \item
      Zeigen Sie, dass $P$ endlich erzeugt ist, wenn $M$ endlich erzeugt ist.
    \item
      Zeigen Sie, dass $M$ endlich erzeugt ist, wenn $P$ und $N$ endlich erzeugt sind.
  \end{enumerate}
\end{question}


% TODO: Adding a solution.



\begin{question}[subtitle=Charakterisierungen noetherscher Moduln]
  Es sei $M$ ein $R$-Modul.
  Zeigen Sie, dass die folgenden Bedingungen äquivalent sind:
  \begin{enumerate}
    \item
      Jeder $R$-Untermodul von $M$ ist endlich erzeugt.
    \item
      Jede aufsteigende Kette
      \[
        N_0 \subseteq N_1 \subseteq N_2 \subseteq N_3 \subseteq N_4 \subseteq \dotso
      \]
      von $R$-Moduln stabilisiert, i.e.\ es gibt ein $i \geq 0$ mit $N_j = N_i$ für alle $j \geq i$.
    \item
      Jede nicht-leere Menge $\mathcal{S}$ bestehend aus $R$-Untermoduln von $M$ besitzt ein (bezüglich der Inklusion) maximales Element.
  \end{enumerate}
\end{question}

% TODO: Adding a solution
