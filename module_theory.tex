\section{Modultheorie}


\begin{question}
  Zeigen Sie, dass es auf jeder abelschen Gruppe genau eine $\Integer$-Modulstruktur gibt.
\end{question}


% TODO: Adding a solution.


\begin{question}
  Es sei $R$ ein kommutativer Ring und $M$ ein $R$-Modul.
  Es sei $I \subseteq R$ ein Ideal.
  \begin{enumerate}
    \item
      Zeigen Sie, dass sich die $R$-Modulstruktur auf $M$ genau dann zu einer $R/I$-Modulstruktur fortsetzen lässt, wenn $IM =  0$ (d.h.\ wenn $am = 0$ für alle $a \in I$ und $m \in M$).
    \item
      Es sei $S \subseteq R$ eine multiplikative Teilmenge.
      Zeigen Sie, dass sich die $R$-Modulstruktur auf $M$ genau dann zu einer $R_S$-Modulstruktur fortsetzen lässt, wenn für jedes $s \in S$ die Abbildung $\lambda_s \colon M \to M$, $m \mapsto sm$ bijektiv ist.
  \end{enumerate}
\end{question}


% TODO: Adding a solution.


\begin{question}
  Es sei $0 \to N \xrightarrow{f} M \xrightarrow{g} P \to 0$ eine kurze exakte Folge von $R$-Moduln.
  \begin{enumerate}
    \item
      Zeigen Sie, dass $P$ endlich erzeugt ist, wenn $M$ endlich erzeugt ist.
    \item
      Zeigen Sie, dass $M$ endlich erzeugt ist, wenn $P$ und $N$ endlich erzeugt sind.
  \end{enumerate}
\end{question}


% TODO: Adding a solution.

