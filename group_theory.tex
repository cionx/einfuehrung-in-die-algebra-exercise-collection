\section{Gruppentheorie}


% TODO:
% Calculations:
%   Applications of Sylow’s theorems
%   Orbit combinatorics
% Theorems:


\begin{question}[subtitle = Multiple Choice]
  \begin{enumerate}
    \item
      Für jeden Körper $K$ und jedes $n \geq 1$ ist $C_n(K) \coloneqq \{x \in K \mid x^n = 1\}$ eine zyklische Untergruppe von $K^\times$.
    \item
      Ist $G$ ein Gruppe und $(N_i)_{i \in I}$ eine Familie normaler Untergruppen $N_i \subseteq G$, so ist auch $\bigcap_{i \in I} N_i \subseteq G$ normal.
  \end{enumerate}
\end{question}


\begin{solution}
  \begin{enumerate}
    \item
      Die Aussage ist wahr:
      Dass $C_n(K)$ eine Untergruppe von $K^\times$ ist, ergibt sich durch direktes Nachrechnen;
      alternativ erkennt man, dass die Abbildung $K^\times \to K^\times$, $x \mapsto x^n$ ein Gruppenhomomorphismus ist, und $C_n(K)$ ihr Kern ist.
      Die Gruppe $C_n(K)$ ist endlich, da sie die Nullstellenmenge des Polynoms $X^n - 1 \in K[X]$ ist.
      Als endliche Untergruppe der multiplikativen Gruppe eines Körpers ist $C_n(K)$ zyklisch.
    \item
      Die Aussage ist wahr:
      Für jedes $g \in G$ gilt nach Annahme $g N_i g^{-1} \subseteq N_i$ für alle $i \in I$;
      somit gilt auch $g \left( \bigcap_{i \in I} N_i \right) g^{-1} \subseteq \bigcap_{i \in I} (g N_i g^{-1}) \subseteq \bigcap_{i \in I} N_i$.
  \end{enumerate}
\end{solution}


\begin{question}
  Zeigen Sie, dass die symmetrische Gruppe $S_n$ von dem Zykel $\rho = (1 \, 2 \, \dotso \, n)$ und der Permutation $\tau = (1 \, 2)$ erzeugt wird.
\end{question}


\begin{solution}
  Für alle $i = 1, \dotsc, n$ gilt $\rho^{i-1}(1) = i$ und $\rho^{i-1}(2) = i+1$, und somit
  \[
    \rho^{i-1} \tau (\rho^{i-1})^{-1}
    = \rho^{i-1} (1 \quad 2) (\rho^{i-1})^{-1}
    = (\rho^{i-1}(1) \quad \rho^{i-1}(2))
    = (i \quad i+1).
  \]
  Deshalb gilt $(1 \quad 2), (2 \quad 3), \dotsc, (n-1 \quad n) \in \generate{\rho, \tau}$.
  Da $S_n$ von diese Elementartranspositionen erzeugt wird, gilt bereits $S_n \subseteq \generate{\rho, \tau} \subseteq S_n$, und somit $S_n = \generate{\rho, \tau}$.
\end{solution}


\begin{question}[subtitle = Zur Ordnung]
  Es sei $G$ eine endliche Gruppe.
  \begin{enumerate}
    \item
      Es seien $H, K \subseteq G$ zwei Untergruppen, so dass $|H|$ und $|K|$ teilerfremd sind.
      Zeigen Sie, dass $H \cap K = 1$.
    \item
      Es sei $N \subseteq G$ eine Untergruppe, so dass $H$ die einzige Untergruppe von Ordnung $|N|$ ist.
      Zeigen Sie, dass $N$ normal in $G$ ist.
    \item
      Es sei $|G|$ prim.
      Zeigen Sie, dass $G$ zyklisch ist.
    \item
      Es sei $N \subseteq G$ eine normale Untergruppe, so dass $|N|$ und $[G : N]$ teilerfremd sind.
      Zeigen Sie, dass $N$ die einzige Untergruppe von $G$ von Ordnung $|N|$ ist.
  \end{enumerate}
\end{question}


\begin{solution}
  \begin{enumerate}
    \item
      Der Schnitt $H \cap K$ ist sowohl von $H$ als auch von $K$ eine Untergruppe.
      Deshalb gilt $|H \cap K| \mid |H|$ und $|H \cap K| \mid |K|$.
      Da $|H|$ und $|K|$ teilerfremd sind, gilt bereits $|H \cap K| = 1$ und somit $H \cap K = 1$.
      
    \item
      Für $g \in G$ ist die Abbildung $c_g \colon G \to G$, $x \mapsto gxg^{-1}$ ein Gruppenautomorphismus (siehe Übung~\ref{question: inner automorphims}) ist somit auch $c_g(H) = gHg^{-1}$ eine Untergruppe von $G$ von Ordnung $|H|$.
      Aus der Eindeutigkeit von $H$ bezüglich dieser Eigenschaft folgt, dass bereits $g H g^{-1} = H$ gilt.
      Da dies für jedes $g \in G$ gilt ist $H$ normal.
    
    \item
      Es sei $g \in G$ mit $g \neq 1$.
      Dann ist $\generate{g} = \{g^n \mid n \in \Integer\}$ eine Untergruppe von $G$, weshalb $|{\generate{g}}| \mid G$ gilt.
      Da $|G|$ prim ist, gilt also entweder $|{\generate{g}}| = 1$ und somit $\generate{g} = 1$ oder $|{\generate{g}}| = G$ und somit $\generate{g} = G$.
      Da $1 \neq g \in \generate{g}$ gilt, kann der erste Fall ausgeschlossen werden.
    
    \item
      Es sei $\pi \colon G \to G/N$ die kanonische Projektion und $H \subseteq G$ eine Untergruppe von Ordnung $|N|$.
      Dann ist $\pi(H) \subseteq G/N$ eine Untergruppe, weshalb $|\pi(H)| \mid |G/N| = [G : N]$ gilt.
      Außerdem gilt $|H/(H \cap N)| = [H : H \cap N] \mid |H| = |N|$.
      
      Aus $\ker \pi|_H = H \cap \ker \pi = H \cap N$ erhalten wir, dass $\pi(H) \cong H / \ker \pi|_H \cong H / (H \cap N)$ gilt.
      Somit erhalten wir, dass $\pi(H)$ sowohl $[G : N]$ als auch $|N|$ teilt.
      Da $[G : N]$ und $|N|$ teilerfremd sind folgt hieraus, dass $\pi(H) = 1$ gilt.
      Es gilt also $H \subseteq \ker \pi = N$, und wegen $|H| = |N|$ somit bereits $H = N$.
  \end{enumerate}
\end{solution}


\begin{question}[subtitle = Bahnenkombinatorik]
  Es sei $G$ eine Gruppe der Ordnung $77$ die auf einer $17$-elementigen Menge $X$ wirkt.
  Zeigen Sie, dass die Wirkung mindestens $3$ Fixpunkte hat.
  (Ein Element $x \in X$ ist ein Fixpunkt falls $g.x = x$ für alle $g \in G$.)
\end{question}


\begin{solution}
  Ist $B \in X/G$ eine $G$-Bahn und $x \in X$ mit $B = G.x$, so gilt $|B| = |G.x| = [G : G_x] \mid |G|$.
  Für die Ordnung von $G$ gilt $|G| = 77 = 7 \cdot 11$, also ist $|B| \in \{1, 7, 11, 77\}$ für jede $G$-Bahn $B \in X/G$.
  Dabei gilt genau dann $|B| = 1$, falls $B = G.x$ für einen Fixpunkt $x \in X$ gilt;
  es gilt also zu zeigen, dass es mindestens drei einelementige $G$-Bahnen in $X$ gibt.
  
  Nach der Bahnengleichung gilt $17 = |X| = \sum_{B \in X/G} |B|$.
  Die einzigen Möglichkeiten, die Zahl $17$ als Summe der Zahlen $1$, $7$, $11$ und $77$ darzustellen, sind
  \[
      17
    = 11 + 6 \cdot 1
    = 2 \cdot 7 + 3 \cdot 1
    = 7 + 10 \cdot 1
    = 17 \cdot 1.
  \]
  In jeder der Möglichkeiten kommt der Summand $1$ mindestens dreimal vor, wodurch sich die Aussage ergibt.
\end{solution}



\begin{question}[subtitle = Mehr Bahnenkombinatorik]
  Es sei $p > 0$ prim und $G$ eine endliche $p$-Gruppe.
  \begin{enumerate}
    \item
      Es sei $X$ eine endliche Menge mit $p \nmid |X|$ und $G$ wirke auf $X$.
      Zeigen Sie, dass die Wirkung einen Fixpunkt besitzt, d.h.\ dass es ein $x \in X$ gibt, so dass $g.x = x$ für alle $g \in G$ gilt.
      (\emph{Hinweis}:
       Nutzen Sie die Bahnengleichung.)
    \item
      Zeigen Sie für $G \neq 1$, dass auch $\Center(G) \neq 1$.
      (\emph{Hinweis}:
       Lassen sich $G$ auf sich selbst durch Konjugation wirken.)
  \end{enumerate}
\end{question}


\begin{solution}
  \begin{enumerate}
    \item
      Es sei $B \in X/G$ eine $G$-Bahn.
      Es gilt genau dann $|B| = 1$, wenn $B = \{x\}$ für einen Fixpunkt $x \in X$.
      Ist $|B| \neq 1$ und $x \in B$, so folgt aus $|B| = G.x = [G : G_x] \mid |G| = p^r$ mit $r \geq 0$, dass bereits $p \mid |B|$ gilt.
      
      Aus der Bahnengleichung erhalten wir, dass $|X| = \sum_{B \in X/G} |B|$.
      Aus $p \nmid |X|$ erhalten wir, dass $p \nmid |B|$ für eine $G$-Bahn $B \in X/G$.
      Nach den obigen Beobachtungen erhalten wir, dass es einen Fixpunkt gibt.
    
    \item
      Es sei $X \coloneqq G$.
      Die Gruppe $G$ wirkt auf $X$ durch Konjugation, also durch $g.x = g x g^{-1}$ für alle $g \in G$, $x \in X$
      Ein Punkt $x \in X$ ist genau dann ein Fixpunkt wenn $g x g^{-1} = x$ für alle $g \in G$, wenn also $x \in \Center(G)$.
      Wie im vorherigen Aussagenteil erhalten wir damit, dass für $B \in X/G$ mit $B = G.x$ für $x \in X$ genau dann $p \mid |B|$ gilt wenn $x \notin \Center(G)$ (hier bezeichnet $X/G$ die Menge der $G$-Bahnen in $X$, also die Menge der Konjugationsklassen in $G$, und nicht den Quotienten trivialen $G/G$).
      Da $|G| = p^r$ mit $r \geq 1$ gilt, erhalten wir aus der Bahnengleichung, dass
      \[
                0
        \equiv  |G|
        =       \sum_{B \in X/G} |B|
        \equiv  \sum_{\substack{B \in X/G \\ |B| = 1}} |B|
        =       |{\Center(G)}|
        \mod    p.
      \]
      Deshalb gilt $p \mid |{\Center(G)}|$.
      Insbesondere ist $|{\Center(G)}| \neq 1$ und somit $\Center(G) \neq 1$.
 \end{enumerate} 
\end{solution}


\begin{question}
  Es sei $K$ ein Körper.
  \begin{enumerate}
    \item
      Zeigen Sie, dass
      \[
                  \Borel_2(K)
        \coloneqq \left\{
                    \begin{pmatrix}
                      a_1 & b   \\
                      0   & a_2
                    \end{pmatrix}
                  \,\middle|\,
                    a_1, a_2 \in K^\times,
                    b \in K
                  \right\}.
      \]
      eine Untergruppe von $\GL_2(K)$ ist.
    \item
      Zeigen Sie, dass $\Borel_2(K)$ nicht normal in $\GL_2(K)$ ist.
    \item
      Zeigen Sie, dass
      \[
                  \Upper_2(K)
        \coloneqq \left\{
                    \begin{pmatrix}
                      1 & b \\
                      0 & 1
                    \end{pmatrix}
                  \,\middle|\,
                    b \in K
                  \right\}.
      \]
      eine normale Untergruppe von $\Borel_2(K)$ ist, und dass $\Borel_2(K)/\Upper_2(K) \cong K^\times \times K^\times$.
    \item
      Entscheiden Sie, ob $\Borel_2(K) \cong \Upper_2(K) \times K^\times \times K^\times$.
  \end{enumerate}
\end{question}


\begin{solution}
  \begin{enumerate}
    \item
      Die Einheitsmatrix liegt in $\Borel_2(K)$, da $1 \in K^\times$.
      Es gilt
      \begin{equation}
        \label{equation: product of upper triangular matrices}
        \begin{pmatrix}
          a_1 & b   \\
          0   & a_2
        \end{pmatrix}
        \begin{pmatrix}
          a'_1  & b'   \\
          0     & a'_2
        \end{pmatrix}
        =
        \begin{pmatrix}
          a_1 a'_1  & a'_1 b + b' a_2 \\
          0         & a_2 a'_2
        \end{pmatrix}
      \end{equation}
      mit $a_1 a'_1, a_2 a'_2 \in K^\times$ falls $a_1, a'_1, a_2, a'_2 \in K^\times$;
      deshalb ist $\Borel_2(K)$ abgeschlossen unter Multiplikation ist.
      Für $\begin{psmallmatrix} a_1 & b \\ 0 & a_2 \end{psmallmatrix} \in \Borel_2(K)$ gilt auch
      \[
        \begin{pmatrix}
          a_1 & b   \\
          0   & a_2
        \end{pmatrix}^{-1}
        =
        \frac{1}{a_1 a_2}
        \begin{pmatrix}
          a_2 &           -b    \\
          0   & \phantom{-}a_1
        \end{pmatrix}
        \in \Borel_2(K),
      \]
      also ist $\Borel_2(K)$ auch unter Inversion abgeschlossen.
    
    \item
      Es gilt
      \[
        \begin{pmatrix}
          0 & 1 \\
          1 & 0
        \end{pmatrix}
        \begin{pmatrix}
          1 & 1 \\
          0 & 1
        \end{pmatrix}
        \begin{pmatrix}
          0 & 1 \\
          1 & 0
        \end{pmatrix}
        =
        \begin{pmatrix}
          1 & 0 \\
          1 & 1
        \end{pmatrix}
        \notin \Borel_2(K).
      \]
      
    \item
      Aus \eqref{equation: product of upper triangular matrices} ergibt sich, dass die Abbildung
      \[
                \varphi
        \colon  \Borel_2(K) \to K^\times \times K^\times,
        \quad   \begin{pmatrix}
                  a_1 & b   \\
                  0   & a_2
                \end{pmatrix}
                \mapsto
                (a_1, a_2)
      \]
      ein surjektiver Gruppenhomomorphismus ist, und es gilt $\ker \varphi = \Upper_2(K)$.
      Folglich ist $\Upper_2(K)$ normal in $\Borel_2(K)$ und $\Borel_2(K) / \Upper_2(K) \cong \im \varphi = K^\times \times K^\times$.
    
    \item
      Für $K = \Field_2$ sind die Gruppen isomorph:
      Es gilt $\Field_2^\times \cong 1$ und somit auch $\Field_2^\times \times \Field_2^\times \cong 1$ sowie $\Borel_2(\Field_2) = \Upper_2(\Field_2)$.
      Deshalb gilt
      \[
              \Borel_2(\Field_2)
        \cong \Upper_2(\Field_2)
        \cong \Upper_2(\Field_2) \times \Field_2^\times \times \Field_2^\times.
      \]
      
      Für $K \neq \Field_2$ sind die Gruppen nicht isomorph:
      Die Gruppe $\Borel_2(K)$ ist dann nicht abelsch, denn es gibt $a_1, a_2 \in K^\times$ mit $a_1 \neq a_2$ und somit gilt
      \[
        \begin{pmatrix}
          a_1 & 0   \\
          0   & a_2
        \end{pmatrix}
        \begin{pmatrix}
          1 & 1 \\
          0 & 1
        \end{pmatrix}
        =
        \begin{pmatrix}
          a_1 & a_1 \\
          0   & a_2
        \end{pmatrix}
        \neq
        \begin{pmatrix}
          a_1 & a_2 \\
          0   & a_2
        \end{pmatrix}
        =
        \begin{pmatrix}
          1 & 1 \\
          0 & 1
        \end{pmatrix}
        \begin{pmatrix}
          a_1 & 0   \\
          0   & a_2
        \end{pmatrix}.
      \]
      Die Gruppe $\Upper_2(K)$, und damit auch die Gruppe $\Upper_2(K) \times K^\times \times K^\times$, ist allerdings abelsch, denn für alle $b_1, b_2 \in K$ gilt
      \[
        \begin{pmatrix}
          1 & b_1 \\
          0 & 1
        \end{pmatrix}
        \begin{pmatrix}
          1 & b_2 \\
          0 & 1
        \end{pmatrix}
        =
        \begin{pmatrix}
          1 & b_1 + b_2 \\
          0 & 1
        \end{pmatrix}
        =
        \begin{pmatrix}
          1 & b_2 + b_1 \\
          0 & 1
        \end{pmatrix}
        =
        \begin{pmatrix}
          1 & b_2 \\
          0 & 1
        \end{pmatrix}
        \begin{pmatrix}
          1 & b_1 \\
          0 & 1
        \end{pmatrix}.
      \]
  \end{enumerate}
\end{solution}


\begin{question}
  Es sei $G$ eine Gruppe.
  \begin{enumerate}
    \item
      Es seien $H, H_1, H_2 \subseteq G$ Untergruppen mit $H \subseteq H_1 \cup H_2$,
      Zeigen Sie dass bereits $H \subseteq H_1$ oder $H \subseteq H_2$ gilt.
    \item
      Folgern Sie:
      Sind $H_1, H_2 \subseteq G$ seien zwei Untergruppen, so ist $H_1 \cup H_2$ genau dann eine Untergruppe ist, wenn $H_1 \subseteq H_2$ oder $H_2 \subseteq H_1$.
    \item
      Geben Sie ein Beispiel für eine eine Gruppe $G$ und Untergruppen $H_1, H_2, H_3 \subseteq G$ an, so dass zwar $H_i \subsetneq H_j$ für alle $i \neq j$, aber $H_1 \cup H_2 \cup H_3$ eine Untergruppe von $G$ ist.
  \end{enumerate}
\end{question}


\begin{solution}
  \begin{enumerate}
    \item
      Würde $H \nsubseteq H_2$ und $H \nsubseteq H_1$ gelten, so gebe es $h_1, h_2 \in H$ mit $h_1 \notin H_2$ und $h_2 \in H_1$.
      Da $h_1, h_2 \in H \subseteq H_1 \cup H_2$ gilt, müsste allerdings $h_1 \in H_1$ und $h_2 \in H_2$ gelten.
      Für das Produkt $h_1 h_2$ würde dann $h_1 h_2 \in H_1$ gelten, denn sonst wäre $h_2 = h_1^{-1} h_1 h_2 \in H_1$, im Widerspruch zur Wahl von $h_1$.
      Analog ergebe sich aber auch, dass $h_1 h_2 \notin H_2$ gilt.
      Es müsste aber $h_1 h_2 \in H \subseteq H_1 \cup H_2$ gelten, da $H$ ein Untergruppe ist.
      
    \item
      Gilt $H_1 \subseteq H_2$ oder $H_2 \subseteq H_1$, so gilt $H_1 \cup H_2 = H_2$ oder $H_1 \cup H_2 = H_1$, weshalb $H_1 \cup H_2$ dann eine Untergruppe ist.
      
      Ist andererseits $H_1 \cup H_2$ eine Untergruppe, so ergibt sich aus den vorherigen Aufgabenteil mit $H = H_1 \cup H_2$ dass bereits $H_1 \cup H_2 \subseteq H_1$ oder $H_1 \cup H_2 \subseteq H_2$ gilt, und somit $H_2 \subseteq H_1$ oder $H_1 \subseteq H_2$.
      
    \item
      Es sei $G = \Integer/2 \oplus \Integer/2$ und es seien
      \begin{align*}
        H_1 &= \generate{ (1,0) } = \{ (0,0), (1,0) \},
        \\
        H_2 &= \generate{ (1,1) } = \{ (0,0), (1,1) \},
        \\
        H_3 &= \generate{ (0,1) } = \{ (0,0), (0,1) \}.
      \end{align*}
      Dann gilt $H_i \subseteq H_j$ für alle $1 \leq i \neq j \leq n$ und $H_1 \cup H_2 \cup H_3 = G$.
  \end{enumerate}
\end{solution}


\begin{question}[subtitle = Ein Kriterium für maximale Untergruppen]
  Es sei $G$ ein Gruppe und $H \subseteq G$ eine Untergruppe, so dass $[G : H]$ endlich und prim ist.
  Zeigen Sie, dass $H$ eine maximale echte Untergruppe von $G$ ist. Entscheiden Sie, ob $H$ notwendigerweise normal in $G$ ist.
\end{question}


\begin{solution}
  Es sei $p \coloneqq [G : H]$.
  Da $p$ eine Primzahl ist gilt inbesondere $p \neq 1$, weshalb $H$ eine echte Untergruppe von $G$ ist.
  Ist $K \subsetneq G$ eine echte Untergruppe von $G$ mit $H \subseteq K$, so gilt wegen der Multiplikativität des Index’, dass 
  \[
      p
    = [G : H]
    = [G : K] [K : H].
  \]
  Da $p$ eine Primzahl ist, gilt entweder $[G : K] = p$ und $[K : H] = 1$, oder $[G : K] = 1$ und $[K : H] = p$.
  Es gilt $[G : K] > 1$, da $K$ eine echte Untergruppe von $G$ ist, und somit $[K : H] = 1$.
  Also ist $K = H$, und somit $H$ eine maximale echte Untergruppe.
  
  $H$ ist nicht notwendigerweise normal in $G$:
  Für $G = S_3$ und $H = \generate{(1\,2)} = \{\id, (1\,2)\}$ ist $H$ zwar nicht normal in $G$, aber $[G : H] = |G|/|H| = 6/2 = 3$ ist prim.
\end{solution}


\begin{question}[subtitle = Innere Automorphismen]
  \label{question: inner automorphims}
  Es sei $G$ eine Gruppe.
  \begin{enumerate}
    \item
      Zeigen Sie, dass für jedes $g \in G$ die Abbildung $c_g \colon G \to G$, $h \mapsto g h g^{-1}$ ein Gruppenautomorphismus ist.
    \item
      Zeigen Sie, dass die Abbildung $c \colon G \to G$, $g \mapsto c_g$ ein Gruppenhomomorphismus ist.
    \item
      Zeigen Sie, dass $\ker c = \Center(G)$.
    \item
      Zeigen Sie, dass $\Inner G \coloneqq \im c$ eine normale Untergruppe von $\Aut G$ ist.
  \end{enumerate}
  Man bezeichnet $\Inner G$ als die Gruppe der \emph{inneren Automorphismen} von $G$.
\end{question}


\begin{solution}
  \begin{enumerate}
    \item
      Für alle $h_1, h_2 \in G$ gilt
      \[
          c_g(h_1 h_2)
        = g h_1 h_2 g^{-1}
        = g h_1 g^{-1} g h_2 g^{-1}
        = c_g(h_1) c_g(h_2),
      \]
      also ist $c_g$ ein Gruppenhomomorphismus.
      Für alle $h \in G$ gilt
      \[
          c_g( c_{g^{-1}}(h) )
        = g g^{-1} h g g^{-1}
        = h
        = g^{-1} g h g^{-1} g
        = c_{g^{-1}}( c_g(h) ),
      \]
      also ist $c_g$ bijektiv mit $c_g^{-1} = c_{g^{-1}}$.
    \item
      Für alle $g_1, g_2 \in G$ gilt
      \[
          c_{g_1 g_2}(h)
        = (g_1 g_2) h (g_1 g_2)^{-1}
        = g_1 g_2 h g_2^{-1} g_1^{-1}
        = c_{g_1}( c_{g_2}(h) )
        \qquad
        \text{für alle $h \in G$}
      \]
      und somit $c_{g_1 g_2} = c_{g_1} c_{g_2}$.
    \item
      Für $g \in G$ gilt
      \begin{align*}
              g \in \ker c
        &\iff c_g = \id_G
         \iff \forall h \in G: c_g(h) = h
        \\
        &\iff \forall h \in G: ghg^{-1} = h
         \iff \forall h \in G: gh = hg
         \iff g \in \Center(G).
      \end{align*}
    \item
      Da $c$ ein Gruppenhomomorphismus ist, ist $\Inner G$ ein Untergruppe von $\Aut G$.
      Für jedes $\phi \in \Aut G$ und jedes $g \in G$ gilt $\phi c_g \phi^{-1} = c_{\phi(g)}$, denn für alle $h \in G$ gilt
      \[
        (\phi c_g \phi^{-1})(h)
        = \phi( c_g( \phi^{-1}(h) ) )
        = \phi(g \phi^{-1}(h) g^{-1})
        = \phi(g) h \phi(g)^{-1}
        = c_{\phi(g)}(h).
      \]
      Folglich ist $\phi \, {\Inner G} \, \phi^{-1} \subseteq \im c$ für alle $\phi \in \Aut G$, also $\im c$ normal in $\Aut G$.
  \end{enumerate}
\end{solution}


\begin{question}
  Es sei $G$ eine Gruppe, die auf einer Menge $X$ vermöge $G \times X \to X$, $(g,x) \mapsto g.x$ wirkt.
  \begin{enumerate}
    \item
      Definieren Sie die Bahn $G.x$ und den Stabilisator $G_x$ eines Elementes $x \in X$.
    \item
      Zeigen Sie, dass $G_x$ für alle $x \in X$ eine Untergruppe von $G$ ist.
    \item
      Konstruieren Sie für jedes $x \in X$ eine Bijektion $G/G_x \to G.x$.
    \item
      Es seien $x, y \in X$ zwei Elemente mit gleicher $G$-Bahn.
      Zeigen Sie, dass die Stabilisatoren $G_x$ und $G_y$ konjugiert zueinander sind.
    \item
      Entscheiden Sie, ob auch die Umkerung der obigen Aussage notwendigerweise gilt.
    \item
      Zeigen Sie, dass $X$ die disjunkte Vereinigung der $G$-Bahnen ist.
  \end{enumerate}
\end{question}


\begin{solution}
  \begin{enumerate}
    \item
      Es gilt $G.x = \{g.x \mid g \in G\}$, der Stabilisator von $x$ ist $G_x = \{g \in G \mid g.x = x\}$.
    \item
      Es gilt $1 \in G_x$ da $1.x = x$.
      Für $g_1, g_2 \in G_x$ gilt $(g_1 g_2).x = g_1.(g_2.x) = g_1.x = x$ und somit auch $g_1 g_2 \in G_x$.
      Für $g \in G$ gilt $g^{-1}.x = g^{-1}.(g.x) = (g^{-1}.g).x = 1.x = x$ und somit auch $g^{-1} \in G_x$.
      Ingesamt zeigt dies, dass $G_x$ ein Untergruppe von $G$ ist.
    \item
      Die Abbildung $f \colon G \to G.x$, $g \mapsto g.x$ ist surjektiv, und für $g_1, g_2 \in G$ gilt
      \begin{align*}
              f(g_1) = f(g_2)
        &\iff g_1.x = g_2.x
        \iff  g_2^{-1}.g_1.x = x
        \\
        &\iff (g_2^{-1} g_1).x = x
        \iff  g_2^{-1} g_1 \in G_x
        \iff  g_1 G_x = g_2 G_x,
      \end{align*}
      weshalb $f$ durch eine wohldefinierte Bijektion $\overline{f} \colon G/G_x \to G.x$, $\overline{g} \mapsto g.x$ faktorisiert.
    \item
      Haben $x$ und $y$ die Gleiche $G$-Bahn, so gibt es $g \in G$ mit $y = g^{-1}.x$.
      Für alle $h \in G$ gilt dann
      \begin{align*}
              h \in G_y
        &\iff h.y = y
        \iff  h.g^{-1}.x = g^{-1}.x
        \\
        &\iff g.h.g^{-1}.x = x
        \iff  (g h g^{-1}).x = x
        \iff  g h g^{-1} \in G_x.
      \end{align*}
      Wegen der Bijektivität der Konjugationsabbildung $G \to G$, $h \mapsto ghg^{-1}$ folgt, dass $g G_y g^{-1} = G_x$.
    \item
      Die Umkehrung gilt nicht:
      Gilt etwa $G = 1$, so gilt $G_x = G$ für alle $x \in X$, aber alle Bahnen sind einelementig.
      Für $|X| \geq 2$ ergibt dies ein Gegenbeispiel.
      
      Allgemeiner kann man eine beliebige Gruppe $G$ auf einer Menge $X$ mit $|X| \geq 2$ trivial wirken lassen, d.h.\ es gelte $g.x = x$ für alle $g \in G$ und $x \in X$.
      Dann gilt $G_x = G$ für alle $x \in X$ aber alle Bahnen sind einelementig.
    \item
      Es genügt zu zeigen, dass $x \sim y \iff x \in G.y$ eine Äquivalenzrelation auf $X$ definiert, denn dann sind die $G$-Bahnen genau die Äquivalenzklassen von ${\sim}$.
      Da $x = 1.x \in G.x$ ist die Relation reflexiv.
      Gilt $x \sim y$ so gibt es $g \in G$ mit $x = g.x$;
      dann gilt auch $y = g^{-1}.x \in G.x$ und somit $y \sim x$.
      Für $x, y, z \in X$ mit $x \sim y$ und $y \sim z$ gibt es $g, h \in G$ mit $x = g.y$ und $y = h.z$;
      dann gilt auch $x = g.y = g.h.z = (gh).z \in G.z$ und somit $x \sim z$.
  \end{enumerate}
\end{solution}


% \begin{question}
%   Es sei $G$ ein Gruppe.
%   \begin{enumerate}
%     \item
%       Definieren Sie die Kommutatoruntergruppe $[G,G]$ von $G$.
%     \item
%       Zeigen Sie, dass $[G,G]$ eine normale Untergruppe von $G$ ist, und dass $G/[G,G]$ abelsch ist.
%     \item
%       Es sei $N \subseteq G$ eine normale Untergruppe, so dass $G/N$ abelsch ist.
%       Zeigen Sie, dass $N \subseteq [G,G]$.
%     \item
%       Zeigen Sie, dass $G/[G,G]$ die folgende universelle Eigenschaft hat:
%       Ist $A$ eine abelsche Gruppe und $\varphi \colon G \to A$ ein Gruppenhomomorphismus, so gibt es einen eindeutigen Gruppenhomomorphismus $\hat{\varphi} \colon G/[G,G] \to A$, der das folgende Diagram zum kommutieren bringt:
%       \[
%         \begin{tikzcd}[ampersand replacement = \&]
%               G
%               \arrow{dr}{\varphi}
%               \arrow[swap]{dd}{\pi}
%           \&  {}
%           \\
%               {}
%           \&  A
%           \\
%               G/[G,G]
%               \arrow[swap]{ru}{\hat{\varphi}}
%           \&  {}
%         \end{tikzcd}
%       \]
%       Dabei bezeichnet $\pi \colon G \to G/[G,G]$, $g \mapsto \overline{g}$ die kanonische Projektion.
%   \end{enumerate}
% \end{question}


% \begin{question}[subtitle = Multiple Choice I]
%   Entscheiden Sie, ob die folgenden Aussagen allgemein gültig sind, und geben sie gegebenenfalls ein Gegenbeispiel an.
%   \begin{enumerate}
%     \item
%       Ist $G$ eine Gruppe und $N \subseteq G$ eine normale Untergruppe, so gilt $G \cong (G/N) \times N$.
%     \item
%       Ist $G$ eine endliche Gruppe, so dass $G/N$ für normale Untergruppe $N \subseteq G$ mit $N \neq 1$ abelsch ist, so ist auch $G$ abelsch.
%     \item
%       Zwei Gruppen $G_1$ und $G_2$ sind genau dann isomorph, wenn $G_1 \times H \cong G_2 \times H$ für jede Gruppe $H$.
%     \item
%       Sind $G_1$ und $G_2$ zwei Gruppen, so ist jede Untergruppe von $G_1 \times G_2$ von der Form $H_1 \times H_2$ für Untergruppen $H_1 \subseteq G_1$ und $H_2 \subseteq G_2$.
%     \item
%       Sind $G_1$ und $G_2$ zwei Gruppen, so dass es Gruppenepimorphismen $\phi \colon G_1 \to G_2$ und $\psi \colon G_2 \to G_1$ gibt, so gilt $G_1 \cong G_2$.
%   \end{enumerate}
% \end{question}
% 
% 
% \begin{solution}
%   \begin{enumerate}
%     \item
%       Die Aussage ist falsch:
%       Es sei $G = \Integer$ und $N = 2 \Integer$.
%       Dann ist
%       \[
%               (G/N) \times N
%         \cong (\Integer / 2 \Integer) \times (2 \Integer)
%         \cong (\Integer / 2 \Integer) \times \Integer.
%       \]
%       Es ist allerdings $(\Integer / 2 \Integer) \times \Integer \ncong \Integer$, da $(\Integer / 2 \Integer) \times \Integer$ ein Element der Ordnung $2$ enthält (nämlich $(1,0)$), $\Integer$ aber nicht.
%     \item
%       Die Aussage ist falsch:
%       Die einzige nicht-trivialen normalen Untergruppe von $S_3$ sind $N = \generate{(1 \, 2 \, 3)} = \{\id, (1 \, 2 \, 3), (1 \, 3 \, 2) \}$ und $S_3$ selbst.
%       Der Quotient $S_3 / N$ hat Ordnung $2$, weshalb $S_3 / N \cong \Integer / 2 \Integer$ abelsch ist, und $S_3/S_3 = 1$ ist ohnehin abelsch.
%       Die Gruppe $S_3$ selbst ist allerdings nicht abelsch.
%       
%       Alternativ ist $A_n$ für $n \geq 5$ einfach, weshalb $A_n$ der einzige nicht-triviale Normalteiler von $A_n$ ist, aber $A_4$ ist für $n \geq 4$ nicht abelsch.
%     \item
%       Die Aussage ist wahr:
%       Gilt $G_1 \cong G_2$, so gibt es einen Isomorphismus $\phi \colon G_1 \to G_2$.
%       Für jede Gruppe $H$ ist dann $\phi \times \id_H \colon G_1 \times H \to G_2 \times H$ ein Isomorphismus, und somit $G_1 \times H \cong G_2 \times H$.
%       Gilt andererseits $G_1 \times H \cong G_2 \times H$ für jede Gruppe $H$, so gilt inbesondere $G_1 \cong G_1 \times 1 \cong G_2 \times 1 \cong G_2$.
%     \item
%       Die Aussage ist falsch:
%       Ist $G \neq 1$ eine Gruppe und $G_1 = G_2 = G$, so ist die Diagonale $\Delta = \{(g,g) \mid g \in G\}$ eine Untergruppe von $G_1 \times G_2 = G \times G$, die sich nicht als ein solches Produkt schreiben lässt.
%     \item
%       Die Aussage ist falsch:
%       Für die Gruppen
%       \begin{gather*}
%         G_1
%       = \bigoplus_{n \in \Natural} \Integer
%       = \Integer \oplus \Integer \oplus \Integer \oplus \dotsb
%       \shortintertext{und}
%         G_2
%       = \Integer/2\Integer \oplus \bigoplus_{n \in \Natural} \Integer
%       = \Integer/2\Integer \oplus \Integer \oplus \Integer \oplus \dotsb
%       \end{gather*}
%       gibt es Gruppenepimorphismen
%       \begin{gather*}
%         \phi \colon G_1 \to G_2,
%         \quad
%         (n_1, n_2, n_3, \dotsc)
%         \mapsto
%         (\overline{n_1}, n_2, n_3, \dotsc)
%       \shortintertext{und}
%         \psi \colon G_2 \to G_1,
%         \quad
%         (\overline{n_1}, n_2, n_3, \dotsc)
%         \mapsto
%         (n_2, n_3, \dotsc).
%       \end{gather*}
%       Es gilt aber $G_1 \ncong G_2$, denn $G_2$ enthält ein Element der Ordnung $2$, $G_1$ jedoch nicht.
%   \end{enumerate}
% \end{solution}
% 
% 
% \begin{question}
%   Es seien $G_1$ und $G_2$ zwei Gruppen, $N_1 \subseteq G_1$ und $N_2 \subseteq G_2$ zwei normale Untergruppen.
%   Geben Sie jeweils Beispiele für die folgenden Situationen:
%   \begin{enumerate}
%     \item
%       Es gilt $G_1 \cong G_2$ und $N_1 \cong N_2$, aber $G_1/N_1 \ncong G_2/N_2$.
%     \item
%       Es gilt $G_1 \cong G_2$ und $G_1/N_1 \cong G_2/N_2$, aber $N_1 \ncong N_2$.
%     \item
%       Es gilt $G_1/N_1 \cong G_2/N_2$ und $N_1 \cong N_2$, aber $G_1 \ncong G_2$.
%   \end{enumerate}
% \end{question}
% 
% 
% \begin{solution}
%   \begin{enumerate}
%     \item
%       Es seien $G_1 = G_2 = \bigoplus_{n \geq 0} \Integer$, sowie \mbox{$N_1 = \bigoplus_{n \geq 1} \Integer$} und $N_2 = \bigoplus_{n \geq 2} \Integer$.
%       Dann gilt $G_1 = G_2 \cong N_1 \cong N_2$ aber
%       \[
%         G_1/N_1 \cong \Integer \ncong \Integer \oplus \Integer = G_2/N_2.
%       \]
%     \item
%       Es seien $G_1 = G_2 = \bigoplus_{n \geq 0} \Integer$ und
%       \begin{gather*}
%         N_1 \coloneqq \Integer \oplus 0 \oplus 0 \oplus 0 \oplus \dotsb
%       \shortintertext{und}
%         N_2 \coloneqq \Integer \oplus \Integer \oplus 0 \oplus 0 \oplus \dotsb
%       \end{gather*}
%       Dann gilt
%       \[
%               G_1/N_1
%         \cong \bigoplus_{n \geq 1} \Integer
%         \cong \bigoplus_{n \geq 2} \Integer
%         =     G_2/N_2.
%       \]
%       Es gilt aber $N_1 \ncong N_2$, denn $N_1 \cong \Integer$ ist zyklisch, $\Integer \oplus \Integer$ aber nicht.
%     \item
%       Es seien $G_1 = \Integer/4\Integer$ und $G_2 = \Integer/2\Integer \oplus \Integer/2\Integer$, sowie $N_1 = 2\Integer/4\Integer = \{\overline{0}, \overline{2}\}$ und $N_2 = \Integer/2\Integer \oplus 0$.
%       Wegen der Kommutativität von $G_1$ und $G_2$ handelt es sich jeweils um eine normale Untergruppe.
%       Da $N_1$ und $N_2$ beide zweielementig sind, gilt
%       \[
%         N_1 \cong \Integer/2\Integer \cong N_2
%       \]
%       (denn $\Integer/2\Integer$ ist die bis auf Isomorphie eindeutige zweielementige Gruppe).
%       Nach dem zweiten (oder dritten) Isomorphiesatz gilt
%       \[
%               G_1 / N_1
%         =     (\Integer/4\Integer) / (2\Integer/4\Integer)
%         \cong \Integer/2\Integer,
%       \]
%       und für den anderen Quotienten gilt
%       \begin{align*}
%                 G_2 / N_2
%         &=      (\Integer/2\Integer \oplus \Integer/2\Integer) / (\Integer/2\Integer \oplus 0)
%         \\
%         &\cong  ((\Integer/2\Integer)/(\Integer/2\Integer)) \oplus ((\Integer/2\Integer)/0)
%         \cong   0 \oplus \Integer/2\Integer
%         \cong   \Integer/2\Integer.
%       \end{align*}
%       Also gilt auch $G_1/N_1 \cong G_2/N_2$.
%       Es gilt aber $G_1 \ncong G_2$, da $G_1$ ein Element der Ordnung $4$ enthält, $G_2$ jedoch nicht.
%   \end{enumerate}
% \end{solution}
% 
% 
% 
% \begin{question}[subtitle = Gruppen mit trivialer Automorphismengruppe]
%   Es sei $G$ eine Gruppe mit $\Aut(G) = 1$.
%   \begin{enumerate}
%     \item
%       Zeigen Sie, dass $G$ abelsch ist.
%     \item
%       Zeigen Sie, dass $g = -g$ für alle $g \in G$.
%     \item
%       Folgern Sie, dass es eine eindeutige $\Field_2$-Vektorraumstruktur auf $G$ gibt.
%     \item
%       Folgern Sie, dass $G = 0$ oder $G \cong \Integer / 2 \Integer$.
%   \end{enumerate}
% \end{question}
% 
% 
% \begin{solution}
%   \begin{enumerate}
%     \item
%       Für $g \in G$ sei $c_g \colon G \to G$ die Konjugation mit $g$.
%       Dies ist ein Automorphismus von $G$, weshalb $c_g = \id_G$.
%       Somit ist $g \in \Center(G)$.
%     \item
%       Wegen der Kommutativität von $G$ ist die Abbildung $n \colon G \to G$, $g \mapsto -g$ ein Automorphismus von $G$.
%       Somit ist $n = \id_G$, also $-g = g$ für alle $g \in G$.
%     \item
%       Nach dem vorherigen Aufgabenteil ist $2 g = 0$ für alle $g \in G$.
%       Deshalb gibt es eine eindeutige $\Field_2$-Vektorraumstruktur auf $G$ via
%       \[
%         \overline{n} \cdot g = n \cdot g
%         \quad
%         \text{für alle $n \in \Integer$, $g \in G$},
%       \]
%       wie sich durch direktes Nachrechnen ergibt.
%     \item
%       Es sei $(b_i)_{i \in I}$ eine Basis von $G$ als $\Field_2$-Vektorraum.
%       Ist $G \neq 0$ und $G \ncong \Integer/2$, so ist $\dim_{\Field_2} G \geq 2$.
%       Es gibt daher $i_1, i_2 \in I$ with $i_1 \neq i_2$.
%       Die Permutation
%       \[
%         \sigma \colon \{b_i\}_{i \in I} \to \{b_i\}_{i \in I},
%         \quad
%         b_j
%         \mapsto
%         \begin{cases}
%           b_{i_2} & \text{falls $j = i_1$}, \\
%           b_{i_1} & \text{falls $j = i_2$}, \\
%           b_j     & \text{sonst},
%         \end{cases}
%       \]
%       induziert einen nicht-trivialen $\Field_2$-Vek\-tor\-raum\-auto\-mor\-phis\-mus $\alpha \colon G \to G$ mit
%       \[
%           \alpha\left( \sum_{i \in I} \lambda_i b_i \right)
%         = \sum_{i \in I} \lambda_i b_{\sigma(i)}.
%       \]
%       Dann ist $\alpha$ aber insbesondere ein nicht-trivialer Gruppenautomorphismus, im Widerspruch zu $\Aut(G) = 1$.
%   \end{enumerate}
% \end{solution}





