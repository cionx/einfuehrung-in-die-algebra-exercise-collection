\section{Gruppentheorie}


\begin{question}
  \begin{enumerate}
    \item
      Es sei $G$ eine Gruppe und $H_1, H_2 \subseteq G$ seien zwei Untergruppen.
      Zeigen Sie, dass $H_1 \cup H_2$ genau dann eine Untergruppe ist, wenn $H_1 \subseteq H_2$ oder $H_2 \subseteq H_1$.
    \item
      Geben Sie ein Beispiel für eine eine Gruppe $G$ und Untergruppen $H_1, H_2, H_3 \subseteq G$ an, so dass zwar $H_i \subsetneq H_j$ für alle $i \neq j$, aber $H_1 \cup H_2 \cup H_3$ eine Untergruppe von $G$ ist.
  \end{enumerate}
\end{question}


\begin{solution}
  \begin{enumerate}
    \item
      Gilt $H_1 \subseteq H_2$ oder $H_2 \subseteq H_1$, so gilt $H_1 \cup H_2 = H_2$ oder $H_1 \cup H_2 = H_1$, weshalb $H_1 \cup H_2$ dann eine Untergruppe ist.
      
      Gilt $H_1 \nsubseteq H_2$ und $H_2 \nsubseteq H_1$, so gibt es $h_1 \in H_1$ mit $h_1 \notin H_2$ und $h_2 \in H_2$ mit $h_2 \notin H_1$.
      Es ist $h_1 h_2 \notin H_1$, da sonst $h_2 = h_1^{-1} h_1 h_2 \in H_1$ gelten würde;
      analog gilt auch $h_1 h_2 \notin H_2$.
      Insgesamt gilt somit $h_1 h_2 \notin H_1 \cup H_2$, obwohl $h_1, h_2 \in H_1 \cup H_2$.
      Also ist $H_1 \cup H_2$ nicht multiplikativ abgeschlossen, und somit keine Untergruppe von $G$.
    \item
      Es sei $G = \Integer/2 \oplus \Integer/2$ und es seien
      \begin{align*}
        H_1 &= \generate{ (1,0) } = \{ (0,0), (1,0) \},
        \\
        H_2 &= \generate{ (1,1) } = \{ (0,0), (1,1) \},
        \\
        H_3 &= \generate{ (0,1) } = \{ (0,0), (0,1) \}.
      \end{align*}
      Dann gilt $H_i \subseteq H_j$ für alle $1 \leq i \neq j \leq n$ und $H_1 \cup H_2 \cup H_3 = G$.
  \end{enumerate}
\end{solution}


\begin{question}[subtitle = Ein Kriterium für maximale Untergruppen]
  Es sei $G$ ein Gruppe und $H \subseteq G$ eine Untergruppe, so dass $[G : H]$ endlich und prim ist.
  Zeigen Sie, dass $H$ eine maximale echte Untergruppe von $G$ ist. Entscheiden Sie, ob $H$ notwendigerweise normal in $G$ ist.
\end{question}


\begin{solution}
  Es sei $p \coloneqq [G : H]$.
  Da $p$ eine Primzahl ist gilt inbesondere $p \neq 1$, weshalb $H$ eine echte Untergruppe von $G$ ist.
  Ist $K \subsetneq G$ eine echte Untergruppe von $G$ mit $H \subseteq K$, so gilt wegen der Multiplikativität des Index’, dass 
  \[
      p
    = [G : H]
    = [G : K] [K : H].
  \]
  Da $p$ eine Primzahl ist, gilt entweder $[G : K] = p$ und $[K : H] = 1$, oder $[G : K] = 1$ und $[K : H] = p$.
  Es gilt $[G : K] > 1$, da $K$ eine echte Untergruppe von $G$ ist, und somit $[K : H] = 1$.
  Also ist $K = H$, und somit $H$ eine maximale echte Untergruppe.
  
  $H$ ist nicht notwendigerweise normal in $G$:
  Für $G = S_3$ und $H = \generate{(1\,2)} = \{\id, (1\,2)\}$ ist $H$ zwar nicht normal in $G$, aber $[G : H] = |G|/|H| = 6/2 = 3$ ist prim.
\end{solution}


\begin{question}[subtitle = Innere Automorphismen]
  Es sei $G$ eine Gruppe.
  \begin{enumerate}
    \item
      Zeigen Sie, dass für jedes $g \in G$ die Abbildung $c_g \colon G \to G$, $h \mapsto g h g^{-1}$ ein Gruppenautomorphismus ist.
    \item
      Zeigen Sie, dass die Abbildung $c \colon G \to G$, $g \mapsto c_g$ ein Gruppenhomomorphismus ist.
    \item
      Zeigen Sie, dass $\ker c = \Center(G)$.
    \item
      Zeigen Sie, dass $\Inner G \coloneqq \im c$ eine normale Untergruppe von $\Aut G$ ist.
  \end{enumerate}
  Man bezeichnet $\Inner G$ als die Gruppe der \emph{inneren Automorphismen} von $G$.
\end{question}


\begin{solution}
  \begin{enumerate}
    \item
      Für alle $h_1, h_2 \in G$ gilt
      \[
          c_g(h_1 h_2)
        = g h_1 h_2 g^{-1}
        = g h_1 g^{-1} g h_2 g^{-1}
        = c_g(h_1) c_g(h_2),
      \]
      also ist $c_g$ ein Gruppenhomomorphismus.
      Für alle $h \in G$ gilt
      \[
          c_g( c_{g^{-1}}(h) )
        = g g^{-1} h g g^{-1}
        = h
        = g^{-1} g h g^{-1} g
        = c_{g^{-1}}( c_g(h) ),
      \]
      also ist $c_g$ bijektiv mit $c_g^{-1} = c_{g^{-1}}$.
    \item
      Für alle $g_1, g_2 \in G$ gilt
      \[
          c_{g_1 g_2}(h)
        = (g_1 g_2) h (g_1 g_2)^{-1}
        = g_1 g_2 h g_2^{-1} g_1^{-1}
        = c_{g_1}( c_{g_2}(h) )
        \qquad
        \text{für alle $h \in G$}
      \]
      und somit $c_{g_1 g_2} = c_{g_1} c_{g_2}$.
    \item
      Für $g \in G$ gilt
      \begin{align*}
              g \in \ker c
        &\iff c_g = \id_G
         \iff \forall h \in G: c_g(h) = h
        \\
        &\iff \forall h \in G: ghg^{-1} = h
         \iff \forall h \in G: gh = hg
         \iff g \in \Center(G).
      \end{align*}
    \item
      Da $c$ ein Gruppenhomomorphismus ist, ist $\Inner G$ ein Untergruppe von $\Aut G$.
      Für jedes $\phi \in \Aut G$ und jedes $g \in G$ gilt $\phi c_g \phi^{-1} = c_{\phi(g)}$, denn für alle $h \in G$ gilt
      \[
        (\phi c_g \phi^{-1})(h)
        = \phi( c_g( \phi^{-1}(h) ) )
        = \phi(g \phi^{-1}(h) g^{-1})
        = \phi(g) h \phi(g)^{-1}
        = c_{\phi(g)}(h).
      \]
      Folglich ist $\phi \, {\Inner G} \, \phi^{-1} \subseteq \im c$ für alle $\phi \in \Aut G$, also $\im c$ normal in $\Aut G$.
  \end{enumerate}
\end{solution}


\begin{question}
  Es sei $G$ eine Gruppe, die auf einer Menge $X$ vermöge $G \times X \to X$, $(g,x) \mapsto g.x$ wirkt.
  \begin{enumerate}
    \item
      Definieren Sie die Bahn $G.x$ und den Stabilisator $G_x$ eines Elementes $x \in X$.
    \item
      Zeigen Sie, dass $G_x$ für alle $x \in X$ eine Untergruppe von $G$ ist.
    \item
      Konstruieren Sie für jedes $x \in X$ eine Bijektion $G/G_x \to G.x$.
    \item
      Es seien $x, y \in X$ zwei Elemente mit gleicher $G$-Bahn.
      Zeigen Sie, dass die Stabilisatoren $G_x$ und $G_y$ konjugiert zueinander sind.
    \item
      Entscheiden Sie, ob auch die Umkerung der obigen Aussage notwendigerweise gilt.
    \item
      Zeigen Sie, dass $X$ die disjunkte Vereinigung der $G$-Bahnen ist.
  \end{enumerate}
\end{question}


\begin{solution}
  \begin{enumerate}
    \item
      Es gilt $G.x = \{g.x \mid g \in G\}$, der Stabilisator von $x$ ist $G_x = \{g \in G \mid g.x = x\}$.
    \item
      Es gilt $1 \in G_x$ da $1.x = x$.
      Für $g_1, g_2 \in G_x$ gilt $(g_1 g_2).x = g_1.(g_2.x) = g_1.x = x$ und somit auch $g_1 g_2 \in G_x$.
      Für $g \in G$ gilt $g^{-1}.x = g^{-1}.(g.x) = (g^{-1}.g).x = 1.x = x$ und somit auch $g^{-1} \in G_x$.
      Ingesamt zeigt dies, dass $G_x$ ein Untergruppe von $G$ ist.
    \item
      Die Abbildung $f \colon G \to G.x$, $g \mapsto g.x$ ist surjektiv, und für $g_1, g_2 \in G$ gilt
      \begin{align*}
              f(g_1) = f(g_2)
        &\iff g_1.x = g_2.x
        \iff  g_2^{-1}.g_1.x = x
        \\
        &\iff (g_2^{-1} g_1).x = x
        \iff  g_2^{-1} g_1 \in G_x
        \iff  g_1 G_x = g_2 G_x,
      \end{align*}
      weshalb $f$ durch eine wohldefinierte Bijektion $\overline{f} \colon G/G_x \to G.x$, $\overline{g} \mapsto g.x$ faktorisiert.
    \item
      Haben $x$ und $y$ die Gleiche $G$-Bahn, so gibt es $g \in G$ mit $y = g^{-1}.x$.
      Für alle $h \in G$ gilt dann
      \begin{align*}
              h \in G_y
        &\iff h.y = y
        \iff  h.g^{-1}.x = g^{-1}.x
        \\
        &\iff g.h.g^{-1}.x = x
        \iff  (g h g^{-1}).x = x
        \iff  g h g^{-1} \in G_x.
      \end{align*}
      Wegen der Bijektivität der Konjugationsabbildung $G \to G$, $h \mapsto ghg^{-1}$ folgt, dass $g G_y g^{-1} = G_x$.
    \item
      Die Umkehrung gilt nicht:
      Gilt etwa $G = 1$, so gilt $G_x = G$ für alle $x \in X$, aber alle Bahnen sind einelementig.
      Für $|X| \geq 2$ ergibt dies ein Gegenbeispiel.
      
      Allgemeiner kann man eine beliebige Gruppe $G$ auf einer Menge $X$ mit $|X| \geq 2$ trivial wirken lassen, d.h.\ es gelte $g.x = x$ für alle $g \in G$ und $x \in X$.
      Dann gilt $G_x = G$ für alle $x \in X$ aber alle Bahnen sind einelementig.
    \item
      Es genügt zu zeigen, dass $x \sim y \iff x \in G.y$ eine Äquivalenzrelation auf $X$ definiert, denn dann sind die $G$-Bahnen genau die Äquivalenzklassen von ${\sim}$.
      Da $x = 1.x \in G.x$ ist die Relation reflexiv.
      Gilt $x \sim y$ so gibt es $g \in G$ mit $x = g.x$;
      dann gilt auch $y = g^{-1}.x \in G.x$ und somit $y \sim x$.
      Für $x, y, z \in X$ mit $x \sim y$ und $y \sim z$ gibt es $g, h \in G$ mit $x = g.y$ und $y = h.z$;
      dann gilt auch $x = g.y = g.h.z = (gh).z \in G.z$ und somit $x \sim z$.
  \end{enumerate}
\end{solution}


\begin{question}
  Es sei $G$ ein Gruppe.
  \begin{enumerate}
    \item
      Definieren Sie die Kommutatoruntergruppe $[G,G]$ von $G$.
    \item
      Zeigen Sie, dass $[G,G]$ eine normale Untergruppe von $G$ ist, und dass $G/[G,G]$ abelsch ist.
    \item
      Es sei $N \subseteq G$ eine normale Untergruppe, so dass $G/N$ abelsch ist.
      Zeigen Sie, dass $N \subseteq [G,G]$.
    \item
      Zeigen Sie, dass $G/[G,G]$ die folgende universelle Eigenschaft hat:
      Ist $A$ eine abelsche Gruppe und $\varphi \colon G \to A$ ein Gruppenhomomorphismus, so gibt es einen eindeutigen Gruppenhomomorphismus $\hat{\varphi} \colon G/[G,G] \to A$, der das folgende Diagram zum kommutieren bringt:
      \[
        \begin{tikzcd}[ampersand replacement = \&]
              G
              \arrow{dr}{\varphi}
              \arrow[swap]{dd}{\pi}
          \&  {}
          \\
              {}
          \&  A
          \\
              G/[G,G]
              \arrow[swap]{ru}{\hat{\varphi}}
          \&  {}
        \end{tikzcd}
      \]
      Dabei bezeichnet $\pi \colon G \to G/[G,G]$, $g \mapsto \overline{g}$ die kanonische Projektion.
  \end{enumerate}
\end{question}


% TODO: Adding a solution


\begin{question}[subtitle = Multiple Choice]
  \begin{enumerate}
    \item
      Für jeden Körper $K$ und jedes $n \geq 1$ ist $\{x \in K \mid x^n = 1\}$ eine zyklische Untergruppe von $K^\times$.
    \item
      Es sei $G$ eine Gruppe.
      Sind $K \subseteq N \subseteq G$ Untergruppen, so dass $N$ normal in $G$, und $K$ normal in $N$ ist, so ist $K$ bereits normal in $G$.
      (Mit anderen Worten:
       Normalität ist transitiv.)
  \end{enumerate}
\end{question}


\begin{solution}
  \begin{enumerate}
    \item
      Die Aussage ist wahr:
      % TODO: Adding a more detailed solution.
    \item
      Die Aussage ist falsch:
      % TODO: Adding a more detailed solution.
  \end{enumerate}
\end{solution}


% \begin{question}[subtitle = Multiple Choice I]
%   Entscheiden Sie, ob die folgenden Aussagen allgemein gültig sind, und geben sie gegebenenfalls ein Gegenbeispiel an.
%   \begin{enumerate}
%     \item
%       Ist $G$ eine Gruppe und $N \subseteq G$ eine normale Untergruppe, so gilt $G \cong (G/N) \times N$.
%     \item
%       Ist $G$ eine endliche Gruppe, so dass $G/N$ für normale Untergruppe $N \subseteq G$ mit $N \neq 1$ abelsch ist, so ist auch $G$ abelsch.
%     \item
%       Zwei Gruppen $G_1$ und $G_2$ sind genau dann isomorph, wenn $G_1 \times H \cong G_2 \times H$ für jede Gruppe $H$.
%     \item
%       Sind $G_1$ und $G_2$ zwei Gruppen, so ist jede Untergruppe von $G_1 \times G_2$ von der Form $H_1 \times H_2$ für Untergruppen $H_1 \subseteq G_1$ und $H_2 \subseteq G_2$.
%     \item
%       Sind $G_1$ und $G_2$ zwei Gruppen, so dass es Gruppenepimorphismen $\phi \colon G_1 \to G_2$ und $\psi \colon G_2 \to G_1$ gibt, so gilt $G_1 \cong G_2$.
%   \end{enumerate}
% \end{question}
% 
% 
% \begin{solution}
%   \begin{enumerate}
%     \item
%       Die Aussage ist falsch:
%       Es sei $G = \Integer$ und $N = 2 \Integer$.
%       Dann ist
%       \[
%               (G/N) \times N
%         \cong (\Integer / 2 \Integer) \times (2 \Integer)
%         \cong (\Integer / 2 \Integer) \times \Integer.
%       \]
%       Es ist allerdings $(\Integer / 2 \Integer) \times \Integer \ncong \Integer$, da $(\Integer / 2 \Integer) \times \Integer$ ein Element der Ordnung $2$ enthält (nämlich $(1,0)$), $\Integer$ aber nicht.
%     \item
%       Die Aussage ist falsch:
%       Die einzige nicht-trivialen normalen Untergruppe von $S_3$ sind $N = \generate{(1 \, 2 \, 3)} = \{\id, (1 \, 2 \, 3), (1 \, 3 \, 2) \}$ und $S_3$ selbst.
%       Der Quotient $S_3 / N$ hat Ordnung $2$, weshalb $S_3 / N \cong \Integer / 2 \Integer$ abelsch ist, und $S_3/S_3 = 1$ ist ohnehin abelsch.
%       Die Gruppe $S_3$ selbst ist allerdings nicht abelsch.
%       
%       Alternativ ist $A_n$ für $n \geq 5$ einfach, weshalb $A_n$ der einzige nicht-triviale Normalteiler von $A_n$ ist, aber $A_4$ ist für $n \geq 4$ nicht abelsch.
%     \item
%       Die Aussage ist wahr:
%       Gilt $G_1 \cong G_2$, so gibt es einen Isomorphismus $\phi \colon G_1 \to G_2$.
%       Für jede Gruppe $H$ ist dann $\phi \times \id_H \colon G_1 \times H \to G_2 \times H$ ein Isomorphismus, und somit $G_1 \times H \cong G_2 \times H$.
%       Gilt andererseits $G_1 \times H \cong G_2 \times H$ für jede Gruppe $H$, so gilt inbesondere $G_1 \cong G_1 \times 1 \cong G_2 \times 1 \cong G_2$.
%     \item
%       Die Aussage ist falsch:
%       Ist $G \neq 1$ eine Gruppe und $G_1 = G_2 = G$, so ist die Diagonale $\Delta = \{(g,g) \mid g \in G\}$ eine Untergruppe von $G_1 \times G_2 = G \times G$, die sich nicht als ein solches Produkt schreiben lässt.
%     \item
%       Die Aussage ist falsch:
%       Für die Gruppen
%       \begin{gather*}
%         G_1
%       = \bigoplus_{n \in \Natural} \Integer
%       = \Integer \oplus \Integer \oplus \Integer \oplus \dotsb
%       \shortintertext{und}
%         G_2
%       = \Integer/2\Integer \oplus \bigoplus_{n \in \Natural} \Integer
%       = \Integer/2\Integer \oplus \Integer \oplus \Integer \oplus \dotsb
%       \end{gather*}
%       gibt es Gruppenepimorphismen
%       \begin{gather*}
%         \phi \colon G_1 \to G_2,
%         \quad
%         (n_1, n_2, n_3, \dotsc)
%         \mapsto
%         (\overline{n_1}, n_2, n_3, \dotsc)
%       \shortintertext{und}
%         \psi \colon G_2 \to G_1,
%         \quad
%         (\overline{n_1}, n_2, n_3, \dotsc)
%         \mapsto
%         (n_2, n_3, \dotsc).
%       \end{gather*}
%       Es gilt aber $G_1 \ncong G_2$, denn $G_2$ enthält ein Element der Ordnung $2$, $G_1$ jedoch nicht.
%   \end{enumerate}
% \end{solution}
% 
% 
% \begin{question}
%   Es seien $G_1$ und $G_2$ zwei Gruppen, $N_1 \subseteq G_1$ und $N_2 \subseteq G_2$ zwei normale Untergruppen.
%   Geben Sie jeweils Beispiele für die folgenden Situationen:
%   \begin{enumerate}
%     \item
%       Es gilt $G_1 \cong G_2$ und $N_1 \cong N_2$, aber $G_1/N_1 \ncong G_2/N_2$.
%     \item
%       Es gilt $G_1 \cong G_2$ und $G_1/N_1 \cong G_2/N_2$, aber $N_1 \ncong N_2$.
%     \item
%       Es gilt $G_1/N_1 \cong G_2/N_2$ und $N_1 \cong N_2$, aber $G_1 \ncong G_2$.
%   \end{enumerate}
% \end{question}
% 
% 
% \begin{solution}
%   \begin{enumerate}
%     \item
%       Es seien $G_1 = G_2 = \bigoplus_{n \geq 0} \Integer$, sowie \mbox{$N_1 = \bigoplus_{n \geq 1} \Integer$} und $N_2 = \bigoplus_{n \geq 2} \Integer$.
%       Dann gilt $G_1 = G_2 \cong N_1 \cong N_2$ aber
%       \[
%         G_1/N_1 \cong \Integer \ncong \Integer \oplus \Integer = G_2/N_2.
%       \]
%     \item
%       Es seien $G_1 = G_2 = \bigoplus_{n \geq 0} \Integer$ und
%       \begin{gather*}
%         N_1 \coloneqq \Integer \oplus 0 \oplus 0 \oplus 0 \oplus \dotsb
%       \shortintertext{und}
%         N_2 \coloneqq \Integer \oplus \Integer \oplus 0 \oplus 0 \oplus \dotsb
%       \end{gather*}
%       Dann gilt
%       \[
%               G_1/N_1
%         \cong \bigoplus_{n \geq 1} \Integer
%         \cong \bigoplus_{n \geq 2} \Integer
%         =     G_2/N_2.
%       \]
%       Es gilt aber $N_1 \ncong N_2$, denn $N_1 \cong \Integer$ ist zyklisch, $\Integer \oplus \Integer$ aber nicht.
%     \item
%       Es seien $G_1 = \Integer/4\Integer$ und $G_2 = \Integer/2\Integer \oplus \Integer/2\Integer$, sowie $N_1 = 2\Integer/4\Integer = \{\overline{0}, \overline{2}\}$ und $N_2 = \Integer/2\Integer \oplus 0$.
%       Wegen der Kommutativität von $G_1$ und $G_2$ handelt es sich jeweils um eine normale Untergruppe.
%       Da $N_1$ und $N_2$ beide zweielementig sind, gilt
%       \[
%         N_1 \cong \Integer/2\Integer \cong N_2
%       \]
%       (denn $\Integer/2\Integer$ ist die bis auf Isomorphie eindeutige zweielementige Gruppe).
%       Nach dem zweiten (oder dritten) Isomorphiesatz gilt
%       \[
%               G_1 / N_1
%         =     (\Integer/4\Integer) / (2\Integer/4\Integer)
%         \cong \Integer/2\Integer,
%       \]
%       und für den anderen Quotienten gilt
%       \begin{align*}
%                 G_2 / N_2
%         &=      (\Integer/2\Integer \oplus \Integer/2\Integer) / (\Integer/2\Integer \oplus 0)
%         \\
%         &\cong  ((\Integer/2\Integer)/(\Integer/2\Integer)) \oplus ((\Integer/2\Integer)/0)
%         \cong   0 \oplus \Integer/2\Integer
%         \cong   \Integer/2\Integer.
%       \end{align*}
%       Also gilt auch $G_1/N_1 \cong G_2/N_2$.
%       Es gilt aber $G_1 \ncong G_2$, da $G_1$ ein Element der Ordnung $4$ enthält, $G_2$ jedoch nicht.
%   \end{enumerate}
% \end{solution}
% 
% 
% 
% \begin{question}[subtitle = Gruppen mit trivialer Automorphismengruppe]
%   Es sei $G$ eine Gruppe mit $\Aut(G) = 1$.
%   \begin{enumerate}
%     \item
%       Zeigen Sie, dass $G$ abelsch ist.
%     \item
%       Zeigen Sie, dass $g = -g$ für alle $g \in G$.
%     \item
%       Folgern Sie, dass es eine eindeutige $\Field_2$-Vektorraumstruktur auf $G$ gibt.
%     \item
%       Folgern Sie, dass $G = 0$ oder $G \cong \Integer / 2 \Integer$.
%   \end{enumerate}
% \end{question}
% 
% 
% \begin{solution}
%   \begin{enumerate}
%     \item
%       Für $g \in G$ sei $c_g \colon G \to G$ die Konjugation mit $g$.
%       Dies ist ein Automorphismus von $G$, weshalb $c_g = \id_G$.
%       Somit ist $g \in \Center(G)$.
%     \item
%       Wegen der Kommutativität von $G$ ist die Abbildung $n \colon G \to G$, $g \mapsto -g$ ein Automorphismus von $G$.
%       Somit ist $n = \id_G$, also $-g = g$ für alle $g \in G$.
%     \item
%       Nach dem vorherigen Aufgabenteil ist $2 g = 0$ für alle $g \in G$.
%       Deshalb gibt es eine eindeutige $\Field_2$-Vektorraumstruktur auf $G$ via
%       \[
%         \overline{n} \cdot g = n \cdot g
%         \quad
%         \text{für alle $n \in \Integer$, $g \in G$},
%       \]
%       wie sich durch direktes Nachrechnen ergibt.
%     \item
%       Es sei $(b_i)_{i \in I}$ eine Basis von $G$ als $\Field_2$-Vektorraum.
%       Ist $G \neq 0$ und $G \ncong \Integer/2$, so ist $\dim_{\Field_2} G \geq 2$.
%       Es gibt daher $i_1, i_2 \in I$ with $i_1 \neq i_2$.
%       Die Permutation
%       \[
%         \sigma \colon \{b_i\}_{i \in I} \to \{b_i\}_{i \in I},
%         \quad
%         b_j
%         \mapsto
%         \begin{cases}
%           b_{i_2} & \text{falls $j = i_1$}, \\
%           b_{i_1} & \text{falls $j = i_2$}, \\
%           b_j     & \text{sonst},
%         \end{cases}
%       \]
%       induziert einen nicht-trivialen $\Field_2$-Vek\-tor\-raum\-auto\-mor\-phis\-mus $\alpha \colon G \to G$ mit
%       \[
%           \alpha\left( \sum_{i \in I} \lambda_i b_i \right)
%         = \sum_{i \in I} \lambda_i b_{\sigma(i)}.
%       \]
%       Dann ist $\alpha$ aber insbesondere ein nicht-trivialer Gruppenautomorphismus, im Widerspruch zu $\Aut(G) = 1$.
%   \end{enumerate}
% \end{solution}





