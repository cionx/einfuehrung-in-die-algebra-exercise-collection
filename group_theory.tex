\section{Gruppentheorie}


\begin{question}[subtitle = Wahr oder Falsch?]
  \begin{enumerate}
    \item
      Für jeden Körper $K$ und jedes $n \geq 1$ ist $C_n(K) \coloneqq \{x \in K \mid x^n = 1\}$ eine zyklische Untergruppe von $K^\times$.
    \item
      Ist $G$ ein Gruppe und $(N_i)_{i \in I}$ eine Familie normaler Untergruppen $N_i \subseteq G$, so ist auch $\bigcap_{i \in I} N_i \subseteq G$ normal.
    \item
      Die Gruppe $\Integer/2 \times \Integer/3$ ist zyklisch.
    \item
      Jede abelsche Gruppe ist auflösbar.
    \item
      Jede Gruppe der Ordnung $60$ ist abelsch.
    \item
      Jede Gruppe der Ordnung $101$ ist abelsch.
    \item
      Ist $G$ eine Gruppe und $N \subseteq G$ eine normale Untergruppe, so dass $N$ und $G/N$ abelsch sind, so ist auch $G$ abelsch.
  \end{enumerate}
\end{question}


\begin{solution}
  \begin{enumerate}
    \item
      Die Aussage ist wahr:
      Dass $C_n(K)$ eine Untergruppe von $K^\times$ ist, ergibt sich durch direktes Nachrechnen;
      alternativ erkennt man, dass die Abbildung $K^\times \to K^\times$, $x \mapsto x^n$ ein Gruppenhomomorphismus ist, und $C_n(K)$ ihr Kern ist.
      Die Gruppe $C_n(K)$ ist endlich, da sie die Nullstellenmenge des Polynoms $X^n - 1 \in K[X]$ ist.
      Als endliche Untergruppe der multiplikativen Gruppe eines Körpers ist $C_n(K)$ zyklisch.
    \item
      Die Aussage ist wahr:
      Für jedes $g \in G$ gilt nach Annahme $g N_i g^{-1} \subseteq N_i$ für alle $i \in I$;
      somit gilt auch $g \left( \bigcap_{i \in I} N_i \right) g^{-1} \subseteq \bigcap_{i \in I} (g N_i g^{-1}) \subseteq \bigcap_{i \in I} N_i$.
    \item
      Die Aussage ist wahr:
      Nach dem chinesischen Restklassensatz gilt $\Integer/2 \times \Integer/3 \cong \Integer/6$, und $\Integer/6$ ist zyklisch.
    \item
      Die Aussage ist wahr:
      Ist $G$ eine abelsche Gruppe, so ist $0 \subseteq G$ eine normale Untergruppe, so dass $G/0 \cong G$ abelsch ist.
      Die Kette $0 \subseteq G$ leistet also das Gewünschte.
    \item
      Die Aussage ist falsch:
      Die alternierende Gruppe $A_5 \subseteq S_5$ hat Ordnung $60$, ist aber nicht abelsch.
    \item
      Die Aussage ist wahr:
      Jede Gruppe der Ordnung $101$ ist zyklisch (siehe Übung~\ref{question: basic calculatings with orders}) und somit abelsch (siehe Übung~\ref{question: classification of cyclic groups}).
    \item
      Die Aussage ist falsch:
      Die Untergruppe $A_3 \subseteq S_3$ ist normal, und da $|A_3| = 3$ und $|S_3/A_3| = |S_3|/|A_3| = 2$ gelten, sind $A_3 \cong \Integer/3$ und $S_3/A_3 \cong \Integer/2$ abelsch.
      Aber $S_3$ ist nicht abelsch.
  \end{enumerate}
\end{solution}


\begin{question}[subtitle = Klassifikation zyklischer Gruppen]
  Es sei $G$ eine zyklische Gruppe, d.h.\ es gebe $g \in G$ mit $G = \generate{g}$.
  \begin{enumerate}
    \item
      Zeigen Sie, dass $G$ abelsch ist.
    \item
      Zeigen Sie, dass $G \cong \Integer/n$ für eine eindeutiges $n \in \Integer$ mit $n \geq 0$.
  \end{enumerate}
\end{question}


\begin{solution}
  \label{question: classification of cyclic groups}
  Nach Annahme gibt es ein $g \in G$ mit $G = \generate{g} = \{g^n \mid n \in \Integer\}$.
  \begin{enumerate}
    \item
      Für $x, y \in G$ gibt es $a, b \in \Integer$ mit $x = g^a$ und $y = g^b$, und deshalb gilt
      \[
        xy = g^a g^b = g^{a+b} = g^{b+a} = g^b g^a = yx.
      \]
      
    \item
      Die Abbildung $\varphi \colon \Integer \to G$, $n \mapsto g^n$ ist ein Gruppenhomomorphismus, und nach Annahme surjektiv.
      Für $n \in \Integer$ mit $n \geq 0$ und $\ker \varphi = (n)$ induziert deshalb $\varphi$ einen Isomorphismus
      \[
        \overline{\varphi} \colon \Integer/n \to G,
        \quad
        \overline{n} \mapsto g^n.
      \]
      Ist $G$ unendlich, so muss $n = 0$ gelten;
      ist $G$ endlich, so muss $n = |\Integer/n| = |G|$ gelten.
      Das zeigt die Eindeutigkeit von $n$.
  \end{enumerate}
\end{solution}


\begin{question}[subtitle = Zur Definition von Untergruppen]
  Es sei $G$ eine Gruppe und $H \subseteq G$ eine nichtleere Teilmenge.
  \begin{enumerate}
    \item
      Zeigen Sie, dass $H$ genau dann eine Untergruppe ist, wenn für alle $x, y \in G$ auch $x y^{-1} \in G$ gilt.
    \item
      Für alle $x, y \in H$ gelte $xy \in H$, und $H$ sei zusammen mit der (Einschränkung der) Multiplikation von $G$ ebenfalls eine Gruppe.
      Zeigen Sie, dass $H$ bereits eine Untergruppe von $G$ ist.
  \end{enumerate}
\end{question}


\begin{solution}
  \begin{enumerate}
    \item
      Da $H$ nichtleer ist, gibt es ein Element $x \in H$.
      Deshalb $1 = x x^{-1} \in H$.
      Für jedes $x \in H$ gilt somit auch $x^{-1} = 1 \cdot x^{-1} \in H$.
      Für alle $x, y \in H$ gilt dann $y^{-1} \in H$, und somit auch $xy = x (y^{-1})^{-1} \in H$.
      Ingesamt zeigt dies, dass $H$ eine Untergruppe von $G$ ist.
    \item
      Es sei $1_H \in H$ das neutrale Element von $H$, und es sei $x \in G$ das in $G$ inverse Element zu $1_H$.
      Es gilt $1_H^2 = 1_H$, da $1_H$ neutral in $H$ ist, und somit
      \[
          1_H
        = 1_H 1_G
        = 1_H 1_H x
        = 1_H^2 x
        = 1_H x
        = 1_G \,.
      \]
      Also besitzen $H$ und $G$ das gleiche neutrale Element.
      (Hier nutzen wir, dass in einer Gruppe $K$ die Identität $1_K$ das einzige Element $k \in K$ mit $k^2 = k$ ist.)
      Ist $x \in H$ und $y \in H$ das zu $x$ in $H$ inverse Element, so gilt deshalb $xy = 1_H = 1_G$, weshalb $y$ auch in $G$ invers zu $x$ ist.
      
      Zusammen zeigt dies, dass $H$ bereits eine Untergruppe von $G$ ist.
  \end{enumerate}
\end{solution}



\begin{question}[subtitle = Untergruppen von endlichen Gruppen]
  Es sei $G$ eine endliche Gruppe.
  Zeigen Sie, dass eine nichtleere Teilmenge $H \subseteq G$ genau dann eine Untergruppe ist, wenn $xy \in H$ für alle $x,y \in H$ gilt.
\end{question}


\begin{solution}
  Ist $H$ eine Untergruppe, so gilt per Definition für alle $x, y \in H$ auch $xy \in H$.
  
  Andererseits sei nun $H \subseteq G$ eine entsprechende Teilmenge.
  Es gibt ein Element $h \in H$, da $H$ nicht leer ist.
  Da $G$ endlich ist, hat $h$ endliche Ordnung, d.h.\ es gibt $n \geq 1$ mit $h^n = 1$.
  Inbesondere gilt $1 = h^n \in H$.
  Ist $h \neq 1$, so gilt außerdem $n \geq 2$, und somit auch $h^{-1} = h^{n-1} \in H$.
  Also ist $H$ eine Untergruppe.
\end{solution}


\begin{question}[subtitle = Zur Definition von Gruppen]
  Es sei $G$ eine nichtleere Menge zusammen mit einer assoziativen binären Verknüpfung $G \times G \to G$, $(g_1, g_2) \mapsto g_1 g_2$.
  Zeigen Sie, dass die folgenden Bedingungen äquivalent sind:
  \begin{enumerate}[label=\roman*)]
    \item
      \label{enum: its a group}
      Es handelt sich bei $G$ um eine Gruppe.
    \item
      \label{enum: equations are uniquely solvable}
      Für alle $a, b \in G$ sind die Gleichungen $ax = b$ und $xa = b$ jeweils eindeutig lösbar.
    \item
      \label{enum: equations are solvable}
      Für alle $a, b \in G$ sind die Gleichungen $ax = b$ und $xa = b$ jeweils lösbar.
  \end{enumerate}
\end{question}


\begin{solution}
  (\ref{enum: its a group} $\implies$ \ref{enum: equations are uniquely solvable})
  Ist $G$ eine Gruppe und sind $a, b \in G$, so gilt für alle $x \in G$, dass genau dann $ax = b$ gilt, wenn $x = a^{-1} b$ gilt, und genau dann $xa = b$ gilt, wenn $x = b^{-1} a$ gilt.
  Also sind beide Gleichungen eindeutig lösbar.
  
  (\ref{enum: equations are solvable} $\implies$ \ref{enum: equations are uniquely solvable})
  Andererseits seien nun die Gleichungen $ax = b$ und $xa = b$ für alle $a, b \in G$ eindeutig lösbar.
  Es gibt ein Element $a \in G$, da $G$ nicht leer ist.
  Nach Annahme gibt es ein Element $e \in G$ mit $ea = a$.
  Für jedes $b \in G$ gilt dann ebenfalls $eb = b$, denn nach Annahme gibt es ein $c \in G$ mit $ac = b$, und somit gilt
  \[
      eb
    = eac
    = ac
    = b\,.
  \]
  Das zeigt, dass $e$ linksneutral in $G$ ist.
  Für jedes $a \in G$ gibt es nun nach Annahme ein $b \in G$ mit $ba = e$, also eine Linksinverses zu $a$.
\end{solution}


\begin{question}[subtitle = Innere Automorphismen]
  \label{question: inner automorphims}
  Es sei $G$ eine Gruppe.
  \begin{enumerate}
    \item
      Zeigen Sie, dass für jedes $g \in G$ die Abbildung $c_g \colon G \to G$, $h \mapsto g h g^{-1}$ ein Gruppenautomorphismus ist.
    \item
      Zeigen Sie, dass die Abbildung $c \colon G \to G$, $g \mapsto c_g$ ein Gruppenhomomorphismus ist.
    \item
      Zeigen Sie, dass $\ker c = \Center(G)$ gilt.
    \item
      Zeigen Sie, dass $\Inner G \coloneqq \im c$ eine normale Untergruppe von $\Aut G$ ist.
  \end{enumerate}
  Man bezeichnet $\Inner G$ als die Gruppe der \emph{inneren Automorphismen} von $G$.
\end{question}


\begin{solution}
  \begin{enumerate}
    \item
      Für alle $h_1, h_2 \in G$ gilt
      \[
          c_g(h_1 h_2)
        = g h_1 h_2 g^{-1}
        = g h_1 g^{-1} g h_2 g^{-1}
        = c_g(h_1) c_g(h_2),
      \]
      also ist $c_g$ ein Gruppenhomomorphismus.
      Für alle $h \in G$ gilt
      \[
          c_g( c_{g^{-1}}(h) )
        = g g^{-1} h g g^{-1}
        = h
        = g^{-1} g h g^{-1} g
        = c_{g^{-1}}( c_g(h) ),
      \]
      also ist $c_g$ bijektiv mit $c_g^{-1} = c_{g^{-1}}$.
    \item
      Für alle $g_1, g_2 \in G$ gilt
      \[
          c_{g_1 g_2}(h)
        = (g_1 g_2) h (g_1 g_2)^{-1}
        = g_1 g_2 h g_2^{-1} g_1^{-1}
        = c_{g_1}( c_{g_2}(h) )
        \qquad
        \text{für alle $h \in G$}
      \]
      und somit $c_{g_1 g_2} = c_{g_1} c_{g_2}$.
    \item
      Für $g \in G$ gilt
      \begin{align*}
              g \in \ker c
        &\iff c_g = \id_G
         \iff \forall h \in G: c_g(h) = h
        \\
        &\iff \forall h \in G: ghg^{-1} = h
         \iff \forall h \in G: gh = hg
         \iff g \in \Center(G).
      \end{align*}
    \item
      Da $c$ ein Gruppenhomomorphismus ist, ist $\Inner G$ ein Untergruppe von $\Aut G$.
      Für jedes $\phi \in \Aut G$ und jedes $g \in G$ gilt $\phi c_g \phi^{-1} = c_{\phi(g)}$, denn für alle $h \in G$ gilt
      \[
        (\phi c_g \phi^{-1})(h)
        = \phi( c_g( \phi^{-1}(h) ) )
        = \phi(g \phi^{-1}(h) g^{-1})
        = \phi(g) h \phi(g)^{-1}
        = c_{\phi(g)}(h).
      \]
      Folglich ist $\phi \, {\Inner G} \, \phi^{-1} \subseteq \Inner G$ für alle $\phi \in \Aut G$, also $\Inner G$ normal in $\Aut G$.
  \end{enumerate}
\end{solution}


\begin{question}[subtitle = Ein Erzeugendensystem von $S_n$]
  Zeigen Sie, dass die symmetrische Gruppe $S_n$ von dem Zykel $\rho = (1 \, 2 \, \dotso \, n)$ und der Permutation $\tau = (1 \, 2)$ erzeugt wird.
\end{question}


\begin{solution}
  Für alle $i = 1, \dotsc, n$ gilt $\rho^{i-1}(1) = i$ und $\rho^{i-1}(2) = i+1$, und somit
  \[
    \rho^{i-1} \, \tau \, (\rho^{i-1})^{-1}
    = \rho^{i-1} (1 \quad 2) (\rho^{i-1})^{-1}
    = (\rho^{i-1}(1) \quad \rho^{i-1}(2))
    = (i \quad i+1).
  \]
  Deshalb gilt $(1 \quad 2), (2 \quad 3), \dotsc, (n-1 \quad n) \in \generate{\rho, \tau}$.
  Da $S_n$ von diesen Elementartranspositionen erzeugt wird, gilt bereits $S_n \subseteq \generate{\rho, \tau} \subseteq S_n$, und somit $S_n = \generate{\rho, \tau}$.
\end{solution}


\begin{question}[subtitle = Zur Ordnung]
  \label{question: basic calculatings with orders}
  Es sei $G$ eine endliche Gruppe.
  \begin{enumerate}
    \item
      Es seien $H, K \subseteq G$ zwei Untergruppen, so dass $|H|$ und $|K|$ teilerfremd sind.
      Zeigen Sie, dass $H \cap K = 1$.
    \item
      Es sei $N \subseteq G$ eine Untergruppe, so dass $H$ die einzige Untergruppe von Ordnung $|N|$ ist.
      Zeigen Sie, dass $N$ normal in $G$ ist.
    \item
      Es sei $|G|$ prim.
      Zeigen Sie, dass $G$ zyklisch ist.
    \item
      Es sei $N \subseteq G$ eine normale Untergruppe, so dass $|N|$ und $[G : N]$ teilerfremd sind.
      Zeigen Sie, dass $N$ die einzige Untergruppe von $G$ von Ordnung $|N|$ ist.
  \end{enumerate}
\end{question}


\begin{solution}
  \begin{enumerate}
    \item
      Der Schnitt $H \cap K$ ist sowohl von $H$ als auch von $K$ eine Untergruppe.
      Deshalb gilt $|H \cap K| \mid |H|$ und $|H \cap K| \mid |K|$.
      Da $|H|$ und $|K|$ teilerfremd sind, gilt bereits $|H \cap K| = 1$ und somit $H \cap K = 1$.
      
    \item
      Da für $g \in G$ ist die Abbildung $c_g \colon G \to G$, $x \mapsto gxg^{-1}$ ein Gruppenautomorphismus ist (siehe Übung~\ref{question: inner automorphims}), ist auch $c_g(H) = gHg^{-1}$ eine Untergruppe von $G$ von Ordnung $|H|$.
      Aus der Eindeutigkeit von $H$ bezüglich dieser Eigenschaft folgt, dass bereits $g H g^{-1} = H$ gilt.
      Da dies für jedes $g \in G$ gilt, ist $H$ normal.
    
    \item
      Es sei $g \in G$ mit $g \neq 1$.
      Dann ist $\generate{g} = \{g^n \mid n \in \Integer\}$ eine Untergruppe von $G$, weshalb $|{\generate{g}}| \mid G$ gilt.
      Da $|G|$ prim ist, gilt also entweder $|{\generate{g}}| = 1$ und somit $\generate{g} = 1$ oder $|{\generate{g}}| = G$ und somit $\generate{g} = G$.
      Da $1 \neq g \in \generate{g}$ gilt, kann der erste Fall ausgeschlossen werden.
    
    \item
      Es sei $\pi \colon G \to G/N$ die kanonische Projektion und $H \subseteq G$ eine Untergruppe von Ordnung $|N|$.
      Dann ist $\pi(H) \subseteq G/N$ eine Untergruppe, weshalb $|\pi(H)| \mid |G/N| = [G : N]$ gilt.
      Außerdem gilt $|H/(H \cap N)| = [H : H \cap N] \mid |H| = |N|$.
      
      Aus $\ker \pi|_H = H \cap \ker \pi = H \cap N$ erhalten wir, dass $\pi(H) \cong H / \ker \pi|_H \cong H / (H \cap N)$ gilt.
      Somit erhalten wir, dass $\pi(H)$ sowohl $[G : N]$ als auch $|N|$ teilt.
      Da $[G : N]$ und $|N|$ teilerfremd sind folgt hieraus, dass $\pi(H) = 1$ gilt.
      Es gilt also $H \subseteq \ker \pi = N$, und wegen $|H| = |N|$ somit bereits $H = N$.
  \end{enumerate}
\end{solution}


\begin{question}[subtitle = Ein Kriterium für maximale Untergruppen]
  Es sei $G$ ein Gruppe und $H \subseteq G$ eine Untergruppe, so dass $[G : H]$ endlich und prim ist.
  Zeigen Sie, dass $H$ eine maximale echte Untergruppe von $G$ ist. Entscheiden Sie, ob $H$ notwendigerweise normal in $G$ ist.
\end{question}


\begin{solution}
  Es sei $p \coloneqq [G : H]$.
  Da $p$ eine Primzahl ist gilt inbesondere $p \neq 1$, weshalb $H$ eine echte Untergruppe von $G$ ist.
  Ist $K \subsetneq G$ eine echte Untergruppe von $G$ mit $H \subseteq K$, so gilt wegen der Multiplikativität des Index’, dass 
  \[
      p
    = [G : H]
    = [G : K] [K : H].
  \]
  Da $p$ eine Primzahl ist, gilt entweder $[G : K] = p$ und $[K : H] = 1$, oder $[G : K] = 1$ und $[K : H] = p$.
  Es gilt $[G : K] > 1$, da $K$ eine echte Untergruppe von $G$ ist, und somit $[K : H] = 1$.
  Also ist $K = H$, und somit $H$ eine maximale echte Untergruppe.
  
  $H$ ist nicht notwendigerweise normal in $G$:
  Für $G = S_3$ und $H = \generate{(1\,2)} = \{\id, (1\,2)\}$ ist $H$ zwar nicht normal in $G$, aber $[G : H] = |G|/|H| = 6/2 = 3$ ist prim.
\end{solution}


\begin{question}
  Es sei $G$ eine Gruppe.
  Zeigen Sie, dass $G$ abelsch ist, wenn $G/{\Center(G)}$ zyklisch ist.
\end{question}


\begin{solution}
  Es sei $\class{g} \in G/{\Center(G)}$ ein zyklischer Erzeuger.
  Für alle $x, y \in G$ gibt es dann $n, m \in \Integer$ mit $\class{x} = \class{g}^m = \class{g^m}$ und $y = \class{g}^n = \class{g^n}$.
  Es gibt dann $x', y' \in \Center(G)$ mit $x = g^m x'$ und $y = g^n y'$.
  Die Element $x', y', g^n, g^m$ kommutieren paarweise miteinander, weshalb auch $x$ und $y$ kommutieren.
\end{solution}


\begin{question}[subtitle = Grundbegriffe der Gruppenwirkungen]
  Es sei $G$ eine Gruppe, die auf einer Menge $X$ vermöge $G \times X \to X$, $(g,x) \mapsto g.x$ wirkt.
  \begin{enumerate}
    \item
      Definieren Sie die Bahn $G.x$ und den Stabilisator $G_x$ eines Elementes $x \in X$.
    \item
      Zeigen Sie, dass $G_x$ für alle $x \in X$ eine Untergruppe von $G$ ist.
    \item
      Konstruieren Sie für jedes $x \in X$ eine Bijektion $G/G_x \to G.x$.
    \item
      Es seien $x, y \in X$ zwei Elemente mit gleicher $G$-Bahn.
      Zeigen Sie, dass die Stabilisatoren $G_x$ und $G_y$ konjugiert zueinander sind.
    \item
      Entscheiden Sie, ob auch die Umkerung der obigen Aussage notwendigerweise gilt.
    \item
      Zeigen Sie, dass $X$ die disjunkte Vereinigung der $G$-Bahnen ist.
  \end{enumerate}
\end{question}


\begin{solution}
  \begin{enumerate}
    \item
      Es sind $G.x = \{g.x \mid g \in G\}$ und $G_x = \{g \in G \mid g.x = x\}$.
    \item
      Es gilt $1 \in G_x$ da $1.x = x$.
      Für $g_1, g_2 \in G_x$ gilt $(g_1 g_2).x = g_1.(g_2.x) = g_1.x = x$ und somit auch $g_1 g_2 \in G_x$.
      Für $g \in G$ gilt $g^{-1}.x = g^{-1}.(g.x) = (g^{-1}.g).x = 1.x = x$ und somit auch $g^{-1} \in G_x$.
      Ingesamt zeigt dies, dass $G_x$ ein Untergruppe von $G$ ist.
    \item
      Die Abbildung $f \colon G \to G.x$, $g \mapsto g.x$ ist surjektiv, und für $g_1, g_2 \in G$ gilt
      \begin{align*}
              f(g_1) = f(g_2)
        &\iff g_1.x = g_2.x
        \iff  g_2^{-1}.g_1.x = x
        \\
        &\iff (g_2^{-1} g_1).x = x
        \iff  g_2^{-1} g_1 \in G_x
        \iff  g_1 G_x = g_2 G_x,
      \end{align*}
      weshalb $f$ durch eine wohldefinierte Bijektion $\overline{f} \colon G/G_x \to G.x$, $\overline{g} \mapsto g.x$ faktorisiert.
    \item
      Haben $x$ und $y$ die Gleiche $G$-Bahn, so gibt es $g \in G$ mit $y = g^{-1}.x$.
      Für alle $h \in G$ gilt dann
      \begin{align*}
              h \in G_y
        &\iff h.y = y
        \iff  h.g^{-1}.x = g^{-1}.x
        \\
        &\iff g.h.g^{-1}.x = x
        \iff  (g h g^{-1}).x = x
        \iff  g h g^{-1} \in G_x.
      \end{align*}
      Wegen der Bijektivität der Konjugationsabbildung $G \to G$, $h \mapsto ghg^{-1}$ folgt, dass $g G_y g^{-1} = G_x$ gilt.
    \item
      Die Umkehrung gilt nicht:
      Gilt etwa $G = 1$, so gilt $G_x = G$ für alle $x \in X$, aber alle Bahnen sind einelementig.
      Für $|X| \geq 2$ ergibt dies ein Gegenbeispiel.
      
      Allgemeiner kann man eine beliebige Gruppe $G$ auf einer Menge $X$ mit $|X| \geq 2$ trivial wirken lassen, d.h.\ es gelte $g.x = x$ für alle $g \in G$ und $x \in X$.
      Dann gilt $G_x = G$ für alle $x \in X$ aber alle Bahnen sind einelementig.
    \item
      Es genügt zu zeigen, dass $x \sim y \iff x \in G.y$ eine Äquivalenzrelation auf $X$ definiert, denn dann sind die $G$-Bahnen genau die Äquivalenzklassen von ${\sim}$.
      Da $x = 1.x \in G.x$ ist die Relation reflexiv.
      Gilt $x \sim y$ so gibt es $g \in G$ mit $x = g.y$;
      dann gilt auch $y = g^{-1}.x \in G.x$ und somit $y \sim x$.
      Für $x, y, z \in X$ mit $x \sim y$ und $y \sim z$ gibt es $g, h \in G$ mit $x = g.y$ und $y = h.z$;
      dann gilt auch $x = g.y = g.h.z = (gh).z \in G.z$ und somit $x \sim z$.
  \end{enumerate}
\end{solution}


\begin{question}[subtitle = Bahnenkombinatorik]
  \begin{enumerate}
    \item
      Es sei $G$ eine Gruppe der Ordnung $77$, die auf einer $17$-elementigen Menge $X$ wirkt.
      Zeigen Sie, dass die Wirkung mindestens $3$ Fixpunkte hat.
      (Ein Element $x \in X$ ist ein Fixpunkt falls $g.x = x$ für alle $g \in G$.)
    \item
      Es sei $p > 0$ prim und $G$ eine endliche $p$-Gruppe.
      Es sei $X$ eine endliche Menge mit $p \nmid |X|$ und $G$ wirke auf $X$.
      Zeigen Sie, dass die Wirkung einen Fixpunkt besitzt.
  \end{enumerate}
  (\emph{Hinweis}:
   Nutzen Sie jeweils die Bahnengleichung.)
\end{question}


\begin{solution}
  \begin{enumerate}
    \item
      Ist $B \in X/G$ eine $G$-Bahn und $x \in X$ mit $B = G.x$, so gilt $|B| = |G.x| = [G : G_x] \mid |G|$.
      Für die Ordnung von $G$ gilt $|G| = 77 = 7 \cdot 11$, also gilt $|B| \in \{1, 7, 11, 77\}$ für jede $G$-Bahn $B \in X/G$.
      Dabei gilt genau dann $|B| = 1$, falls $B = \{x\}$ für einen Fixpunkt $x \in X$ gilt;
      es gilt also zu zeigen, dass es mindestens drei einelementige $G$-Bahnen in $X$ gibt.
      
      Nach der Bahnengleichung gilt $17 = |X| = \sum_{B \in X/G} |B|$.
      Die einzigen Möglichkeiten, die Zahl $17$ als Summe der Zahlen $1$, $7$, $11$ und $77$ darzustellen, sind
      \[
          17
        = 11 + 6 \cdot 1
        = 2 \cdot 7 + 3 \cdot 1
        = 7 + 10 \cdot 1
        = 17 \cdot 1.
      \]
      In jeder der Möglichkeiten kommt der Summand $1$ mindestens dreimal vor, wodurch sich die Aussage ergibt.
      
    \item
      Es sei $B \in X/G$ eine $G$-Bahn.
      Es gilt genau dann $|B| = 1$, wenn $B = \{x\}$ für einen Fixpunkt $x \in X$ gilt.
      Ist $|B| \neq 1$ und $x \in B$, so folgt aus $|B| = G.x = [G : G_x] \mid |G| = p^r$ mit $r \geq 0$, dass bereits $p \mid |B|$ gilt.
      
      Aus der Bahnengleichung erhalten wir, dass $|X| = \sum_{B \in X/G} |B|$.
      Aus $p \nmid |X|$ erhalten wir, dass es eine $G$-Bahn $B \in X/G$ mit $p \nmid |B|$ gibt.
      Nach den obigen Beobachtungen erhalten wir, dass diese Bahn von der Form $B = \{x\}$ für einen Fixpunkt $x \in X$ ist.
  \end{enumerate}
\end{solution}


\begin{question}[subtitle = Mehr Bahnenkombinatorik]
  Es sei $p > 0$ prim und $G$ eine endliche $p$-Gruppe mit $G \neq 1$.
  Zeigen Sie, dass auch $\Center(G) \neq 1$ gilt.
  \\
  (\emph{Hinweis}:
   Lassen sich $G$ auf sich selbst durch Konjugation wirken.)
\end{question}


\begin{solution}
  Es sei $X \coloneqq G$.
  Die Gruppe $G$ wirkt auf $X$ durch Konjugation, also durch $g.x = g x g^{-1}$ für alle $g \in G$, $x \in X$
  Ein Punkt $x \in X$ ist genau dann ein Fixpunkt, wenn $g x g^{-1} = x$ für alle $g \in G$ gilt, wenn also $x \in \Center(G)$ gilt.
  
  Ist $B \in X/G$ eine $G$-Bahn mit $|B| \neq 1$, so gibt es $x \in X$ mit $B = G.x$, und es folgt aus $|B| = |G.x| = [G : G_x] \mid |G|$, dass bereits $p \mid |B|$ gilt.
  Damit erhalten wir für $x \in X$, dass genau dann $x \in \Center(G)$ gilt, wenn $p \nmid |G.x|$.
  Da $|G| = p^r$ mit $r \geq 1$ gilt, erhalten wir aus der Bahnengleichung, dass
  \[
            0
    \equiv  |G|
    =       \sum_{B \in X/G} |B|
    \equiv  \sum_{\substack{B \in X/G \\ |B| = 1}} |B|
    =       |{\Center(G)}|
    \mod    p.
  \]
  Deshalb gilt $p \mid |{\Center(G)}|$.
  Insbesondere ist $|{\Center(G)}| \neq 1$ und somit $\Center(G) \neq 1$.
\end{solution}


\begin{question}[subtitle = Die oberen $2 \times 2$-Matrizen]
  Es sei $K$ ein Körper.
  \begin{enumerate}
    \item
      Zeigen Sie, dass
      \[
                  \Borel_2(K)
        \coloneqq \left\{
                    \begin{pmatrix}
                      a_1 & b   \\
                      0   & a_2
                    \end{pmatrix}
                  \,\middle|\,
                    a_1, a_2 \in K^\times,
                    b \in K
                  \right\}.
      \]
      eine Untergruppe von $\GL_2(K)$ ist.
    \item
      Zeigen Sie, dass $\Borel_2(K)$ nicht normal in $\GL_2(K)$ ist.
    \item
      Zeigen Sie, dass
      \[
                  \Upper_2(K)
        \coloneqq \left\{
                    \begin{pmatrix}
                      1 & b \\
                      0 & 1
                    \end{pmatrix}
                  \,\middle|\,
                    b \in K
                  \right\}.
      \]
      eine normale Untergruppe von $\Borel_2(K)$ ist, und dass $\Borel_2(K)/{\Upper_2(K)} \cong K^\times \times K^\times$ gilt.
    \item
      Entscheiden Sie, ob $\Borel_2(K) \cong \Upper_2(K) \times K^\times \times K^\times$ gilt.
  \end{enumerate}
\end{question}


\begin{solution}
  \begin{enumerate}
    \item
      Die Einheitsmatrix liegt in $\Borel_2(K)$, da $1 \in K^\times$.
      Es gilt
      \begin{equation}
        \label{equation: product of upper triangular matrices}
        \begin{pmatrix}
          a_1 & b   \\
          0   & a_2
        \end{pmatrix}
        \begin{pmatrix}
          a'_1  & b'   \\
          0     & a'_2
        \end{pmatrix}
        =
        \begin{pmatrix}
          a_1 a'_1  & a_1 b' + b a'_2 \\
          0         & a_2 a'_2
        \end{pmatrix}
      \end{equation}
      mit $a_1 a'_1, a_2 a'_2 \in K^\times$ falls $a_1, a'_1, a_2, a'_2 \in K^\times$;
      deshalb ist $\Borel_2(K)$ abgeschlossen unter Multiplikation ist.
      Für $\begin{psmallmatrix} a_1 & b \\ 0 & a_2 \end{psmallmatrix} \in \Borel_2(K)$ gilt auch
      \[
        \begin{pmatrix}
          a_1 & b   \\
          0   & a_2
        \end{pmatrix}^{-1}
        =
        \frac{1}{a_1 a_2}
        \begin{pmatrix}
          a_2 &           -b    \\
          0   & \phantom{-}a_1
        \end{pmatrix}
        \in \Borel_2(K),
      \]
      also ist $\Borel_2(K)$ auch unter Inversion abgeschlossen.
    
    \item
      Es gilt
      \[
        \begin{pmatrix}
          0 & 1 \\
          1 & 0
        \end{pmatrix}
        \begin{pmatrix}
          1 & 1 \\
          0 & 1
        \end{pmatrix}
        \begin{pmatrix}
          0 & 1 \\
          1 & 0
        \end{pmatrix}
        =
        \begin{pmatrix}
          1 & 0 \\
          1 & 1
        \end{pmatrix}
        \notin \Borel_2(K).
      \]
      
    \item
      Aus \eqref{equation: product of upper triangular matrices} ergibt sich, dass die Abbildung
      \[
                \varphi
        \colon  \Borel_2(K) \to K^\times \times K^\times,
        \quad   \begin{pmatrix}
                  a_1 & b   \\
                  0   & a_2
                \end{pmatrix}
                \mapsto
                (a_1, a_2)
      \]
      ein surjektiver Gruppenhomomorphismus ist, und es gilt $\ker \varphi = \Upper_2(K)$.
      Folglich ist $\Upper_2(K)$ normal in $\Borel_2(K)$ und $\Borel_2(K) / \Upper_2(K) \cong \im \varphi = K^\times \times K^\times$.
    
    \item
      Für $K = \Field_2$ sind die Gruppen isomorph:
      Es gilt $\Field_2^\times = 1$ und somit auch $\Field_2^\times \times \Field_2^\times \cong 1$, sowie $\Borel_2(\Field_2) = \Upper_2(\Field_2)$.
      Deshalb gilt
      \[
              \Borel_2(\Field_2)
        =     \Upper_2(\Field_2)
        \cong \Upper_2(\Field_2) \times \Field_2^\times \times \Field_2^\times.
      \]
      
      Für $K \neq \Field_2$ sind die Gruppen nicht isomorph:
      Die Gruppe $\Borel_2(K)$ ist dann nicht abelsch, denn es gibt $a_1, a_2 \in K^\times$ mit $a_1 \neq a_2$, weshalb auch
      \[
        \begin{pmatrix}
          a_1 & 0   \\
          0   & a_2
        \end{pmatrix}
        \begin{pmatrix}
          1 & 1 \\
          0 & 1
        \end{pmatrix}
        =
        \begin{pmatrix}
          a_1 & a_1 \\
          0   & a_2
        \end{pmatrix}
        \neq
        \begin{pmatrix}
          a_1 & a_2 \\
          0   & a_2
        \end{pmatrix}
        =
        \begin{pmatrix}
          1 & 1 \\
          0 & 1
        \end{pmatrix}
        \begin{pmatrix}
          a_1 & 0   \\
          0   & a_2
        \end{pmatrix}.
      \]
      Die Gruppe $\Upper_2(K)$, und damit auch die Gruppe $\Upper_2(K) \times K^\times \times K^\times$, ist allerdings abelsch, denn für alle $b_1, b_2 \in K$ gilt
      \[
        \begin{pmatrix}
          1 & b_1 \\
          0 & 1
        \end{pmatrix}
        \begin{pmatrix}
          1 & b_2 \\
          0 & 1
        \end{pmatrix}
        =
        \begin{pmatrix}
          1 & b_1 + b_2 \\
          0 & 1
        \end{pmatrix}
        =
        \begin{pmatrix}
          1 & b_2 + b_1 \\
          0 & 1
        \end{pmatrix}
        =
        \begin{pmatrix}
          1 & b_2 \\
          0 & 1
        \end{pmatrix}
        \begin{pmatrix}
          1 & b_1 \\
          0 & 1
        \end{pmatrix}.
      \]
  \end{enumerate}
\end{solution}


\begin{question}[subtitle = Vereinigung von Untergruppen]
  Es sei $G$ eine Gruppe.
  \begin{enumerate}
    \item
      Es seien $H, H_1, H_2 \subseteq G$ Untergruppen mit $H \subseteq H_1 \cup H_2$,
      Zeigen Sie dass bereits $H \subseteq H_1$ oder $H \subseteq H_2$ gilt.
    \item
      Folgern Sie:
      Sind $H_1, H_2 \subseteq G$ seien zwei Untergruppen, so ist $H_1 \cup H_2$ genau dann eine Untergruppe ist, wenn $H_1 \subseteq H_2$ oder $H_2 \subseteq H_1$ gilt.
    \item
      Geben Sie ein Beispiel für eine eine Gruppe $G$ und Untergruppen $H_1, H_2, H_3 \subseteq G$ an, so dass zwar $H_i \nsubseteq H_j$ für alle $i \neq j$, aber $H_1 \cup H_2 \cup H_3$ eine Untergruppe von $G$ ist.
  \end{enumerate}
\end{question}


\begin{solution}
  \begin{enumerate}
    \item
      Würde $H \nsubseteq H_2$ und $H \nsubseteq H_1$ gelten, so gebe es $h_1, h_2 \in H$ mit $h_1 \notin H_2$ und $h_2 \notin H_1$.
      Da $h_1, h_2 \in H \subseteq H_1 \cup H_2$ gilt, müsste allerdings $h_1 \in H_1$ und $h_2 \in H_2$ gelten.
      Für das Produkt $h_1 h_2$ würde dann $h_1 h_2 \notin H_1$ gelten, denn sonst wäre $h_2 = h_1^{-1} h_1 h_2 \in H_1$, im Widerspruch zur Wahl von $h_1$.
      Analog ergebe sich aber auch, dass $h_1 h_2 \notin H_2$ gilt.
      Es müsste aber $h_1 h_2 \in H \subseteq H_1 \cup H_2$ gelten, da $H$ ein Untergruppe ist.
      
    \item
      Gilt $H_1 \subseteq H_2$ oder $H_2 \subseteq H_1$, so gilt $H_1 \cup H_2 = H_2$ oder $H_1 \cup H_2 = H_1$, weshalb $H_1 \cup H_2$ dann eine Untergruppe ist.
      
      Ist andererseits $H_1 \cup H_2$ eine Untergruppe, so ergibt sich aus den vorherigen Aussagenteil mit $H = H_1 \cup H_2$ dass bereits $H_1 \cup H_2 \subseteq H_1$ oder $H_1 \cup H_2 \subseteq H_2$ gilt, und somit $H_2 \subseteq H_1$ oder $H_1 \subseteq H_2$.
      
    \item
      Es sei $G = \Integer/2 \oplus \Integer/2$ und es seien
      \begin{align*}
        H_1 &= \generate{ (1,0) } = \{ (0,0), (1,0) \},
        \\
        H_2 &= \generate{ (1,1) } = \{ (0,0), (1,1) \},
        \\
        H_3 &= \generate{ (0,1) } = \{ (0,0), (0,1) \}.
      \end{align*}
      Dann gilt $H_i \nsubseteq H_j$ für alle $1 \leq i \neq j \leq 3$, aber $H_1 \cup H_2 \cup H_3 = G$.
  \end{enumerate}
\end{solution}


\begin{question}[subtitle = Produkte von Normalteilern und auflösbaren Gruppen]
  Es seien $G_1$ und $G_2$ zwei Gruppen.
  \begin{enumerate}
    \item
      Es seien $N_1 \subseteq G_1$ und $N_2 \subseteq G_2$ zwei normale Untergruppen.
      Zeigen Sie, dass auch $N_1 \times N_2 \subseteq G_1 \times G_2$ eine normale Untergruppe ist, und dass
      \[
        (G_1 \times G_2)/(N_1 \times N_2) \cong (G_1/N_1) \times (G_2/N_2)
      \]
      gilt.
    \item
      Folgern Sie, dass $G_1 \times G_2$ auflösbar ist, wenn $G_1$ und $G_2$ auflösbar sind.
  \end{enumerate}
\end{question}


\begin{solution}
  \begin{enumerate}
    \item
      Die kanonischen Projektionen $\pi_i \colon G_i \to G_i/N_i$, $g \mapsto \overline{g}$ induzieren einen Gruppenhomomorphismus
      \[
        \pi
        \coloneqq
                \pi_1 \times \pi_2
        \colon  G_1 \times G_2
        \to     (G_1/N_1) \times (G_2/N_2),
        \quad
                (g_1, g_2)
        \mapsto (\overline{g_1}, \overline{g_2}).
      \]
      Da $\pi_1$ und $\pi_2$ surjektiv sind, ist es auch $\pi$.
      Außerdem gilt
      \[
          \ker \pi
        = \ker \pi_1 \times \ker \pi_2
        = N_1 \times N_2
      \]
      Somit ist $N_1 \times N_2$ eine normale Untergruppe von $G_1 \times G_2$, und $\pi$ induziert einen Isomorphismus
      \[
                \overline{\pi}
        \colon  (G_1 \times G_2)/(N_1 \times N_2)
        \to     (G_1/N_1) \times (G_2/N_2),
        \quad
                \overline{(g_1, g_2)}
        \mapsto (\overline{g_1}, \overline{g_2}).
      \]
      
    \item
      Da $G_1$ auflösbar ist gibt es eine Kette von Untergruppen
      \[
                  1
        =         N_0
        \subseteq N_1
        \subseteq N_2
        \subseteq N_3
        \subseteq \dotsb
        \subseteq N_{s-1}
        \subseteq N_s
        =         G_1,
      \]
      so dass für alle $i = 1, \dotsc, s$ die Untergruppe $N_{i-1}$ normal in $N_i$ ist, und der Quotient $N_i/N_{i-1}$ abelsch ist.
      Aus der Auflösbarkeit von $G_2$ ergibt sich analog eine Kette von Untergruppen
      \[
                  1
        =         K_0
        \subseteq K_1
        \subseteq K_2
        \subseteq K_3
        \subseteq \dotsb
        \subseteq K_{t-1}
        \subseteq K_t
        =         G_1,
      \]
      so dass für alle $j = 1, \dotsc, t$ die Untergruppe $K_{j-1}$ normal in $K_j$ ist, und der Quotient $K_j/K_{j-1}$ abelsch ist.
      Zusammen erhalten wir damit eine Kette von Untergruppen
      \[
                  1       \times 1
        =         N_0     \times 1
        \subseteq \dotsb
        \subseteq N_s     \times 1
        =         G_1     \times K_0
        \subseteq \dotsb
        \subseteq G_1     \times K_t
        =         G_1     \times G_2.
      \]
  \end{enumerate}
  Nach dem vorherigen Aussagenteil ist in dieser Kette jede Untergruppe normal in der jeweils nächsten Untergruppe.
  Die Quotienten sind abelsch, denn für alle $i = 1, \dotsc, s$ gilt
  \[
          (N_i \times 1)/(N_{i-1} \times 1)
    \cong (N_i/N_{i-1}) \times (1 / 1)
    \cong (N_i/N_{i-1}) \times 1
    \cong N_i/N_{i-1}
  \]
  und für alle $j = 1, \dotsc, t$ gilt
  \[
          (G_1 \times K_j)/(G_1 \times K_{j-1})
    \cong (G_1/G_1) \times (K_j/K_{j-1})
    \cong 1 \times (K_j/K_{j-1})
    \cong K_i/K_{j-1}.
  \]
\end{solution}


\begin{question}[subtitle = Erste Sylowschritte]
  \begin{enumerate}
    \item
      Es sei $G$ ein endliche Gruppe und $p$ eine Primzahl.
      Zeigen Sie, dass eine $p$-Sylow\-un\-ter\-grup\-pe $H$ von $G$ genau dann normal ist, wenn sie die einzige $p$-Sylowuntergruppe von $G$ ist.
    \item
      Zeigen Sie, dass jede Gruppe der Ordnung $35$ einen Normalteiler der Ordnung $5$ besitzt.
    \item
      Zeigen Sie, dass jede Gruppe der Ordnung $279$ einen Normalteiler der Ordnung $9$ besitzt.
    \item
      Zeigen Sie, dass es keine einfachen Gruppen der Ordnung $275$ gibt.
    \item
      Zeigen Sie, dass jede Gruppe der Ordnung $21$ auflösbar ist.
    \item
      Zeigen Sie, dass jede Gruppe der Ordnung $22$ auflösbar ist.
    \item
      Es seien $p$ und $q$ zwei verschiedene Primzahlen.
      Zeigen Sie, dass jede Gruppe der Ordnung $pq$ auflösbar ist.
  \end{enumerate}
\end{question}


\begin{solution}
  \begin{enumerate}
    \item
      Für jedes $g \in G$ ist auch $g H g^{-1}$ eine $p$-Sylowuntergruppe von $G$.
      Ist $H$ die einzige $p$-Sylowuntergruppe von $G$, so ist deshalb $gHg^{-1} = H$ für alle $g \in G$, also $H$ normal in $G$.
      
      Ist andererseits $H$ normal in $G$ und $H'$ eine weitere $p$-Sylowuntergruppe von $G$, so gibt es nach den Sylowsätzen ein $g \in G$ mit $H' = g H g^{-1}$.
      Da $H$ normal ist, gilt dabei bereits $g H g^{-1} = H$ und somit $H' = H$.
  \end{enumerate}
  Wir betrachten im folgenden jeweils eine beliebige Gruppe $G$ der gegebenen Ordnung.
  \begin{enumerate}[resume]
    \item
      Es gilt $35 = 5 \cdot 7$.
      Bezeichnet $n_5$ die Anzahl der $5$-Sylowgruppen in $G$, so gilt nach den Sylowsätzen, dass $n_5 \equiv 1 \pmod{5}$ und $n_5 \mid 7$.
      Da $7$ prim ist, folgt, dass $n_5 = 1$.
      Die eindeutige $5$-Sylowgruppe $H$ von $G$ ist von Ordnung $5$ und nach dem ersten Aussagenteil normal.
      
    \item
      Es gilt $279 = 3^2 \cdot 31$.
      Bezeichnet $n_3$ die Anzahl der $3$-Sylowgruppen in $G$, so gilt nach den Sylowsätzen, dass $n_3 \equiv 1 \pmod{3}$ und $n_5 \mid 31$.
      Da $31$ prim ist, folgt, dass $n_3 = 1$.
      Die eindeutige $3$-Sylowgruppe $H$ von $G$ ist von Ordnung $9$ und nach dem ersten Aussagenteil normal.
      
    \item
      Es gilt $275 = 5^2 \cdot 11$.
      Bezeichnet $n_{11}$ die Anzahl der $11$-Sylowgruppen in $G$, so gilt nach den Sylowsätzen, dass $n_{11} \equiv 1 \pmod{11}$ und $n_{11} \mid 25$.
      Es folgt, dass $n_{11}  = 1$.
      Die eindeutige $11$-Sylowgruppe $H$ von $G$ ist von Ordnung $11$ und nach dem ersten Aussagenteil normal.
      Somit besitzt $G$ einen nicht-trivialen Normalteiler und ist somit nicht einfach.
      
    \item
      Es gilt $22 = 2 \cdot 11$.
      Bezeichnet $n_{11}$ die Anzahl der $11$-Sylowgruppen in $G$, so gilt nach den Sylowsätzen, dass $n_{11} \equiv 1 \pmod{11}$ und $n_{11} \mid 2$.
      Es folgt, dass $n_{11}  = 1$.
      Die eindeutige $11$-Sylowgruppe $H$ von $G$ ist von Ordnung $11$, also zyklisch (siehe Übung~\ref{question: basic calculatings with orders}) und somit abelsch (siehe Übung~\ref{question: classification of cyclic groups}).
      Nach gleicher Argumentation ist der Quotient $G/H$ mit $|G/H| = |G|/|H| = 2$ abelsch.
      Somit erfüllt die Kette $1 \subseteq H \subseteq G$ das Gewünschte.
      
    \item
      Es gilt $21 = 3 \cdot 7$.
      Bezeichnet $n_7$ die Anzahl der $7$-Sylowgruppen in $G$, so gilt nach den Sylowsätzen, dass $n_7 \equiv 1 \pmod{7}$ und $n_7 \mid 3$.
      Es folgt, dass $n_7  = 1$.
      Die eindeutige $7$-Sylowgruppe $H$ von $G$ ist von Ordnung $7$, also zyklisch und somit abelsch.
      Nach gleicher Argumentation ist der Quotient $G/H$ mit $|G/H| = |G|/|H| = 3$ abelsch.
      Somit erfüllt die Kette $1 \subseteq H \subseteq G$ das Gewünschte.
      
    \item
      Wir können o.B.d.A.\ davon ausgehen, dass $p < q$ gilt.
      Bezeichnet $n_q$ die Anzahl der $q$-Sylowgruppen in $G$, so gilt nach den Sylowsätzen, dass $n_q \equiv 1 \pmod{q}$ und $n_q \mid p$.
      Es folgt, dass $n_q  = 1$, da $1 + q > p$.
      Die eindeutige $q$-Sylowgruppe $H$ von $G$ ist von Ordnung $q$, also zyklisch und somit abelsch.
      Nach gleicher Argumentation ist der Quotient $G/H$ mit $|G/H| = |G|/|H| = p$ abelsch.
      Somit erfüllt die Kette $1 \subseteq H \subseteq G$ das Gewünschte.
  \end{enumerate}
\end{solution}


\begin{question}[subtitle = Quadrate in endlichen Körpern]
  Es sei $p > 0$ prim, $n$ eine positive natürliche Zahl und
  \[
    \Quad(p,n) = \{x^2 \mid x \in \Field_{p^n}\}
  \]
  die Menge der Quadrate in $\Field_{p^n}$.
  \begin{enumerate}
    \item
      Bestimmen Sie die Anzahl der Elemente von $\Quad(p,n)$ in Abhängigkeit von $p$.
    \item
      Entscheiden Sie, für welche $p$ und $n$ die Menge $\Quad(p,n)$ eine Untergruppe der additiven Gruppe von $\Field_{p^n}$ ist.
    \item
      Zeigen Sie dass $xy \in \Quad(p,n)$ für alle $x, y \in \Field_{p^n}$ mit $x, y \notin \Quad(p,n)$ gilt.
      (Das Produkt zweier Nicht-Quadrate ist also ein Quadrat.)
  \end{enumerate}
\end{question}


\begin{solution}
  \begin{enumerate}
    \item
      Im Fall $p = 2$ ist der Frobenius-Homomorphismus $\Field_{2^n} \to \Field_{2^n}$, $x \mapsto x^2$ ein Automorphismus und somit $\Quad(2,n) = \Field_{p^n}$.
      Inbesondere gilt dann $|{\Quad(2,n)}| = |\Field_{2^n}| = 2^n$.
      
      Im Fall $p \neq 2$ gilt
      \[
          \Quad(p,n)
        = \{ x^2 \mid x \in \Field_{p^n} \}
        = \{0\} \cup \{x^2 \mid x \in \Field_{p^n}^\times\}
        = \{0\} \cup \im q
      \]
      für den Gruppenhomomorphismus $q \colon \Field_{p^n}^\times \to \Field_{p^n}^\times$, $x \mapsto x^2$.
      Es gilt
      \[
          \ker q
        = \{x \in \Field_{p^n}^\times \mid x^2 = 1\}
        = \{1,-1\}
      \]
      und somit $|{\ker q}| = 2$ (hier nutzen wir, dass $\ringchar \Field_{p^n} = p \neq 2$ und somit $1 \neq -1$).
      Deshalb gilt $|{\im q}| = [\Field_{p^n}^\times : \ker q] = |\Field_{p^n}^\times|/|\ker q| = (p^n - 1)/2$.
      Ingesamt gilt somit, dass
      \[
          |{\Quad(p,n)}| =
          \begin{cases}
            2^n                   & \text{falls $p = 2$},     \\
            \frac{p^n - 1}{2} + 1 & \text{falls $p \neq 2$}.
          \end{cases}
      \]
      
    \item
      Im Falle $p = 2$ gilt, wie bereits zuvor gesehen, dass $\Quad(2,n) = \Field_{2^n}$, weshalb es sich um eine Untergruppe handelt.
      Im Falle $p \neq 2$ gilt $(p^n - 1)/2 + 1 \nmid p^n$, denn sonst würde
      \begin{align*}
            &\, \left. \frac{p^n - 1}{2} + 1 \,\middle|\, p^n \right.
        \iff    \frac{p^n - 1}{2} + 1 \equiv 0 \pmod{p^n}
        \\
        \iff&\, \frac{p^n - 1}{2} \equiv - 1 \pmod{p^n}
        \iff    p^n - 1 \equiv -2 \pmod{p^n}
        \\
        \iff&\, -1 \equiv -2 \pmod{p^n}
        \iff    1 \equiv 0  \pmod{p^n}
        \iff    p^n \mid 1
      \end{align*}
      gelten.
      Somit  ist $|{\Quad(p,n)}|$ in diesem Fall kein Teiler von $|\Field_{p^n}|$, und deshalb $\Quad(p,n)$ keine Untergruppe der additiven Gruppe von $\Field_{p^n}$.
      
    \item
      Für $x, y \notin \Quad(p,n)$ gelten inbesondere $x, y \neq 0$ und somit $x, y \in \Field_{p^n}^\times$.
      Die Gruppe $\Field_{p^n}^\times$ zyklisch, da $\Field_{p^n}$ ein endlicher Körper ist;
      es sei $g \in \Field_{p^n}^\times$ ein Erzeuger.
      Dann gibt es $a, b \in \Natural$ mit $x = g^a$ und $y = g^b$;
      da $x$ und $y$ keine Quadrate sind, müssen $a$ and $b$ ungerade sein (denn sonst wäre beispielsweise $x = g^a = (g^{a/2})^2$).
      Dann ist $a+b$ gerade und somit $xy = g^a g^b = g^{a+b} = (g^{(a+b)/2})^2$ ein Quadrat.
  \end{enumerate}
\end{solution}
