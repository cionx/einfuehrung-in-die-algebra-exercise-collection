\section{Ringtheorie}


% TODO:
% Calculations:
%   greatest common divisor
%   calculating inverses in quotiens via gcd
%   simplification of quotiens
%   isomorphism class of cokernels
%   solving equations with the chinese reminder theorem
%   which of these rings is a field/pid/noetherian
% Theorems:
%   Laurant series as localization of power series
%   K[countable many variables] is factorial


\begin{question}[subtitle = Multiple Choice]
  Entscheiden Sie, welche der folgenden Aussagen wahr oder falsch sind.
  \begin{enumerate}
    \item
      Jeder Körper ist faktoriell.
    \item
      Die beiden Projektionen $\pi_1, \pi_2 \colon \Integer \times \Integer \to \Integer$ mit $\pi_1(x,y) = x$ und $\pi_2(x,y) = y$ für alle $(x,y) \in \Integer \times \Integer$ sind die einzigen beiden Ringhomomorphismen $\Integer \times \Integer \to \Integer$.
    \item
      Für alle Ringe $R_1$ und $R_2$ gilt $(R_1 \times R_2)[X] \cong R_1[X] \times R_2[X]$.
    \item
      Jeder faktorielle Ring ist unendlich.
    \item
      Für jeden kommutativen Ring $R$ gibt es einen Integritätsbereich $S$, so dass $R \cong S/I$ für ein Ideal $I \subseteq S$.
    \item
      Jeder endliche Integritätsbereich ist ein Hauptidealring.
    \item
      Ist $R$ ein endlicher kommutativer Ring, so ist $R[X,Y]$ noethersch.
    \item
      Ist $R$ ein Hauptidealring, so ist auch $R[X]$ ein Hauptidealring.
    \item
      Ist $R$ ein Integritätsbereich und $p \in R$ prim, so ist $p$ irreduzibel.
    \item
      Sind $n_1, n_2, n_3 \in \Integer$ mit $\ggT(n_1, n_2, n_3) = 1$, so hat das Gleichungssystem
      \[
        \left\{
          \begin{array}{ccc}
            x &\equiv&  a_1 \mod n_1,
            \\
            x &\equiv&  a_2 \mod n_2,
            \\
            x &\equiv&  a_3 \mod n_3,
          \end{array}
        \right.
      \]
      für alle $a_1, a_2, a_3 \in \Integer$ eine Lösung.
    \item
      Ist $R$ ein endlicher kommutativer Ring mit $p \coloneqq \ringchar R > 0$ prim, so ist die Abbildung $R \to R$, $x \mapsto x^p$ ein Ringautomorphismus.
  \end{enumerate}
\end{question}


\begin{solution}
  \begin{enumerate}
    \item
      Die Aussage ist wahr:
      Ein Körper ist ein Hauptidealring, und somit faktoriell (Körper sind sogar schon euklidsch).
      Die Aussage lässt sich auch konkret zeigen:
      Ist $K$ ein Körper, so gibt es keine Primelement in $K$, die leere Menge $\emptyset \eqqcolon \mathcal{P} \subseteq K$ ist also ein Repräsentantensystem der Primeleme von $K$.
      Jedes Element $x \in K$ mit $x \neq 0$ lässt sich nun eindeutig als $x = x \cdot \prod_{p \in \mathcal{P}} p$ darstellen, denn es gilt $x \in K^\times$ und $\prod_{p \in \mathcal{P}} = 1$ ist das leere Produkt.
    \item
      Die Aussage ist wahr:
      Ist $\varphi \colon \Integer \times \Integer \to \Integer$ ein Ringhomomorphismus, so gilt für die Elemente $e_1 \coloneqq \varphi(1,0)$ und $e_2 \coloneqq \varphi(0,1)$, dass $e_1^2 = e_1$ und $e_2^2 = e_2$.
      Deshalb gilt $e_1, e_2 \in \{0,1\}$.
      Da außerdem $1 = \varphi(1,1) = \varphi(1,0) + \varphi(0,1) = e_1 + e_2$ gilt, muss entweder $e_1 = 1$ und $e_2 = 0$, oder $e_1 = 0$ und $e_2 = 1$.
      Im ersten Fall gilt $\varphi = \pi_1$, im zweiten Fall gilt $\varphi = \pi_2$.
      
      Die Aussage lässt sich auch durch Untersuchen von $\ker \varphi$ zeigen:
      Es ist $\im \varphi$ ein Unterring von $\Integer$, also bereits $\im \varphi = \Integer$.
      Also induziert $\varphi$ einen Isomorphismus $(\Integer \times \Integer)/\ker \varphi \to \Integer$.
      Deshalb ist $\ker \varphi$ ein Primideal, aber kein maximales Ideal in $\Integer \times \Integer$.
      Somit ist $\ker \varphi = \Integer \times \pideal$ oder $\ker \varphi = \pideal \times \Integer$ für ein Primideal $\pideal \subseteq \Integer$ (siehe Übung~\ref{question: product of ideals is an ideal} und Übung~\ref{questions: ideals in products are products of ideal}); wäre dabei $\pideal$ maximal, so wäre dies auch $\ker \varphi$ (siehe Übung~\ref{question: product of ideals is an ideal}), also kommt nur $\pideal = 0$ in Frage.
      Im Fall $\ker \varphi = 0 \times \Integer$ gilt $\varphi = \pi_1$ und im Fall $\ker \varphi = \Integer \times 0$ gilt $\varphi = \pi_2$.
    \item
      Die Aussage ist wahr:
      Die kanonischen Projektionen $\pi_i \colon R_1 \times R_2 \to R_i$, $(x_1, x_2) \mapsto x_i$ induzieren Ringhomomorphismen
      \[
                \pi_i[X]
        \colon  (R_1 \times R_2)[X] \to R_i[X],
        \quad
                \sum_j \left( x^{(1)}_j, x^{(2)}_j \right) X^j
        \mapsto \sum_j x^{(i)}_j X^j
      \]
      die in einen Ringhomorphismus
      \begin{align*}
                                                    \varphi
         \colon                                     (R_1 \times R_2)[X]
        &\xlongrightarrow{\pi_1[X] \times \pi_2[X]} R_1[X] \times R_2[X],
        \\
                                                    \sum_j (a_j, b_j) X^j
        &\mapsto                                    \left( \sum_j a_j X^j, \sum_j b_j X^j \right)
      \end{align*}
      resultieren.
      Die Bijektivität von $\varphi$ ergibt sich durch direktes Hinsehen.
    \item
      Die Aussage ist falsch:
      Jeder Körper ist ein faktorieller Ring, aber es gibt endliche Körper.
    \item
      Die Aussage ist wahr:
      Der Ring $S \coloneqq \Integer[T_x \mid x \in R]$ ist ein Integritätsbereich, und der Einsetzhomomorphismus $\varphi \colon S \to R$ mit $\varphi(T_x) = x$ für alle $r \in R$ ist surjektiv, induziert also einen Isomorphismus $\overline{\varphi} \colon S / \ker \varphi \to R$.
    \item
      Die Aussage ist wahr:
      Jeder endliche Integritätsbereich ist bereits ein Körper.
    \item
      Die Aussage ist wahr:
      $R$ ist noethersch, und nach iterierter Anwendung des Hilbertschen Basissatzes somit auch $R[X][Y] \cong R[X,Y]$.
    \item
      Die Aussage ist falsch:
      Ist $K$ ein Körper, so ist zwar $R \coloneqq K[X]$ ein Hauptidealring, aber $R[Y] \cong K[X,Y]$ nicht (siehe Übung~\ref{question: examples for non principal and not finitely generated modules}).
      Allgemeiner ist $R[X]$ genau dann ein Hauptidealring, wenn $R$ bereits ein Körper ist (siehe Übung~\ref{question: fields are the only rings for which the polynomial ring is a pid}).
    \item
      Die Aussage ist wahr:
      Es seien $a, b \in R$ mit $p = ab$.
      Da $p$ prim ist, gilt $p \mid a$ oder $p \mid b$;
      wir können o.B.d.A.\ davon ausgehen, dass $p \mid a$.
      Dann gibt es $c \in R$ mit $a = pc$ und es folgt $p = ab = pcb$.
      Da $R$ ein Integritätsbereich ist, und $p \neq 0$ gilt (denn $p$ ist prim) folgt, dass bereits $1 = cb$ gilt.
      Also ist $b$ eine Einheit.
    \item
      Die Aussage ist falsch:
      Man betrachte etwa $n_1 = n_2 = 2$ und $n_3 = 3$ (dann sind $n_1, n_2, n_3$ zwar insgesamt teilerfremd, nicht jedoch paarweise, weshalb sich der chinesische Restklassensatz nicht anwenden lässt).
      Hätte das Gleichungssystem immer ein Lösung, so wäre der Ringhomomorphismus
      \[
                \varphi
        \colon  \Integer \to \Integer/2 \times \Integer/2 \times \Integer/3,
        \quad   x
        \mapsto (\overline{x}, \overline{x}, \overline{x})
      \]
      surjektiv.
      Dies ist aber nicht der Fall, denn für alle $(a_1, a_2, a_3) \in \im \varphi$ gilt $a_1 = a_2$.
    \item
      Die Aussage ist falsch:
      Die Abbildung $\sigma$ ist ein Ringhomomorphismus (siehe Übung~\ref{question: the frobenius homomorphism}), aber betrachtet man etwa den Fall $R = \Field_p[X]/(X^p)$, so gilt $\sigma(\overline{X}) = \overline{X}^p = \overline{X^p} = 0$ und somit $\ker \sigma = \{x \in R \mid x^p = 0\} \neq 0$.
      (Besitzt $R$ keine nichttrivialen nilpotenten Elemente, etwa falls $R$ ein Integritätsbereich oder sogar ein Körper ist, so ist $\sigma$ hingegen ein Automorphismus:
      Dann gilt $\ker \sigma = 0$, und wegen der Endlichkeit von $R$ ist $\sigma$ damit schon bijektiv.)
  \end{enumerate}
\end{solution}


\begin{question}[subtitle = Größte gemeinsame Teiler]
  Bestimmen Sie jeweils einen größten gemeinsamen Teiler und drücken sie diesen als Linearkombination der jeweiligen Elemente aus.
  \begin{enumerate}
    \item
      \begin{enumerate}
        \item
          $54, 24 \in \Integer$
        \item
          $213, 168 \in \Integer$
        \item
          $45, 63, 105 \in \Integer$
        \item
          $70, 63, 42 \in \Integer$
        \item
          $105, 70, 42, 30 \in \Integer$
      \end{enumerate}
    \item
      \begin{enumerate}
        \item
          $t^2 + t - 2, t^2 - 3 t + 2 \in \Rational[t]$
        \item
          $t^2 + 3, t^2 - 3 t + 2 \in \Rational[t]$
        \item
          $t^3 - t^2 + t - 1, t^3 - 3 t^2 + 4 t - 2 \in \Rational[t]$
        \item
          $t^4 - t^2 - 2 t - 1, t^3 - 1 \in \Rational[t]$
        \item
          $t^3 - t^2 + t - 1, t^2 - 1, t^3 + t^2 + t + 1 \in \Rational[t]$
      \end{enumerate}
  \end{enumerate}
\end{question}


% TODO: Adding a solution.


\begin{question}
  Entscheiden Sie, ob die folgenden Polynome jeweils irreduzibel sind:
  \begin{enumerate}
    \item
      $f(X) \coloneqq (X - 3)^2 + 1 \in \Rational[X]$.
    \item
      $f(X) \coloneqq 2X^3 - 14X + 2 \in \Rational[X]$.
    \item
      $f(X) \coloneqq 2X^3 - 14X + 2 \in \Integer[X]$.
    \item
      $f(X) \coloneqq X^3 - 18 X^2 + 6 X + 3 \in \Rational[X]$.
    \item
      $f(X) \coloneqq X^3 - 18 X^2 + 6 X + 3 \in \Real[X]$.
    \item
      $f(X) \coloneqq X^5 + 15 X^2 + 6 X + 21 \in \Integer[X]$.
    \item
      $f(X) \coloneqq X^3 + 2 X^2 + X + 1 \in \Integer[X]$.
    \item
      $f(X) \coloneqq 2X^4 + 200 X^3 + 2000 X^2 + 20000 X + 20 \in \Rational[X]$.
    \item
      $f(X) \coloneqq  X^n - 2t \in K(t)[X]$ für einen Körper $K$ und $n \geq 1$.
    \item
      $f(X,Y) \coloneqq X Y^3 + X^2 Y + 5 X Y^2 + X^2 + 3 X Y + 2 X + Y + 2 \in \Rational[X]$.
    \item
      $f(X,Y) \coloneqq X^3 + Y^3 + X^2 Y + X Y^2 + XY + 6 X + 6 Y + 3 \in \Rational[X,Y]$.
    \item
      $f(X) \coloneqq X^{p-1} + X^{p-2} + \dotsb + X + 1 \in \Rational[X]$ für $p > 0$ prim.
    \item
      $f(X) \coloneqq X^n + X^{n-1} + \dotsb + X + 1 \in \Rational[X]$ für $n \geq 3$ ungerade.
  \end{enumerate}
\end{question}


\begin{solution}
  \begin{enumerate}
    \item
      Wir geben zwei Möglichkeiten an, um die Irreduziblität von $f$ zu zeigen:
      \begin{itemize}
        \item
          Es handelt sich um ein quadratisches Polynom ohne reellen, und damit auch ohne rationale Nullstellen;
          also ist $f$ irreduzibel.
        \item
          Alternativ ergibt sich durch Ausmultiplizieren, dass $f(X) = X^2 - 6X + 10$, und die Irreduziblität von $f$ ergibt sich aus Eisenstein mit $p = 2$.
      \end{itemize}
    \item
      Da $2 \in \Rational$ eine Einheit ist, dürfen wir $f$ durch $2$ teilen und stattdessen das Polynom $\tilde{f}(X) = X^3 - 7X + 1 \in \Rational[X]$ betrachten.
      Da es sich bei $\tilde{f}$ ein kubisches Polynom handelt, ist es genau dann irreduzibel, wenn es keine Nullstelle hat.
      Da $\tilde{f}$ normiert ist und bereits $\tilde{f} \in \Integer[X]$ gilt, ist jede Nullstelle von $\tilde{f}$ schon eine ganze Zahl.
      Da jede Nullstelle $n \in \Integer$ den konstanten Teil von $\tilde{f}$ teilen muss, kommen nur $1$ und $-1$ als mögliche Nullstellen in Frage. Durch direktes Ausprobieren können aber beide ausgeschlossen werden.
      Also hat $\tilde{f}$ keine Nullstelle und ist somit irreduzibel.
    \item
      Das Polynom ist nicht irreduzibel, da es in $f(X) = 2 \cdot (X^3 - 7X + 1)$ faktorisiert, wobei keiner der beiden Faktoren eine Einheit in $\Integer[X]$ ist.
    \item
      Reduzieren bezüglich $p = 3$ liefert $\tilde{f}(X) = X^3 - 4X - 1 = X^3 + 2X + 2 \in \Field_3[X]$.
      Durch direktes Ausprobieren ergibt sich, dass $\tilde{f}$ keine Nullstellen hat, und als kubisches Polynom somit irreduzibel ist.
      Somit ist auch $f$ irreduzibel.
    \item
      Das Polynom ist nach Eisenstein mit $p = 3$ irreduzibel.
    \item
      Das Polynom ist nicht irreduzibel, da es (als Polynom ungeraden Grades über $\Real$) eine Nullstelle hat, aber nicht linear ist.
    \item
      Die Irreduziblität ergibt sich nach Eisenstein mit $p = 3$.
    \item
      Wir geben zwei Möglichkeit an die Irreduziblität von $f$ zu zeigen.
      \begin{itemize}
        \item
          Es genügt zu zeigen, dass $f$ irreduzibel in $\Rational[X]$ ist.
          Als kubisches Polynom ist $f$ genau dann irreduzibel in $\Rational[X]$, wenn es über $\Rational$ keine Nullstelle hat.
          Da $f$ normiert ist, muss jede rationale Nullstelle von $f$ bereits eine ganze Zahl sein.
          Es genügt also zu zeigen, dass $f$ keine ganzen Nullstellen hat.
          Jede ganze Nullstelle von $f$ muss den konstanten Teil von $f$, also $1$, teilen;
          es kommen somit nur $1$ und $-1$ in Frage.
          Durch Ausprobieren ergibt sich, dass keines von beiden eine Nullstelle ist.
          Also ist $f$ irreduzibel.
        \item
          Reduzieren bezüglich $p = 2$ ergibt das Polynom $\tilde{f}(X) = X^3 + X + 1 \in \Field_2[X]$.
          Dann hat $\tilde{f}$ keine Nullstellen und ist als kubisches Polynom deshalb irreduzibel.
          Somit ist auch $f$ schon irreduzibel.
      \end{itemize}
    \item
      Da $2 \in \Rational$ eine Einheit ist, dürfen wir $f$ durch $2$ teilen und somit stattdessen das Polynom $\tilde{f}(X) \coloneqq X^4 + 100 X^3 + 1000 X^2 + 10000 X + 10 \in \Rational[X]$ betrachten.
      Da $\tilde{f}$ normiert, und somit primitiv ist, ergibt sich die Irreduziblität von $\tilde{f}$ durch Eisenstein wahlweise mit $p = 2$ oder $p = 5$.
    \item
      Die Irreduziblität ergibt sich durch Eisenstein mit dem Primelement $t \in K[t]$.
    \item
      Wir betrachten das gegebene Polynom als
      \begin{align*}
            \tilde{f}(X)
        &=  X Y^3 + X^2 Y + 3 X Y^2 + X^2 + 3 X Y + 2 X + Y + 2
        \\
        &=  (Y + 1) X^2 + (Y^3 + 3 Y^2 + 3 Y + 2) X + (Y + 2)
        \in \Rational[Y][X]
      \end{align*}
      Da die Polynome $Y+1, Y + 2 \in \Rational[Y]$ teilerfremd sind, ist dieses Polynom primitiv.
      Außerdem gilt $(Y + 2) \mid (Y^3 + 3 Y^2 + 3 Y + 2)$, da $-2$ eine Nullstelle von $Y^3 + 3 Y^2 + 3 Y + 2$ ist.
      Es lässt sich also Eisenstein mit dem Primelement $Y + 2 \in \Rational[Y]$ anwenden, um die Irreduziblität von $\tilde{f}$ zu erhalten.
    \item
      Wir betrachen das gegeben Polynom als
      \begin{align*}
            \tilde{f}(Y)
        &=  X^3 + Y^3 + X^2 Y + X Y^2 + X Y + 6 X + 6 Y + 3
        \\
        &=  Y^3 + X Y^2 + (X^2 + X + 6) Y + (X^3 + 6X + 3)
        \in \Rational[X][Y]
      \end{align*}
      Da $\tilde{f}$ normiert ist können wir bezüglich $X \in \Rational[X]$ reduzieren, und erhalten
      \[
              \overline{f}(Y)
        =     Y^3 + 6 Y + 3
        \in   (\Rational[X]/(X))[Y]
        \cong \Rational[Y].
      \]
      Nach Eisenstein mit $p = 3$ ist $\overline{f}(Y)$ irreduzibel, also ist auch $\tilde{f}$, und somit $f$, irreduzibel.
    \item
      Es gilt $f(X) = X^{p-1} + \dotsb + X + 1 = (X^p - 1)/(X - 1)$ und somit
      \[
          f(X+1)
        = \frac{(X+1)^p-1}{X}
        = \frac{\sum_{k=0}^p \binom{p}{k} X^k - 1}{X}
        = \sum_{k=1}^p \binom{p}{k} X^{k-1}
        = \sum_{k=0}^{p-1} \binom{p}{k+1} X^k.
      \]
      Dabei gilt $p \mid \binom{p}{k+1}$ für alle $k = 0, \dotsc, p-1$ aber $p^2 \mid p = \binom{p}{1}$.
      Also ist das normierte Polynom $f(X+1)$ nach Eisenstein irreduzibel, und somit auch $f(X)$.
    \item
      Das Polynom ist nicht linear, hat aber $-1$ ein Nullstelle; es ist also reduzibel.
  \end{enumerate}
\end{solution}


 \begin{question}
  Bestimmen Sie, welche der folgenden Ringe isomorph zueinander sind, und welche nicht:
  \begin{gather*}
    \Integer/25,
    \quad
    \Field_{25},
    \quad
    \Field_5 \times \Field_5,
    \quad
    \Field_5[X]/(X^2),
    \\
    \Field_5[X]/(X^2+2),
    \quad
    \Field_5[X]/(X^2+4),
    \quad
    \Integer[X]/(5X).
  \end{gather*}
\end{question}


\begin{solution}
  Wir bezeichnen die Ring in der gegebenen Reihenfolge mit $R_1, \dotsc, R_7$.
  Wir zeigen, dass die Isomorphieklassen der gegebenen Moduln durch $\{R_1\}, \{R_2, R_5\}, \{R_3, R_6\}, \{R_4\}, \{R_7\}$ gegeben sind.
  
  Das Polynom $X^2 + 2 \in \Field_5[X]$ hat keine Nullstellen und ist deshalb irreduzibel (da quadratisch).
  Folglich ist $R_4$ eine quadratische Körpererweiterung von $\Field_5$, also $R_5 \cong \Field_{25} = R_2$.
  
  Das Polynom $X^2 + 4 \in \Field_5[X]$ zerfällt in $X^2 + 4 = X^2 - 1 = (X-1)(X+1)$.
  Nach dem chinesischen Restklassensatz gilt daher
  \[
          R_6
    =     \Field_5[X]/((X-1)(X+1))
    \cong \Field_5[X]/(X-1) \times \Field_5[X]/(X+1)
    \cong \Field_5 \times \Field_5
    =     R_3.
  \]
  
  Es bleibt zu zeigen, dass die Ring $R_1, R_2, R_3, R_4, R_7$ paarweise nicht isomorph sind.
  
  Während $R_1, \dotsc, R_6$ endlich sind (mit je $25$ Elementen) ist $R_7$ unendlich, denn nach dem dritten Isomorphiesatz gilt für das Ideal $(X)/(5X) \subseteq \Integer[X]/(5X)$, dass
  \[
           (\Integer[X]/(5X))/((X)/(5X))
    \cong \Integer[X]/(X)
    \cong \Integer.
  \]
  Somit ist $R_7$ zu keinem der anderen Ringe isomorph.
  
  Es bleibt zu zeigen, dass $R_1, R_2, R_3, R_4$ paarweisen nicht isomorph sind.
  Da $0 \neq \overline{5} \in R_1$ und $0 \neq \overline{X} \in R_4$ nilpotent sind, aber $R_2$ und $R_3$ außer $0$ keine nilpotenten Elemete enthalten, gelten $R_1, R_4 \ncong R_2, R_3$.
  
  Es bleibt zu zeigen, dass $R_1 \ncong R_4$ und $R_2 \ncong R_3$.
  Dass $R_2 \ncong R_3$ folgt daraus, dass $R_2$ ein Körper ist, $R_3$ aber nicht.
  Es gilt $R_4 \cong \Field_5^2$ als $\Field_5$-Vektorraum;
  die unterliegende abelsche Gruppe von $R_4$ ist deshalb $\Integer/5 \oplus \Integer/5$, die unterliegende abelsche Gruppe von $R_1$ ist aber $\Integer/25$ (und die beiden Gruppen sind nicht isomorph).
  Also gilt auch $R_1 \ncong R_4$.
\end{solution}


\begin{question}[subtitle = Initialobjekte in der Kategorie der Ringe]
  \label{qst: Z is inital}
  \begin{enumerate}
    \item
      Überzeugen Sie sich davon, dass es für jeden Ring $R$ genau einen Ringhomomorphismus $\Integer \to R$ gibt.
      (Dies bedeutet, dass $\Integer$ ein Initialobjekt in der Kategorie der Ringe ist.)
    \item
      Es sei $Z$ ein Ring, so dass es für jeden Ring $R$ einen eindeutigen Ringhomomorphismus $Z \to R$ gibt.
      Zeigen Sie, dass $Z \cong \Integer$.
  \end{enumerate}
\end{question}


\begin{solution}
  \begin{enumerate}
    \item
      Ist $\phi \colon \Integer \to R$ ein Ringhomomorphismus, so ist $\phi(1_\Integer) = 1_R$.
      Für alle $n \in \Integer$ ist damit
      \[
          \phi(n)
        = \phi(n \cdot 1_\Integer)
        = n \cdot \phi(1_\Integer)
        = n \cdot 1_R.
      \]
      Also ist $\phi$ eindeutig.
      Durch direktes Nachrechnen ergibt sich auch, dass $\psi \colon \Integer \to R$ mit
      \[
        \psi(n) \coloneqq n \cdot 1_R
        \quad
        \text{für alle $n \in \Integer$}
      \]
      ein Ringhomomorphismus ist.
    \item
      Es gibt einen eindeutigen Ringhomomorphismus $\phi \colon \Integer \to Z$ sowie einen eindeutigen Ringhomomorphismus $\psi \colon Z \to \Integer$.
      Es ist auch $\psi \circ \phi \colon \Integer \to \Integer$ ein Ringhomomorphismus.
      Die Identität $\id_\Integer \colon \Integer \to \Integer$ ist ebenfalls ein Ringhomomorphismus.
      Da es genau einen Ringhomomorphismus $\Integer \to \Integer$ gibt, muss sowohl $\psi \circ \phi$ als auch $\id_\Integer$  dieser eindeutige Ringhomomorphismus $\Integer \to \Integer$ sein.
      Folglich gilt $\psi \circ \phi = \id_\Integer$.
      Analog ergibt sich, dass auch $\phi \circ \psi = \id_Z$ gilt.
  \end{enumerate}
\end{solution}


\begin{question}
  Es sei $R$ ein Ring.
  Konstruieren Sie eine Bijektion zwischen der Menge der Ringhomomorphismen $\Integer[T] \to R$ und $R$.
\end{question}


\begin{solution}
  Aus der Vorlesung ist bekannt, dass die Abbildung
  \begin{align*}
              \{ \text{Ringhomomorphismen $\Integer[T] \to R$} \}
    &\to      \{ \text{Ringhomomorphismen $\Integer \to R$} \} \times R,
    \\
              \phi
    &\mapsto  (\phi|_\Integer, \phi(T))
  \end{align*}
  eine Bijektion ist.
  Da es genau einen Ringhomomorphismus $\Integer \to R$ gibt, ergibt sich ferner, dass die Abbildung
  \[
            \{ \text{Ringhomomorphismen $\Integer \to R$} \} \times R
    \to     R,
    \quad
            (\psi, r)
    \mapsto r
  \]
  eine Bijektion ist.
  Damit ergibt sich insgesamt eine Bijektion
  \[
            \{ \text{Ringhomomorphismen $\Integer[T] \to R$} \}
    \to     R,
    \quad
            \phi
    \mapsto \phi(T).
  \]
\end{solution}


\begin{question}[subtitle = Urbilder von Idealen]
  \label{question: preimages of ideals}
  Es seien $R$ und $S$ zwei kommutative Ringe und $\phi \colon R \to S$ ein Ringhomomorphismus.
  \begin{enumerate}
    \item
      Zeigen Sie, dass für jedes Ideal $\aideal \subseteq S$ das Urbild $\phi^{-1}(\aideal)$ ein Ideal in $R$ ist.
    \item
      Entscheiden Sie, ob $\phi^{-1}(\pideal)$ ein Primideal ist, wenn $\pideal \subseteq S$ ein Primideal ist.
    \item
      Entscheiden Sie, ob $\phi^{-1}(\mideal)$ ein maximales Ideal ist, wenn $\mideal \subseteq S$ ein maximales Ideal ist.
  \end{enumerate}
\end{question}


\begin{solution}
  \begin{enumerate}
    \item
      Es sei $\pi \colon S \to S/\aideal$, $s \mapsto \overline{s}$ die kanonische Projektion.
      Dann ist $\pi \phi$ ein Ringhomomorphismus und somit $\ker (\pi \phi) = \phi^{-1}(\ker \pi) = \phi^{-1}(\aideal)$ ein Ideal in $R$.
    \item
      Die Aussage gilt:
      Es sei $\pi \colon S \to S/\pideal$, $s \mapsto \overline{s}$ die kanonische Projektion und $\qideal \coloneqq \phi^{-1}(\pideal)$.
      Der Quotient $S/\pideal$ ist ein Integritätsbereich, da $\pideal$ ein Primideal ist.
      Nach dem vorherigen Aufgabenteil ist $\qideal$ ein Ideal in $R$, und da $\ker (\pi \phi) = \phi^{-1}(\ker \pi) = \phi^{-1}(\pideal) = \qideal$ induziert $\pi \phi$ einen injektiven Ringhomomorphismus
      \[
        \psi \colon R/\qideal \to S/\pideal
        \quad
        \overline{r} \mapsto \overline{\phi(r)}.
      \]
      Der Ring $\im (\pi \phi) \subseteq S/\pideal$ ist als Unterring eines Integritätsbereichs ebenfalls ein Integritätsbereich.
      Somit ist $R/\qideal \cong \im(\pi \phi)$ ein Integritätsbereich, also $\qideal$ ein Primideal.
    \item
      Die Aussage gilt nicht:
      Es sei etwa $\phi \colon \Integer \to \Rational$ die kanonische Inklusion.
      Dann ist $\mideal \coloneqq 0$ ein maximales Ideal in $\Rational$, aber $\phi^{-1}(0) = 0$ ist kein maximales Ideal in $\Integer$, da $\Integer/\mideal \cong \Integer$ kein Körper ist.
  \end{enumerate}
\end{solution}


\begin{question}
  \label{question: lattice isomorphism for quotients}
  Es sei $R$ ein kommutativer Ring und $I \subseteq R$ ein Ideal.
  Es sei $\pi \colon R \to R/I$, $x \mapsto \overline{x}$ die kanonische Projektion.
  \begin{enumerate}
    \item
      Zeigen Sie, dass
      \begin{align*}
                              \{ \text{Ideale $J \subseteq R$ mit $J \supseteq I$} \}
        &\longleftrightarrow  \{ \text{Ideale $K \subseteq R/I$} \},
        \\
                      J
        &\longmapsto  \pi(J) = J/I,
        \\
                        \pi^{-1}(K)
        &\longmapsfrom  K
      \end{align*}
      eine wohldefinierte Bijektion liefert.
    \item
      Zeigen Sie, dass sich die obige Bijektion sich zu Bijektion zwischen den jeweiligen Primidealen und maximalen Idealen einschränkt.
  \end{enumerate}
\end{question}


\begin{solution}
  \begin{enumerate}
    \item
      Für jedes Ideal $K \subseteq R/I$ ist das Urbild $\pi^{-1}(K) \subseteq R$ ebenfalls ein Ideal, denn Urbilder von Idealen unter Ringhomomorphismen sind ebenfalls Ideale (siehe Übung~\ref{question: preimages of ideals}).
      Aus $0 \subseteq K$ ergibt sich, dass dabei $I = \ker \pi = \pi^{-1}(0) \subseteq \pi^{-1}(K)$.
      
      Wegen der Surjektivität von $\pi$ ist für jedes Ideal $J \subseteq R$ auch $\pi(J) \subseteq R/I$ ein Ideal:
      Für $\overline{x}, \overline{y} \in \pi(J)$ kann $x, y \in J$ gewählt werden;
      dann ist auch $x + y \in J$ und somit $\overline{x} + \overline{y} = \overline{x + y} \in \pi(J)$.
      Für $\overline{x} \in \pi(J)$ und $\overline{r} \in R/I$ kann $x \in J$ gewählt werden;
      dann ist auch $r x \in J$ und somit $\overline{r} \overline{x} = \overline{r x} \in \pi(J)$.
      
      Das zeigt, dass die beiden Abbildungen wohldefiniert sind.
      
      Wegen der Surjektivität von $\pi$ gilt $\pi(\pi^{-1}(K)) = K$ für jede Teilmenge $K \subseteq R/I$, inbesondere also für die Ideale.
      
      Für jedes Ideal $J \subseteq R$ gilt $\pi^{-1}(\pi(J)) = J + I$:
      Es gilt $J \subseteq \pi^{-1}(\pi(J)) \subseteq J$ und wie bereits gezeigt auch $I \subseteq \pi^{-1}(\pi(J))$, und somit $I + J \subseteq \pi^{-1}(\pi(J))$.
      Ist andererseits $x \in \pi^{-1}(\pi(J))$, so gibt es $d \in J$ mit $\overline{x} = \overline{y}$.
      Dann ist $\overline{x-y} = \overline{x} - \overline{y} = 0$, somit $x - y \in I$ und deshalb $x = (x-y) + y \in I + J$.
      Gilt bereits $I \subseteq J$, so ist $I + J = J$ und somit $\pi^{-1}(\pi(J)) = J$.
      
      Das zeigt, dass beide Abbildungen invers zueinander sind.
    \item
      Wir bemerken zunächst, dass
      \[
          \pi(J)
        = \{ \overline{x} \mid x \in J \}
        = \{ x + I \mid x \in J \}
        = J/I
      \]
      für jedes Ideal $J \subseteq R$ mit $J \supseteq I$.
      Inbsbesondere gilt deshalb nach dem dritten Isomorphiesatz, dass $R/J \cong (R/I)/(J/I) \cong (R/I)/\pi(J)$.
      Es gilt somit
      \begin{align*}
              \text{$J$ ist prim}
        &\iff \text{$R/J$ ist ein Integritätsbereich}
        \\
        &\iff \text{$(R/I)/\pi(J)$ ist ein Integritätsbereich}
         \iff \text{$\pi(J)$ ist prim}.
      \end{align*}
      Die Aussage für maximale Ideale ergibt sich analog, indem man \emph{prim} durch \emph{maximal} und \emph{Integritätsbereich} durch \emph{Körper} ersetzt.
  \end{enumerate}
\end{solution}


\begin{question}
  Es sei $R$ ein kommutativer Ring.
  \begin{enumerate}
    \item
      Zeigen Sie, dass ein Ideal $\pideal \subseteq R$ genau dann prim ist, wenn $R/\pideal$ ein Integritätsbereich ist.
    \item
      Zeigen Sie, dass ein Ideal $\mideal \subseteq R$ genau dann maximal ist, wenn $R/\mideal$ ein Körper ist. 
  \end{enumerate}
\end{question}


\begin{solution}
  \begin{enumerate}
    \item
      Für alle $x \in R$ sei $\overline{x} \in R/\pideal$ die entsprechende Äquivalenzklasse.
      Das Ideal $\pideal$ ist genau dann prim, wenn die Aussage
      \begin{equation}
        \label{eqn: quotient is integral definition}
        \forall x, y \in R
        :
        \overline{x} \cdot \overline{y} = 0
        \implies
        \text{$\overline{x} = 0$ oder $\overline{y} = 0$}
      \end{equation}
      gilt. Da $\overline{x} \cdot \overline{y} = \overline{xy}$ für alle $x, y \in R$ gilt, ist die Aussage~\eqref{eqn: quotient is integral definition} äquivalent dazu, dass
      \begin{equation}
        \label{eqn: quotient is integral nearly definition}
        \forall x, y \in R
        :
        \overline{xy} = 0
        \implies
        \text{$\overline{x} = 0$ oder $\overline{y} = 0$}.
      \end{equation}
      Für alle $x \in R$ gilt genau dann $\overline{x} = 0$, wenn $x \in \pideal$.
      Deshalb ist die Aussage~\eqref{eqn: quotient is integral nearly definition} äquivalent dazu, dass
      \begin{equation}
        \label{eqn: quotient is integral via prime }
        \forall x, y \in R
        :
        xy \in \pideal
        \implies
        \text{$x \in \pideal$ oder $y \in \pideal$}.
      \end{equation}
      Dies ist genau die Aussage, dass $\pideal$ ein Primideal ist.
    \item
      Es sei $\pi \colon R \to R/\mideal$, $x \mapsto \overline{x}$ die kanonische Projektion.
      Wir erhalten eine wohldefinierte Bijektion
      \[
            \{ \text{Ideale $I \subseteq R/\mideal$} \}
        \to \{ \text{Ideale $J \subseteq R$ mit $J \supseteq \mideal$} \},
        \quad
        I \mapsto \pi^{-1}(I)
      \]
      (siehe Übung~\ref{question: lattice isomorphism for quotients}).
      Der Ring $R/\mideal$ ist genau dann ein Körper, wenn $R/\mideal$ genau zwei Ideale enthält (siehe Übung~\ref{question: characterization of fields via its ideals}); das Ideal $\mideal$ ist genau dann ein maximales Ideal in $R$, wenn es genau zwei Ideale $J \subseteq R$ mit $J \supseteq \mideal$ gibt.
      Wegen der Existenz der obigen Bijektion sind beide Aussagen äquivalent.
  \end{enumerate}
\end{solution}


\begin{question}
  \label{question: unit group of quotients}
  \begin{enumerate}
    \item
      Zeigen Sie für $n \in \Integer$, dass $\overline{q} \in \Integer/n$ genau dann eine Einheit ist, wenn $n$ und $q$ teilerfremd sind.
    \item
      Zeigen Sie allgemeiner:
      Ist $R$ ein kommutativer Ring und $I \subseteq R$ ein Ideal, so ist $\overline{x} \in R/I$ genau dann eine Einheit, wenn $(x) + I = R$.
  \end{enumerate}
\end{question}


\begin{solution}
  \begin{enumerate}
    \item 
      Es ist $\overline{q} \in \Integer/n$ genau dann eine Einheit, wenn es $b \in \Integer$ mit $\overline{q} \overline{b} = \overline{1}$ gibt.
      Dies ist äquivalent dazu, dass es $a, b \in \Integer$ mit $qb - 1 = an$, also $1 = qb - an$ gibt.
      Dies ist äquivalent dazu, dass bereits $(n,q) = 1$ gilt.
      Da $(n,q) = (\ggT(n,q))$ gilt, ist dies wiederum äquivalent dazu, dass $\ggT(n,q) = 1$ gilt, dass also $n$ und $q$ teilerfremd sind.
    \item
      Es sei $\pi \colon R \to R/I$, $x \mapsto \overline{x}$ die kanonische Projektion.
      Wir erhalten eine wohldefinierte Bijektion
      \[
              \{\text{Ideale in $R$, die $I$ enthalten}\}
        \to   \{\text{Ideale in $R/I$}\},
        \quad J \mapsto \pi(J),
        \quad \pi^{-1}(K) \mapsfrom K
      \]
      (siehe Übung~\ref{question: lattice isomorphism for quotients}).
      Inbesondere entspricht das Ideal $(x) + I \subseteq R$ dem Ideal $(\overline{x}) \subseteq R/I$ und das Ideal $R \subseteq R$ den Ideal $R/I \subseteq R/I$.
      Es ist $\overline{x}$ genau dann eine Einheit in $R/I$, wenn $(\overline{x}) = R/I$;
      aufgrund der obigen Bijekton ist dies äquivalent dazu, dass $(x) + I = R$.
  \end{enumerate}
\end{solution}


\begin{question}
  \label{question: radical ideals}
  Es sei $R$ ein kommutativer Ring und $I \subseteq R$ ein Ideal.
  \begin{enumerate}
    \item
      Definieren Sie das Radikal $\radicalideal{I}$ und zeigen Sie, dass $\radicalideal{I}$ ein Ideal mit $I \subseteq \sqrt{I}$ ist.
    \item
      Zeigen Sie, dass $\radicalideal{\radicalideal{I}} = \radicalideal{I}$.
    \item
      Zeigen Sie, dass $\radicalideal{I}$ genau dann ein echtes Ideal ist, wenn $I$ ein echtes Ideal ist.
    \item
      Zeigen Sie für jedes weitere Ideal $J \subseteq R$ die Gleichheit $\radicalideal{I \cap J} = \radicalideal{I} \cap \radicalideal{J}$.
  \end{enumerate}
  Ein Ideal $I$ ist ein \emph{Radikalideal}, wenn $I = \radicalideal{J}$ für ein Ideal $J \subseteq I$.
  \begin{enumerate}[resume]
    \item
      Zeigen Sie, dass $I$ genau dann ein Radikalideal ist, wenn $\radicalideal{I} = I$.
  \end{enumerate}
  Ein Ring $S$ heißt \emph{reduziert}, falls $0$ das einzige nilpotente Element von $S$ ist.
  \begin{enumerate}[resume]
    \item
      Zeigen Sie, dass $R/I$ genau dann reduziert ist, wenn $I$ ein Radikalideal ist.
    \item
      Zeigen Sie, dass jedes Primideal ein Radikalideal ist.
  \end{enumerate}
\end{question}


\begin{solution}
  \begin{enumerate}
    \item
      Das Radikal $\radicalideal{I}$ ist als
      \[
          \radicalideal{I}
        = \{ r \in R \mid \text{es gibt $n \in \Natural$ mit $r^n \in I$} \}
      \]
      definiert.
      Für alle $x \in I$ gilt $x^1 = x \in I$, weshalb $I \subseteq \radicalideal{I}$.
      
      Insbesondere ist somit $0 \in \radicalideal{I}$, da $0 \in I$.
      Für $x, y \in \radicalideal{I}$ gibt es $n, m \in \Natural$ mit $x^n, y^m \in I$.
      Für alle $k = 0, \dotsc, n+m$ gilt deshalb $x^k \in I$ oder $y^{n+m-k} \in I$, und somit auch
      \[
          (x + y)^{n+m}
        = \sum_{k=0}^{n+m} \binom{n+m}{k} x^k y^{n+m-k} \in I.
      \]
      Deshalb ist auch $x + y \in \radicalideal{I}$.
      Für $r \in R$ und $x \in I$ gibt es $n \in \Natural$ mit $x^n \in I$, weshalb auch
      \[
        (r x)^n = r^n x^n \in I.
      \]
      Somit ist auch $r x \in \radicalideal{I}$.
      
    \item
      Wir wissen bereits, dass $\radicalideal{I} \subseteq \radicalideal{\radicalideal{I}}$.
      Für $x \in \radicalideal{\radicalideal{I}}$ gibt es $n \in \Natural$ mit $x^n \in \radicalideal{I}$, und somit auch noch $m \in \Natural$ mit $(x^n)^m \in I$.
      Damit ist $x^{nm} \in I$, weshalb auch $\radicalideal{\radicalideal{I}} \subseteq \radicalideal{I}$.
    
    \item
      $I$ ist genau dann ein echtes Ideal, wenn $1 \notin I$.
      Da $1^n = 1$ für alle $n \in \Natural$ ist genau dann $1 \notin I$, wenn $1 \notin \radicalideal{I}$.
      Dies ist wiederum äquivalent dazu, dass $\radicalideal{I}$ ein echtes Ideal ist.
    
    \item
      Aus den Inklusionen $I \cap J \subseteq I, J$ folgen die Inklusionen $\radicalideal{I \cap J} \subseteq \radicalideal{I}, \radicalideal{J}$ und damit die Inklusion $\radicalideal{I \cap J} \subseteq \radicalideal{I} \cap \radicalideal{J}$.
      
      Ist andererseits $x \in \radicalideal{I} \cap \radicalideal{J}$, so gibt es $n, m \in \Natural$ mit $x^n \in I$ und $x^m \in J$.
      Dann ist $x^{n+m} = x^n x^m \in I \cap J$ (es gilt $x^n x^m \in I$ da $x^n \in I$, und $x^n x^m \in J$ da $x^m \in J$) und deshalb $x \in \radicalideal{I \cap J}$.
    
    \item
      Gilt $I = \radicalideal{I}$ so erfüllt $I$ die definierende Eigenschaft eines Radikalideals (mit $J = I$).
      Ist andererseits $I = \radicalideal{J}$ für ein Ideal $J \subseteq R$, so gilt
      \[
          \radicalideal{I}
        = \radicalideal{\radicalideal{J}}
        = \radicalideal{J}
        = I.
      \]

    \item
      Der Quotient $R/I$ ist genau reduziert, wenn
      \begin{equation}
        \label{eqn: reduced first equation}
        \text{es gibt $n \in \Natural$ mit $\overline{x}^n = 0$}
        \implies
        \overline{x} = 0
        \qquad
        \text{für alle $x \in R$}.
      \end{equation}
      Dabei gilt $\overline{x}^n = \overline{x^n}$ für alle $x \in R$ und $n \in \Natural$, und für alle $y \in R$ gilt genau dann $\overline{y} = 0$, wenn $y \in I$.
      Daher ist \eqref{eqn: reduced first equation} äquivalent dazu, dass
      \begin{equation}
        \label{eqn: reduced second equation}
        \text{es gibt $n \in \Natural$ mit $x^n \in I$}
        \implies
        x \in I
        \qquad
        \text{für alle $x \in R$}.
      \end{equation}
      Durch Einsetzen der Definition von $\radicalideal{I}$ ergibt sich aus \eqref{eqn: reduced second equation} die äquivalente Bedingung
      \[
        x \in \radicalideal{I}
        \implies
        x \in I
        \qquad
        \text{für alle $x \in R$}.
      \]
      Dies bedeutet gerade, dass $\radicalideal{I} \subseteq I$.
      Da $I \subseteq \radicalideal{I}$ ist dies äquivalent dazu, dass $I = \sqrt{I}$, dass also $I$ ein Radikalideal ist.
      
    \item
      Der Quotient $R/\pideal$ ist ein Integritätsbereich, da $\pideal$ ein Primideal ist.
      Nach dem vorherigen Aufgabenteil genügt es zu zeigen, dass jeder Integritätsbereich $S$ reduziert ist.
      Dies folgt direkt daraus, dass für jedes $x \in S$ mit $x^n = 0$ aus der Nullteilerfreiheit von $S$ folgt, dass $x = 0$.
      
      Alternativ lässt sich die Aussage auch direkt zeigen:
      Für $x \in R$ und $n \geq 1$ mit $x^n \in \pideal$ gilt $x \dotsm x \in \pideal$, und da $\pideal$ prim ist, muss bereits einer der Faktoren in $\pideal$ enthalten sein.
  \end{enumerate}
\end{solution}


\begin{question}
  Es sei $R$ ein kommutativer Ring und $\pideal \subseteq R$ ein Ideal.
  Zeigen Sie, dass $\pideal$ genau dann ein Primideal ist, wenn es einen Körper $K$ und einen Ringhomomorphismus $\phi \colon R \to K$ mit $\ker \phi = \pideal$ gibt.
\end{question}


\begin{solution}
  Ist $\pideal$ ein Primideal, so ist der Quotient $R/\pideal$ ein Integritätsbereich.
  Da die kanonische Inklusion $R/\pideal \to Q(R/\pideal)$ ein injektiver Ringhomomorphismus ist, folgt für die Komposition
  \[
    \phi \colon R \xrightarrow{\pi} R/\pideal \to Q(R/\pideal),
  \]
  dass $\ker \phi = \ker \pi = \pideal$.
  (Hier bezeichnet $\pi \colon R \to R/\pideal$ die kanonische Projektion.)
  Da $Q(R/\pideal)$ ein Körper ist, zeigt dies eine Implikation.
  
  Gibt es andererseits einen Körper $K$ und einen Ringhomomorphismus $\phi \colon R \to K$ mit $\pideal = \ker \phi$, so ist $R/\pideal \cong \im \phi \subseteq K$.
  Der Körper $K$ ist insbesondere ein Integritätsbereich, weshalb auch der Unterring $\im \phi$ ein Integritätsbereich ist.
  Der Quotient $R/\pideal$ ist also ein Integritätsbereich und $\pideal$ somit eine Primideal.
\end{solution}


% \begin{question}
%   Es sei $K$ ein Körper.
%   \begin{enumerate}
%     \item
%       Zeigen Sie, dass es für jedes Polynom $f \in K[X]$ einen eindeutigen $K$-linearen Ringhomomorphismus $\phi_f \colon K[X] \to K[X]$ gibt, so dass $\phi_f(X) = f$.
%       
%       (\emph{Hinweis}:
%        Überlegen Sie sich zunächst, dass ein Ringhomomorphismus $\phi \colon K[X] \to K[X]$ genau dann $K$-linear ist, wenn $\phi|_K = \id_K$.)
%     \item
%       Zeigen Sie, dass $\phi_f(p) = p(f)$ für alle $p \in K[X]$.
%     \item
%       Zeigen Sie, dass $\phi_f$ genau dann ein Ringisomorphismus ist, wenn $\deg f = 1$.
%   \end{enumerate}
% \end{question}
% 
% 
% \begin{solution}
%   \begin{enumerate}
%     \item
%       Ist $\phi \colon K[X] \to K[X]$ ein $K$-linearer Ringhomomorphismus, so muss
%       \[
%           \phi(\lambda)
%         = \phi(\lambda \cdot 1)
%         = \lambda \phi(1)
%         = \lambda \cdot 1
%         = \lambda
%         \qquad
%         \text{für alle $\lambda \in K$},
%       \]
%       weshalb $\phi|_K = \id_K$.
%       Ist andererseits $\phi \colon K[X] \to K[X]$ ein Ringhomomorphismus mit $\phi|_K = \id_K$, so gilt
%       \[
%           \phi(\lambda \cdot f)
%         = \phi(\lambda) \cdot \phi(f)
%         = \lambda \cdot \phi(f)
%         \qquad
%         \text{für alle $\lambda \in K$, $f \in K[X]$}.
%       \]
%       Also ist ein Ringhomomorphismus $\phi \colon K[X] \to K[X]$ genau dann $K$-linear, wenn $\phi|_K = \id_K$.
%       
%       Es gilt also zu zeigen, dass es für jedes $f \in K[X]$ genau einen Ringhomomorphismus $\phi_f \colon K[X] \to K[X]$ mit $\phi_f|_K = \id_K$ und $\phi_f(T) = f$ gibt.
%       Dies gilt nach der universellen Eigenschaft des Polynomrings.
%     \item
%       
%   \end{enumerate}
% \end{solution}


\begin{question}[subtitle = Die Einheitengruppe des Potenzreihenrings]
  Es sei $R$ ein kommutativer Ring.
  Zeigen Sie, dass $\powerseries{R}{T}^\times = \{\sum_{i=0}^\infty f_i T^i \in \powerseries{R}{T} \mid f_0 \in R^\times\}$.
\end{question}


\begin{solution}
  Es sei $f = \sum_{i=0}^\infty f_i T^i \in R$.
  
  Ist $f \in \powerseries{R}{T}^\times$, so gibt es $g = \sum_{i=0}^\infty g_i T^i \in \powerseries{R}{T}$ mit $fg = 1$.
  Inbesondere ist dann $f_0 g_0 = 1$ und somit $f_0 \in R^\times$.
  
  Ist andererseits $f_0 \in R^\times$, so seien die Koeffizienten von $g = \sum_{i=0}^\infty g_i T^i \in \powerseries{R}{T}$ rekursiv durch $g_0 = f_0^{-1}$ und $g_i \coloneqq - f_0^{-1} \sum_{j=0}^{i-1} f_{i-j} g_j$ definiert.
  Für $fg = \sum_{i=0}^\infty h_i T^i$ gilt dann $h_0 = f_0 g_0 = 1$, sowie
  \[
      h_i
    = \sum_{j=0}^i f_{i-j} g_j
    = f_0 g_i + \sum_{j=0}^{i-1} f_{i-j} g_j
    = - \sum_{j=0}^{i-1} f_{i-j} g_j + \sum_{j=0}^{i-1} f_{i-j} g_j
    = 0
  \]
  für alle $i \geq 1$, und somit $fg = 1$.
\end{solution}


\begin{question}[subtitle = Funktorialität der Einheitengruppe]
  Ist $R$ ein kommutativer Ring, so ist
  \[
              R^\times
    \coloneqq \{x \in R \mid \text{$x$ ist eine Einheit}\}
  \]
  die \emph{Einheitengruppe} von $R$.
  Zeigen Sie:
  \begin{enumerate}
    \item
      Ist $R$ ein kommutativer Ring, so bildet $R^\times$ mit der Multiplikation aus $R$ eine abelsche Gruppe.
    \item
      Sind $R$ und $S$ zwei kommutativer Ringe und ist $\phi \colon R \to S$ ein Ringhomomorphismus, so induziert $\phi$ per Einschränkung einen Gruppenhomomorphismus
      \[
        \phi^\times \colon R^\times \to S^\times,
        \quad
        x \mapsto \phi(x).
      \]
    \item
      Für jeden Ring kommutativen $R$ gilt $\id_R^\times = \id_{R^\times}$, und für alle kommutativen Ringe $R_1$, $R_2$ und $R_3$ und Ringhomomorphismen $\phi \colon R_1 \to R_2$ und $\psi \colon R_2 \to R_3$ gilt $(\psi \phi)^\times = \psi^\times \phi^\times$.
    \item
      Ist $R$ ein kommutativer Ring und $\phi \colon R \to S$ ein Isomorphismus von Ringen, so ist $\phi^\times \colon R^\times \to S^\times$ ein Isomorphismus von Gruppen.
  \end{enumerate}
  (Die Aussagen gelten auch für nichtkommutative Ringe, wobei $R^\times$ dann im Allgemeinen nicht abelsch ist.
   Dabei ist ein Element $r \in R$ eines nichtkommutativen Rings $R$ eine Einheit, wenn es $s \in R$ mit $rs = 1 = sr$ gibt.
   Es genügt auch, dass es $s, t \in R$ mit $rs = 1 = tr$ gibt; dann gilt bereits $s = t$.)
\end{question}


\begin{solution}
  \begin{enumerate}
    \item
      Die Multiplikation in $R^\times$ ist assoziativ, da sie es in $R$ ist.
      Dass $R^\times$ abelsch ist ergibt sich aus der Kommutativität von $R$.
      Es gilt $1 \in R^\times$, und da $1$ in ganz $R$ neutral bezüglich der Multiplikation ist, gilt dies auch in $R^\times$.
      Für jedes $x \in R^\times$ gibt es ein $y \in R$ mit $xy = 1$.
      Dann gilt auch $y \in R^\times$ und $y$ ist auch in $R^\times$ invers zu $x$.
    \item
      Für $x \in R^\times$ gilt
      \[
          1
        = \phi(1)
        = \phi(x x^{-1})
        = \phi(x) \phi(x^{-1}).
      \]
      Deshalb ist $\phi(x)$ eine Einheit in $S$ (mit $\phi(x)^{-1} = \phi(x^{-1})$), und somit $\phi(x) \in S^\times$.
      Das zeigt, dass die Einschränkung $\phi^\times$ wohldefiniert ist.
      Da $\phi$ mulitplikativ ist, gilt dies auch für $\phi^\times$, weshalb $\phi^\times$ ein Gruppenhomomorphismus ist.
    \item
      Da $\id_R^\times(x) = \id_R(x) = x = \id_{R^\times}(x)$ für alle $x \in X$ gilt, ist $\id_R^\times = \id_{R^\times}$.
      Für alle $x \in R_1$ gilt
      \[
          (\psi^\times \phi^\times)(x)
        = \psi^\times(\phi^\times(x))
        = \psi(\phi(x))
        = (\psi \phi)(x)
        = (\psi \phi)^\times(x).
      \]
      Deshalb ist $(\psi^\times \phi^\times) = (\psi \phi)^\times$.
    \item
      Es sei $\psi \coloneqq \phi^{-1} \colon S \to R$.
      Es gilt
      \[
          \phi^\times \psi^\times
        = (\phi \psi)^\times
        = \left( \phi \phi^{-1} \right)^\times
        = \id_S^\times
        = \id_{S^\times}
      \]
      und analog auch $\psi^\times \phi^\times = \id_{R^\times}$.
      Also ist der Gruppenhomomorphismus $\phi^\times$ bijektiv mit $(\phi^\times)^{-1} = (\phi^{-1})^\times$, und somit ein Gruppenisomorphismus.
  \end{enumerate}
\end{solution}


\begin{question}
  Die Eulersche Phi-Funktion ist definiert als
  \[
            \varphi
    \colon  \Natural_{>1} \to \Natural,
    \quad   n \mapsto |\{ k \in \{1, \dotsc, n\} \mid \text{$k$ und $n$ sind teilerfremd} \}|.
  \]
  \begin{enumerate}
    \item
      Zeigen Sie, dass $\varphi(n) = | (\Integer/n)^\times |$ für alle $n \geq 1$.
    \item
      Folgern Sie, dass $\varphi(n_1 n_2) = \varphi(n_1) \varphi(n_2)$ für je zwei teilerfremde $n_1, n_2 \geq 1$.
    \item
      Zeigen Sie, dass $\varphi(p^r) = p^r - p^{r-1} = p^{r-1} (p - 1)$ für alle Primzahlen $p \in \Natural$ und $r \geq 1$.
    \item
      Berechnen Sie $\varphi(42)$, $\varphi(57)$ und $\varphi(144)$.
  \end{enumerate}
\end{question}


\begin{solution}
  \begin{enumerate}
    \item
      Die Elemente $1, \dotsc, n$ bilden ein Repräsentantensystem der Restklassen von $\Integer/n$, für $k \in \{1, \dotsc, n\}$ ist $\overline{k} \in \Integer/n$ genau dann eine Einheit, wenn $k$ und $n$ teilerfremd sind (siehe Übung~\ref{question: unit group of quotients}).
    \item
      Nach dem Chinesischen Restklassensatz gilt $\Integer/(n_1 n_2) \cong \Integer/n_1 \times \Integer/n_2$.
      Somit gilt
      \begin{align*}
            \varphi(n_1 n_2)
        &=  | (\Integer/(n_1 n_2))^\times |
         =  | (\Integer/n_1 \times \Integer/n_2)^\times |
        \\
        &=  | (\Integer/n_1)^\times \times (\Integer/n_2)^\times |
         =  | (\Integer/n_1)^\times | | (\Integer/n_2)^\times |
         =  \varphi(n_1) \varphi(n_2).
      \end{align*}
    \item
      Es ist $\{0, \dotsc, p^r - 1\}$ ein Repräsentantensystem der Restklassen von $\Integer/p^r$.
      Eine Zahl $k \in \{0, \dotsc, p^r - 1\}$ ist genau dann teilerfremd so $p^r$, wenn sie kein Vielfaches von $p$ ist.
      Da jede $p$-te Zahl aus dieser Menge ein Vielfaches von $p$ ist, gibt es $p^r/p = p^{r-1}$ viele Vielfache von $p$ in diesem Repräsentantensystem.
      Somit sind $p^r - p^{r-1}$ viele Repräsentanten kein Vielfaches von $p$, also teilerfremd zu $p$.
    \item
      Es gelten
      \begin{align*}
            \varphi(42)
        &=  \varphi(2 \cdot 3 \cdot 7)
         =  \varphi(2) \varphi(3) \varphi(7)
         =  (2 - 1)(3 - 1)(7 - 1)
         =  12,
        \\
            \varphi(57)
        &=  \varphi(3 \cdot 19)
         =  (3 - 1)(19 - 1)
         =  36,
        \\
            \varphi(144)
        &=  \varphi(2^4 \cdot 3^2)
         =  (16 - 8) (9 - 3)
         =  48.
      \end{align*}
  \end{enumerate}
\end{solution}


\begin{question}[subtitle = Der Frobeniushomomorphismus]
  \label{question: the frobenius homomorphism}
  Es sei $R$ ein kommutativer Ring mit $p \coloneqq \ringchar R > 0$ prim.
  \begin{enumerate}
    \item
      Zeigen Sie, dass die Abbildung $\sigma \colon R \to R$, $x \mapsto x^p$ ein Ringhomomorphismus ist.
    \item
      Zeigen Sie, dass $\sigma$ ein Automorphismus ist, falls $R$ ein endlicher Körper ist.
  \end{enumerate}
\end{question}


\begin{solution}
  \begin{enumerate}
    \item
      Es gilt $\sigma(1) = 1^p = 1$ und $\sigma(xy) = (xy)^p = x^p y^p = \sigma(x) \sigma(y)$ für alle $x, y \in R$.
      Es bleibt also nur zu zeigen, dass $\sigma$ additiv ist.
      Für alle $x, y \in R$ gilt
      \begin{equation}
        \label{equation: Frobenis homomorphism without cancelation}
        \sigma(x+y) = (x+y)^p = \sum_{k=0}^p \binom{p}{k} x^k y^{p-k}
      \end{equation}
      Für alle $k = 1, \dotsc, p-1$ gilt dabei $p \mid \binom{p}{k}$, denn in dem Ausdruck $\binom{p}{k} = p!/(k!(p-k)!)$ enthält dann zwar der Zähler $p$ als Primfaktor, der Nenner aber nicht, da $k, p-k < p$.
      Folglich vereinfacht sich \eqref{equation: Frobenis homomorphism without cancelation} zu $\sigma(x+y) = x^p + y^p = \sigma(x) + \sigma(y)$.
    \item
      Es gilt $\ker \sigma = 0$, denn für $x \in R$ mit $x^p = \sigma(x) = 0$ gilt wegen der Nullteilerfreiheit von $R$ bereits, dass $x = 0$.
      Also ist $\sigma$ injektiv, und wegen der Endlichkeit von $R$ damit auch schon bijektiv.
  \end{enumerate}
\end{solution}


\begin{question}
  \label{question: examples for non principal and not finitely generated modules}
  Es sei $K$ ein Körper.
  \begin{enumerate}
    \item
      Zeigen Sie, dass $(X,Y) \subseteq K[X,Y]$ kein Hauptideal ist.
    \item
      Zeigen Sie, dass das Ideal $(X_1, X_2, X_3, \dotsc) \subseteq K[X_i \mid i = 1, 2, 3, \dotsc]$ nicht endlich erzeugt ist.
  \end{enumerate}
\end{question}


\begin{solution}
  \begin{enumerate}
    \item
      Wäre $(X,Y) = (f)$ für eine $f \in K[X,Y]$, so wäre $f \mid X$ und $f \mid Y$, also $f$ ein gemeinsamer Teiler von $X$ und $Y$ (sogar schon ein größter gemeinsamer Teiler, siehe Übung~\ref{question: greatest common divisor via generated ideal}).
      Da $X$ und $Y$ teilerfremd sind, muss $f$ bereits eine Einheit in $K[X,Y]$ sein;
      dann gilt aber $(X,Y) = (1) = K[X,Y]$, was nicht gilt (denn $K[X,Y]/(X,Y) \cong K$).
    \item
      Es sei $R \coloneqq K[X_i \mid i = 1, 2, 3, \dotsc]$.
      Wir nehmen an, dass $I \coloneqq (X_i \mid i \in \Natural)$ endlich erzeugt von $f_1, \dotsc, f_t \in I$ erzeugt wird.
      Man bemerke, dass jedes der Polynome in $R$ nur endlich viele Variablen enthält, und dass $I$ aus all jenen Polynomen besteht, deren konstater Term verschwindet.
      Es folgt, dass $I = \bigcup_{n \geq 1} (X_1, X_2, \dotsc, X_n)$;
      inbesondere gibt es ein $N \geq 1$ mit $f_1, \dotsc, f_t \in (X_1, \dotsc, X_N)$.
      Es gilt also
      \[
        I = (f_1, \dotsc, f_t) \subseteq (X_1, \dotsc, X_N) \subseteq I
      \]
      und somit $I = (X_1, \dotsc, X_N)$.
      Damit gilt aber auch
      \begin{align*}
                K
         \cong  R/I
        &\cong  K[X_1, X_2, \dotsc, X_N, X_{N+1}, X_{N+2}, \dotsc]/(X_1, \dotsc, X_N)
        \\
        &\cong  K[X_{N+1}, X_{N+2}, \dotsc]
         \cong  K[X_1, X_2, X_3, \dotsc]
         =      R,
      \end{align*}
      aber $R$ ist kein Körper.
  \end{enumerate}
\end{solution}


% \begin{question}
%   Es sei $R$ ein kommutativer Ring.
%   Es seien $\aideal, \bideal \subseteq R$ zwei Ideale mit $\aideal = (x_i \mid i \in I)$ und $\bideal = (y_j \mid j \in J)$.
%   Zeigen Sie, dass
%   \[
%     \aideal \bideal = (x_i y_j \mid i \in I, j \in J).
%   \]
% \end{question}
% 
% 
% \begin{solution}
%   Für alle $i \in I$ und $j \in J$ folgt aus $x_i \in \aideal$ und $y_j \in \bideal$, dass $x_i y_j \in \aideal \bideal$.
%   Daraus folgt, dass $(x_i y_j \mid i \in I, j \in J) \subseteq \aideal \bideal$.
%   Sind andererseits $a \in \aideal$ und $b \in \bideal$, so ist $a = \sum_{i \in I} r_i x_i$ und $b = \sum_{j \in J} s_j y_j$ mit $r_i, s_j \in R$, wobei $r_i = 0$ für fast alle $i \in I$ und $s_j = 0$ für fast alle $j \in J$.
%   Deshalb ist
%   \[
%         ab
%     =   \sum_{\substack{i \in I \\ j \in J}} r_i s_j x_i y_j
%     \in (x_i y_j \mid i \in I, j \in J).
%   \]
%   Da jedes Element aus $\aideal \bideal$ von der Form $\sum_{k=1}^n a_k b_k$ mit $a_k \in \aideal$ und $b_k \in \bideal$ ist, folgt daraus, dass $\aideal \bideal \subseteq (x_i y_j \mid i \in I, j \in J)$.
% \end{solution}


\begin{question}[subtitle = Zur Definition von Unterringen]
  Geben Sie ein Beispiel für einen kommutativen Ring $R$ und eine Teilmenge $S \subseteq R$ mit den folgenden Eigenschaften:
  \begin{itemize}
    \item
      $S$ ist abgeschlossen unter der Addition und Multiplikation von $R$, d.h.\ für alle $s_1, s_2 \in S$ ist auch $s_1 + s_2 \in S$ und $s_1 s_2 \in S$.
    \item
      Zusammen mit der Einschränkung der Addition und Multiplikation aus $R$ ist $S$ ebenfalls ein (notwendigerweise kommutativer) Ring.
    \item
      $S$ ist kein Unterring von $R$.
  \end{itemize}
\end{question}


\begin{solution}
  Es sei $R = \Integer \times \Integer$ und $S = \Integer \times 0 = \{(n,0) \mid n \in \Integer\}$.
  Offenbar ist $S$ unter der Addition und Multiplikation abgeschlossen.
  Zusammen mit der Einschränkung dieser Operationen bildet $S$ einen kommutativen Ring, für den $S \cong \Integer$ gilt.
  Da $1_R = (1,1) \notin S$ ist $S$ allerdings kein Unterring von $R$.
\end{solution}


\begin{question}
  Es sei $R$ ein kommutativer Ring.
  \begin{enumerate}
    \item
      Definieren Sie, wann zwei Elemente von $R$ assoziiert sind.
    \item
      Zeigen Sie, dass Assoziiertheit eine Äquivalenzrelation ist.
    \item
      Es sei nun $R$ ein Integritätsbereich.
      Zeigen Sie, dass zwei Elemente $a, b \in R$ genau dann assoziiert sind, wenn $(a) = (b)$.
  \end{enumerate}
\end{question}


\begin{solution}
  \begin{enumerate}
    \item
      Ein Element $y \in R$ ist assoziiert zu einem Element $x \in R$, wenn es eine Einheit $\varepsilon \in R^\times$ mit $y = \varepsilon x$ gibt.
  \end{enumerate}
  Für $x, y \in R$ schreiben wir im Folgenden $x \sim y$, wenn $y$ assoziiert zu $x$ ist.
  \begin{enumerate}[resume]
    \item
      Für jedes $x \in R$ ist $x \sim x$ da $x = 1 \cdot x$ mit $1 \in R^\times$.
      Für $x, y \in R$ mit $x \sim y$ gibt es $\varepsilon \in R^\times$ mit $y = \varepsilon x$;
      dann ist $\varepsilon^{-1} \in R^\times$ mit $x = \varepsilon^{-1} y$ und deshalb $y \sim x$.
      Für $x, y, z \in R$ mit $x \sim y$ und $y \sim z$ gibt es $\varepsilon_1, \varepsilon_2 \in R^\times$ mit $y = \varepsilon_1 x$ und $z = \varepsilon_2 y$;
      dann ist $\varepsilon_2 \varepsilon_1 \in R^\times$ mit $z = \varepsilon_2 y = \varepsilon_2 \varepsilon_1 x$ und somit $x \sim z$.
    \item
      Für $x, y \in R$ mit $x \sim y$ gibt es $\varepsilon \in R^\times$ mit $x = \varepsilon y$.
      Dann ist $R \varepsilon = R$ und deshalb
      \[
          (x)
        = \{r x \mid r \in R\}
        = \{r \varepsilon y \mid r \in R\}
        = \{r' y \mid r' \in R \varepsilon\}
        = \{r' y \mid r' \in R\}
        = (y).
      \]
      Ist andererseits $(x) = (y)$ so ist $x \in (y)$ und $y \in (x)$, also gibt es $\varepsilon_1, \varepsilon_2 \in R$ mit $y = \varepsilon_1 x$ und $x = \varepsilon_2 y$.
      Dann ist $y = \varepsilon_1 x = \varepsilon_1 \varepsilon_2 y$, und da $R$ ein Integritätsbereich ist, somit $\varepsilon_1 \varepsilon_2 = 1$.
      Also ist $\varepsilon_1$ eine Einheit mit $\varepsilon_1^{-1} = \varepsilon_2$.
      Da $y = \varepsilon_1 x$ ist $x \sim y$.
  \end{enumerate}
\end{solution}


\begin{question}
  Es sei $R$ ein kommutativer Ring.
  \begin{enumerate}
    \item
      Zeigen Sie, dass für nilpotentes $n \in R$ das Element $1 - n$ eine Einheit ist, und geben Sie $(1 - n)^{-1}$ an.
    \item
      Zeigen Sie, dass für nilpotentes $n \in R$ das Element $1 + n$ eine Einheit ist, und geben Sie $(1 + n)^{-1}$ an.
    \item
      Zeigen Sie, dass für nilpotentes $n \in R$ und jede Einheit $e \in R^\times$ das Element $e + n$ eine Einheit ist, und geben Sie $(e + n)^{-1}$ an.
  \end{enumerate}
\end{question}


\begin{solution}
  \begin{enumerate}
    \item
      Für $k \geq 0$ mit $n^k = 0$ gilt $(1 - n)(1 + n + \dotsb + n^{k-1}) = 1 - n^k = 1$.
      Also ist $1 - n$ eine Einheit mit $(1-n)^{-1} = \sum_{p=0}^{k-1} n^p = \sum_{p=0}^\infty n^p$.
    \item
      Da $n$ nilpotent ist, gilt dies auch für $-n$.
      Nach dem vorherigen Aufgabenteil ist deshalb $1 + n = 1 - (-n)$ eine Einheit mit $(1 + n)^{-1} = (1 - (-n))^{-1} = \sum_{p=0}^\infty (-1)^p n^p$.
    \item
      Es gilt $e + n = e(1 + e^{-1} n)$, und da $n$ nilpotent ist, gilt dies auch für $e^{-1} n$.
      Nach dem vorherigen Teil ist $1 + e^{-1} n$ eine Einheit, und somit $e + n$ als Produkt zweier Einheiten ebenfalls eine Einheit; ferner gilt
      \[
          (e + n)^{-1}
        = e^{-1} (1 + e^{-1} n)^{-1}
        = e^{-1} \sum_{p=0}^\infty (-1)^p (e^{-1} n)^p
        = \sum_{p=0}^\infty (-1)^p e^{-1-p} n^p.
      \]
  \end{enumerate}
\end{solution}


\begin{question}[subtitle = Ideale in der Lokalisierung]
  Es sei $R$ ein kommutativer Ring und $S \subseteq R$ eine multiplikative Teilmenge.
  \begin{enumerate}
    \item
      Zeigen Sie, dass jedes Ideal $J \subseteq R_S$ von der Form $J = I_S$ für ein Ideal $I \subseteq R$ ist.
    \item
      Zeigen Sie, dass $I_S = (a_i/1 \mid i \in I)$ falls $I = (a_i \mid i \in I)$.
    \item
      Zeigen Sie, dass $R_S$ noethersch ist, wenn $R$ noethersch ist.
    \item
      Zeigen oder widerlegen Sie, dass $R_S$ ein Hauptidealring ist, wenn $R$ ein Hauptidealring ist.
  \end{enumerate}
\end{question}


\begin{solution}
  \begin{enumerate}
    \item
      Es sei $I \coloneqq \{r \in R \mid r/1 \in J\}$.
      Dies ist ein Ideal in $I$:
      
      Es gilt $0 \in I$, da $0/1 \in J$.
      Für $r_1, r_2 \in I$ gelten $r_1/1, r_2/1 \in J$ und somit auch $(r_1 + r_2)/1 = r_1/1 + r_2/1 \in J$, also $r_1 + r_2 \in I$.
      Für $r \in I$ und jedes $r' \in R$ gilt $r/1 \in J$ und somit auch $(r' r)/1 = (r'/1)(r/1) \in J$, also $r' r \in I$.
      Insgesamt zeigt dies, dass $I$ ein Ideal ist.
      
      Alternativ betrachte man den kanonischen Ringhomomorphismus $i \colon R \to R_S$, $r \mapsto r/1$.
      Für diesen gilt $I = i^{-1}(J)$, also ist $I$ ein Ideal (siehe Übung~\ref{question: preimages of ideals}).
      
      Für jedes $r \in I$ und $s \in S$ gilt $r/1 \in J$ und somit auch $r/s = (1/s)(r/1) \in J$, weshalb $I_S \subseteq J$.
      Für jedes $r/s \in J$ gilt $r/1 = (s/1)(r/s) \in J$ und somit auch $r \in I$, also $r/s \in I_S$.
      Deshalb gilt auch $J \subseteq I_S$.
      
    \item
      Es gilt $a_i \in I$ für alle $i \in I$, also $a_i/1 \in I_S$ für alle $i \in I$ und deshalb $(a_i/1 \mid i \in I) \subseteq I_S$.
      Ist andererseits $r/s \in I_S$ mit $r \in R$, so gilt $r = \sum_{i \in I} r_i a_i$ mit $r_i = 0$ für fast alle $i \in I$.
      Dann gilt $r/s = \sum_{i \in I} (r_i/s)(a_i/1) \in (a_i \mid i \in I)$.
      Also gilt auch $I_S \subseteq (a_i/1 \mid i \in I)$.
    
    \item
      Ist $J \subseteq R_S$ ein Ideal, so gilt nach dem ersten Aussagenteil $J = I_S$ für ein Ideal $I \subseteq J$.
      Das Ideal $I$ ist endlich erzeugt, da $R$ noethersch ist, also $I = (a_1, \dotsc, a_n)$.
      Nach dem zweiten Aussagenteil gilt deshalb $J = I_S = (a_1/1, \dotsc, a_n/1)$, weshalb $J$ endlich erzeugt.
      Es ist also jedes Ideal in $R_S$ endlich erzeugt, also $R_S$ noethersch.
    
    \item
      Gilt $0 \in S$, so ist $R_S = 0$ kein Hauptidealring.
      Für $0 \notin S$ ist $R_S$ wieder ein Hauptidealring:
      Analog zum Beweis des vorherigen Aussagenteils erhalten wir, dass jedes Ideal in $R_S$ ein Hauptideal ist.
      Es bleibt daher nur zu zeigen, dass $R_S$ ein Integritätsbereich ist.
      Dies folgt aber daraus, dass $R$ ein Integritätsbereich ist, und dass $R_S$ wegen $0 \notin S$ daher als Unterring des Quotientenkörpers $Q(R)$ realisiert werden kann.
  \end{enumerate}
\end{solution}


\begin{question}
  Es sei $R$ ein Ring und $I \subseteq R$ ein Ideal.
  \begin{enumerate}
    \item
      Zeigen Sie, dass $R/I$ noethersch ist, wenn $R$ noethersch ist.
    \item
      Zeigen Sie widerlegen, dass $R/I$ ein Hauptidealring ist, wenn $R$ ein Hauptidealring ist.
  \end{enumerate}
\end{question}


\begin{solution}
  \begin{enumerate}
    \item
      Es sei $J \subseteq R/I$ ein Ideal.
      Dann gibt es ein Ideal $J' \subseteq R$ mit $J' \supseteq I$ und $J = J'/I$ (siehe Übung~\ref{question: lattice isomorphism for quotients}).
      Das Ideal $J'$ ist endlich erzeugt, da $R$ noethersch ist, also $J' = (a_1, \dotsc, a_n)$.
      Damit gilt $J = J/I = (\overline{a_1}, \dotsc, \overline{a_n})$, weshalb auch $J$ endlich erzeugt ist.
      Der Ring $R/I$ ist also noethersch, da jedes seiner Ideale endlich erzeugt ist.
    \item
      Analog zum Beweis des ersten Aussagenteiles ergibt sich, dass jedes Ideal in $R/I$ ein Hauptideal ist.
      Damit $R/I$ ein Integritätsbereich ist, muss allerdings noch zusätzlich gefordert werden, dass $I$ ein Primideal ist.
      Ist etwa $K$ ein Körper, so ist in $K[X]/(X^2)$ zwar jedes Ideal ein Hauptideal, aber $0 \neq \overline{X} \in K[X]/(X^2)$ ist nilpotent, weshalb $K[X]/(X^2)$ kein Integritätsbereich, und somit auch kein Hauptidealring ist.
  \end{enumerate}
\end{solution}


% \begin{question}
%   Für jedes $d \in \Natural$ sei
%   \[
%               \Integer[\sqrt{-d}]
%     \coloneqq \Integer[i\sqrt{d}]
%     =         \{a + i \sqrt{d} b \mid a, b \in \Integer \}
%     \subseteq \Complex.
%   \]
%   Es darf im Folgenden ohne Beweis genutzt werden, dass $\Integer[\sqrt{-d}]$ ein Unterring von $\Complex$ ist.
%   \begin{enumerate}
%     \item
%       Zeigen Sie, dass $\Integer[\sqrt{-1}]$ ein euklidischer Ring ist.
%     \item
%       Zeigen Sie, dass $\Integer[\sqrt{-2}]$ ein euklidischer Ring ist.
%     \item
%       Zeigen Sie, dass $\Integer[\sqrt{-5}]$ kein euklidischer Ring ist.
%   \end{enumerate}
% \end{question}


\begin{question}
  Es sei $R$ ein euklidischer Ring.
  Zeigen Sie, dass $R$ ein Hauptidealring ist.
\end{question}


\begin{solution}
  Als euklidischer Ring ist $R$ insbesondere ein Integritätsbereich.
  Es sei $g \colon R \to \Natural$ die Gradabbildung und $I \subseteq R$ ein Ideal.
  Ist $I = 0$ so ist $I = (0)$, wir betrachten daher den Fall $I \neq 0$.
  Dann gibt es ein bezüglich $g$ minimales $a \in I$, d.h.\ $a \in I$ mit $a \neq 0$ und $g(a) \leq g(x)$ für alle $x \in I$ mit $x \neq 0$.
  Es gilt $(a) \subseteq I$ und es handelt sich bereits um Gleichheit:
  Ist $x \in I$ so gibt es $b, r \in R$ mit $x = ab + r$, und entweder $r = 0$ oder $g(r) < g(a)$.
  Da $r = x - ab \in I$ kann $g(r) < g(a)$ wegen der Minimalität von $a$ nicht eintretten.
  Also ist $r = 0$ und somit $x = ab \in (a)$.
\end{solution}


\begin{question}
  Es sei $R$ ein Hauptidealring.
  Zeigen Sie, dass jedes Primideal in $R$ bereits ein maximales Ideal ist.
\end{question}


\begin{solution}
  Es sei $\mideal \subseteq R$ ein Primideal und $p \in R$ mit $\mideal = (p)$;
  inbesondere ist $p$ prim.
  Es sei $\aideal \subseteq R$ ein Ideal mit $\mideal \subseteq \aideal$ und $a \in R$ mit $\aideal = (a)$.
  Dass $(p) \subseteq (a)$ gilt, ist äquivalent dazu, dass $a \mid p$ gilt;
  es gibt also $b \in R$ mit $p = ab$.
  Da $p$ prim ist, gilt bereits $p \mid a$ oder $p \mid b$.
  
  Gilt $p \mid b$, so gibt es $c \in R$ mit $b = pc$.
  Dann gilt $p = ab = abp$ und somit $1 = ab$, da $R$ ein Integritätsbereich ist.
  In diesem Fall ist also $a$ eine Einheit und somit $\aideal = (a) = R$ kein echtes Ideal.
  
  Gilt andererseits $p \mid a$, so ist $\aideal = (a) \subseteq (p) = \mideal$, und somit bereits $\mideal = \aideal$.
  
  Ingesamt zeigt dies, dass es kein echtes Ideal $\bideal \subsetneq R$ gibt, so dass $\mideal \subsetneq \bideal$.
  Da $\mideal$ als Primideal inbesondere ein echtes Ideal ist, folgt daraus, dass $\mideal$ ein maximales Ideal ist.
\end{solution}


\begin{question}
  \label{question: fields are the only rings for which the polynomial ring is a pid}
  Es sei $K$ ein kommutativer Ring, so dass $K[X]$ ein Hauptidealring ist.
  Zeigen Sie, dass $K$ bereits ein Körper ist.
\end{question}


\begin{solution}
  Wir geben zwei mögliche Beweise:  
  \begin{enumerate}
    \item
      Es sei $a \in K$ mit $a \neq 0$.
      Das Ideal $(a, X)$ ist nach Annahme ein Hauptideal.
      Also gibt es ein Polynom $f \in K[X]$ mit
      \begin{equation}
        \label{eqn: ideal is principal}
        (a, X) = (f).
      \end{equation}
      Wegen Gleichung \eqref{eqn: ideal is principal} gilt $f \mid a$, d.h.\ es gibt $g \in K[X]$ mit $fg = a$.
      Entscheident ist nun die folgende Beobachtung:
      
      \begin{claim}
        \label{claim: degree is additive}
        Die übliche Gradabbildung $\deg \colon K[X] \to \Natural$ ist additiv.
      \end{claim}
      \begin{proof}
        As Hauptidealring ist $K[X]$ inbesondere ein Integritätsbereich.
        Also ist auch der Unterring $K \subseteq K[X]$ ein Integritätsbereich, woraus die Aussage folgt.
      \end{proof}
      Aus Behauptung~\ref{claim: degree is additive} erhalten wir, dass
      \[
          0
        = \deg(a)
        = \deg(fg)
        = \deg(f) + \deg(g).
      \]
      Es muss $\deg(f) = \deg(g) = 0$ gelten und somit bereits $f, g \in K$.
      
      Da $f \in (a, X)$ gibt es $p, q \in K[X]$ mit $f = a p + X q$.
      Da $f \in K$ und $\deg(X q) \geq 1$ ergibt sich durch Vergleich des $0$-ten Koeffizienten, dass $f = f_0 = a_0 p_0 = a p_0$.
      Deshalb gilt bereits $f = a p_0 \in (a)$.
      Wir haben also
      \[
                  (a, X)
        =         (f)
        \subseteq (a)
        \subseteq (a, X)
      \]
      und somit $(a, X) = (a)$.
      
      Es gibt deshalb $h \in K[X]$ mit $X = a h$.
      Durch Gradvergleich erhalten wir, dass
      \[
          1
        = \deg(X)
        = \deg(a h)
        = \deg(a) + \deg(h)
        = 0 + \deg(h)
        = \deg(h)
      \]
      und deshalb $h(X) = b_1 X + b_0$ für $b_1, b_0 \in K$.
      Durch Koeffizientenvergleich erhalten wir aus der Gleichung
      \[
          X
        = a h(X)
        = a (b_1 X + b_0)
        = a b_1 X + a b_0,
      \]
      dass $a b_1 = 1$.
      Das zeigt, dass $a \in A$ eine Einheit ist.
      
    \item
      Der obige Beweis lässt sich leicht ändern.
      Wir zeigen, dass das Ideal $(X)$ maximal ist.
      Ansonsonsten gebe es $a \in K[X]$, so dass $(X) \subsetneq (a, X) \subsetneq K[X]$.
      Da $(a, X) = (a_0, X)$ können o.B.d.A.\ davon ausgehen, dass $a \in K$.
      Wie zuvor ergibt sich, dass $(a, X) = (X)$, was $(X) \subsetneq (a,X)$ widerspricht.
      Also ist $(X)$ maximal, und $K \cong K[X]/(X)$ somit ein Körper.
  \end{enumerate}
  Der erste Beweis hat den Vorteil, dass er für einen beliebigen kommutativen Ring $R$ zeigt, dass $(a, X)$ für $a \in R$ genau dann ein Hauptidealring ist, wenn $a \in R^\times$.
  Somit ist beispielsweise $(2, X) \subseteq \Integer[X]$ kein Hauptideal.
\end{solution}


\begin{question}
  \label{question: greatest common divisor via generated ideal}
  Es sei $R$ ein kommutativer Ring, $(a_i)_{i \in I}$ eine Familie von Elementen $a_i \in R$ und $a \in R$.
  \begin{enumerate}
    \item
      Zeigen Sie, dass $a$ ein größter gemeinsamer Teiler der $a_i$ ist, falls $(a_i \mid i \in I) = (a)$.
    \item
      Entscheiden Sie, ob auch die Umkehrung der obigen Aussage gilt.
  \end{enumerate}
\end{question}


\begin{question}
  \begin{enumerate}
    \item
      Es ist $a$ ein gemeinsamer Teiler der $a_i$, denn es gilt
      \[
              (a_i \mid i \in I) \subseteq (a)
        \iff  \forall i \in I: a_i \in (a)
        \iff  \forall i \in I: a \mid a_i.
      \]
      Da außerdem $a \in (a) \subseteq (a_i \mid i \in I)$ gilt, ergibt sich $a = \sum_{i \in I} r_i a_i$ für passende $r_i \in R$ mit $r_i = 0$ für fast alle $i \in I$.
      Für jeden gemeinsamen Teiler $b \in R$ der $a_i$ gilt deshalb auch $b \mid a$.
      Somit ist $a$ bereits ein größter gemeinsamer Teiler der $a_i$.
    \item
      Die Umkehrung gilt im Allgemeinen nicht:
      Ist etwa $K$ ein Körper, so ist $1$ ein größter gemeinsamer Teiler von $X, Y \in K[X, Y]$, aber $(X, Y) \subsetneq (1)$.
  \end{enumerate}
\end{question}


\begin{question}[subtitle = Euklid]
  Es sei $K$ ein Körper.
  Zeigen Sie, dass es in $K[X]$ unendlich viele normierte, irreduzible Polynome gibt.
\end{question}


\begin{solution}
  Wir nehmen an, dass es nur endlich viele normierte, irreduzible Polynome in $K[X]$ gibt, nämlich $p_1, \dotsc, p_n \in K[X]$.
  Man bemerke, dass $n \geq 1$, da die Polynome $X - a$ für $a \in K$ irreduzibel und normiert sind.
  Für das Element
  \[
    q \coloneqq 1 + p_1 \dotsm p_n \in K[X]
  \]
  gilt dann $\deg q \geq n \geq 1$.
  Es gilt $q \equiv 1 \pmod{p_i}$ für alle $i = 1, \dotsc, n$, und somit $p_i \nmid q$ für alle $i = 1, \dotsc, n$.
  Da die $p_i$ ein Repräsentantensystem der Primelemente von $K[X]$ sind, widerspricht dies der Existenz einer Primfaktorzerlegung von $q$.
\end{solution}



\begin{question}
  Es sei $R$ ein Ring und $I \subseteq R$ ein echtes Ideal.
  Zeigen Sie mithilfe des Lemmas von Zorn, dass es ein maximales Ideal $\mideal \subsetneq R$ gibt, so dass $I \subseteq \mideal$.
\end{question}


\begin{solution}
  Es sei
  \[
              \mathcal{I}
    \coloneqq \{J \subsetneq R \mid \text{$J$ ist ein echtes Ideal mit $I \subseteq J$}\}.
  \]
  Die Menge $\mathcal{I}$ ist nicht leer, da sie $I$ enthält.
  Bezüglich der üblichen Teilmengeninklusion $\subseteq$ ist $\mathcal{I}$ partiell geordnet.
  
  Ist $\mathcal{K} \subseteq \mathcal{I}$ eine nicht-leere Kette, so ist auch $K \coloneqq \bigcup \mathcal{K} = \bigcup_{J \in \mathcal{K}} J$ wieder ein Ideal in $R$, und es gilt $I \subseteq K$.
  Da alle $J \in \mathcal{K}$ echte Ideale sind, gilt $1 \notin J$ für alle $J \in \mathcal{K}$;
  somit gilt auch $1 \notin K$, weshalb $K$ ein echtes Ideal in $R$ ist.
  Ingesamt ist also $K \in \mathcal{I}$, und da $J \subseteq K$ für alle $J \in \mathcal{K}$ gilt, ist $K$ eine obere Schranke für $\mathcal{K}$ in $\mathcal{I}$.
  
  Nach dem Lemma von Zorn besitzt nun $\mathcal{I}$ ein maximales Element $\mideal \in \mathcal{I}$;
  inbesondere ist $\mideal$ ein echtes Ideal in $R$ mit $I \subseteq \mideal$.
  Wäre $\mideal$ kein maximales Ideal in $R$, so gebe es ein echtes Ideal $\mideal' \subsetneq R$ mit $\mideal \subsetneq \mideal'$.
  Dann wäre aber $I \subseteq \mideal \subsetneq \mideal'$ und somit $\mideal' \in \mathcal{I}$. Da $\mideal \subsetneq \mideal'$ stünde dies im Widerspruch zur Maximalität von $\mideal$ in $\mathcal{I}$.
  Also muss $\mideal$ bereits ein maximales Ideal in $R$ sein.
\end{solution}


\begin{question}
  Es sei $R$ ein kommutativer Ring und $\aideal \subseteq R$ ein Ideal.
  \begin{enumerate}
    \item
      Zeigen Sie, dass für jedes Ideal $\aideal \subseteq R$ die Teilmenge
      \[
                  \aideal[X]
        \coloneqq \left\{
                    \sum_i f_i X^i \in R[X]
                  \,\middle|\,
                    \text{$f_i \in \aideal$ für alle $i$}
                  \right\}
      \]
      ein Ideal in $R[X]$ ist.
    \item
      Zeigen Sie, dass die Abbildung
      \[
        R[X]/\aideal[X] \to (R/\aideal)[X],
        \quad
        \overline{\sum_i a_i X^i} \mapsto \sum_i \overline{a_i} X^i
      \]
      ein wohldefinierter Isomorphismus ist.
    \item
      Zeigen Sie, dass für jedes Primideal $\pideal \subseteq R$ auch $\pideal[X] \subseteq R[X]$ ein Primideal ist.
    \item
      Zeigen oder widerlegen Sie, dass für jedes maximale Ideal $\mideal \subseteq R$ auch $\mideal[X] \subseteq R[X]$ ein maximales Ideal ist.
  \end{enumerate}
  Das Ideal $\aideal[X]$ lässt sich auch noch anders durch $\aideal$ beschreiben.
  \begin{enumerate}[resume]
    \item
      Zeigen Sie, dass $\aideal[X]$ das von $\aideal$ in $R[X]$ erzeugte Ideal ist, d.h.\ dass $(\aideal)_{R[X]} = \aideal[X]$.
  \end{enumerate}
  Damit erhalten wir für jedes Ideal $\aideal \subseteq R$ einen Ringisomorphismus $R[X]/(\aideal) \to (R/\aideal)[X]$, $\overline{\sum_i a_i X^i} \mapsto \sum_i \overline{a_i} X^i$.
  \begin{enumerate}[resume]
    \item
      Veinfachen Sie für die folgenden Ringe $R$ und Ideale $I \subseteq R$ jeweils den Quotienten $R/I$.
      Entscheiden Sie jeweils, ob das Ideal prim oder maximal ist.
      \[
        (7)               \subseteq \Integer[X],
        \quad
        (3, X^2 + 1)      \subseteq \Integer[X],
        \quad
        (5, X^2 + X + 3)  \subseteq \Integer[X],
        \quad
        (X^2 + 1)         \subseteq \Rational[X,Y].
      \]
  \end{enumerate}
\end{question}


\begin{solution}
  \begin{enumerate}
    \item
      Die kanonische Projektion $\pi \colon R \to R/\aideal$, $x \mapsto \overline{x}$ induziert nach der universellen Eigenschaft des Polynomrings $R[X]$ einen Ringhomomorphismus $\varphi \colon R[X] \to (R/\aideal)[X]$ mit $\varphi|_R = \pi$ und $\varphi(X) = \pi(X)$, und dieser ist gegeben durch
      \[
          \varphi\left( \sum_i f_i X^i \right)
        = \sum_i \pi(f_i) X^i
        = \sum_i \overline{f_i} X^i.
      \]
      Für $f = \sum_i f_i X^i \in R[X]$ ist genau dann $f \in \ker \varphi$, wenn $\overline{f_i} = 0$ für alle $i$, also genau dann, wenn $f_i \in \ker \pi = \aideal$ für alle $i$.
      Somit ist $\ker \varphi = \aideal[X]$ ein Ideal in $R[X]$.
    \item
      Es seien $\pi$ und $\varphi$ wie zuvor.
      Wegen der Surjektivität von $\pi$ ist auch $\varphi$ surjektiv.
      Somit induziert $\varphi$ einen Ringisomorphismus
      \[
        \psi \colon R[X] / \ker \varphi \to (R/\pideal)[X],
        \quad
        \overline{\sum_i f_i X^i} \mapsto \sum_i \overline{f_i} X^i.
      \]
      Nach dem vorherigen Aussagenteil gilt $\ker \psi = \aideal[X]$, was die Aussage zeigt.
    \item
      Der Quotient $R/\pideal$ ist ein Integritätsbereich, da $\pideal$ ein Primideal in $R$ ist.
      Damit ist auch $(R/\pideal)[X] \cong R[X]/\pideal[X]$ ein Integritätsbereich ist, und deshalb $\pideal[X]$.
    \item
      Ist $K$ ein Körper, so ist $0 \subseteq K$ ein maximales Ideal, und es gilt $\mideal[X] = 0$.
      Der Quotient $K[X]/\mideal[X] \cong (K/0)[X] \cong K[X]$ ist kein Körper, da $0 \neq X \in K[X]$ keine Einheit ist.
      Also ist $\mideal[X]$ nicht maximal in $K[X]$.
      
      Tatsächlich kann $\mideal[X]$ nicht maximal in $R[X]$ sein, da $R[X]/\mideal[X] \cong (R/\mideal)[X]$, aber es keinen Ring $R'$ gibt, so dass $R'[X]$ ein Körper ist (siehe Übung~\ref{qst: polynomial rings are not fields}).
    \item
      Es gilt $\aideal \subseteq \aideal[X]$ und somit $(\aideal)_{R[X]} \subseteq \aideal[X]$.
      Andererseits ist $a X^i \in (\aideal)_{R[X]}$ für jedes $a \in \aideal$ und $i \geq 0$ und somit $\sum_i a_i X^i \in \in (\aideal)_{R[X]}$ für jedes $\sum_i a_i X^i \in \aideal[X]$.
    \item
      \begin{enumerate}[leftmargin=*]
        \item
          Es gilt $\Integer[X]/(7) \cong (\Integer/7)[X] = \Field_7[X]$.
          Das Ideal ist also prim, aber nicht maximal.
        \item
          Mithilfe des dritten Isomorphiesatzes erhält man, dass
          \begin{align*}
                    \Integer[X]/(3, X^2 + 1)
            &\cong  ( \Integer[X]/(3) ) / ( (3, X^2 + 1) / (3) )
             =      ( \Integer[X]/(3) ) / ( \overline{X^2 + 1} )
            \\
            &\cong  (\Integer/3)[X] / (X^2 + 1)
             =      \Field_3[X] / (X^2 + 1).
          \end{align*}
          Das Polynom $X^2 +1 \in \Field_3[X]$ ist quadratisch und hat keine Nullstellen, ist also irreduzibel.
          Der obige Quotient ist also eine Körpererweiterung von $\Field_3$ von Grad $2$, also $\Field_3[X] / (X^2 + 1) \cong \Field_9$.
          Inbesondere ist das Ideal maximal.
        \item
          Mithilfe des dritten Isomorphiesatzes erhält man, dass
          \begin{align*}
                 &\,  \Integer[X]/(5, X^2 + 6X - 2)
            \cong     (\Integer[X]/(5)) / ((5, X^2 + 6X - 2)/(5))
          \\
            \cong&\,  (\Integer[X]/(5)) / (\overline{X^2 + 6X - 2})
            \cong     (\Integer/5)[X] / (X^2 + 6X -2)
          \\
            \cong&\,  \Field_5[X] / (X^2 - 4X + 3)
            =         \Field_5[X] / ( (X-1) (X-3) ).
          \end{align*}
          Mithilfe des chinesischen Restklassensatzes erhält man weiter, dass
          \[
                  \Field_5[X] / ( (X-1)(X-3))
            \cong \Field_5[X]/(X-1) \times \Field_5[X]/(X-3)
            \cong \Field_5 \times \Field_5.
          \]
          Da $\Field_5 \times \Field_5$ kein Integritätsbereich ist, ist das Ideal nicht prim.
        \item
          Es gilt
          \[
                  \Rational[X,Y]/(X^2 + 1)
            \cong \Rational[X][Y]/(X^2 + 1)
            \cong (\Rational[X]/(X^2 + 1))[Y]
            \cong \Rational(i)[Y].
          \]
          Inbesondere ist das Ideal prim, aber nicht maximal.
      \end{enumerate}
  \end{enumerate}
\end{solution}


\begin{question}
  \label{qst: polynomial rings are not fields}
  Zeigen Sie, dass es keinen Ring $R$ gibt, so dass $R[X]$ ein Körper ist.
\end{question}


\begin{solution}
  Gebe es einen solchen Ring $R$, so wäre $R$ kommutativ, da $R \subseteq R[X]$ ein Unterring ist.
  Es wäre auch $R \neq 0$ da $0[X] = 0$ kein Körper ist.
  Dann wäre aber $0 \neq X \in R[X]$ keine Einheit und $R[X]$ somit kein Körper.
\end{solution}


\begin{question}
  \label{question: product of ideals is an ideal}
  Es seien $R_1, \dotsc, R_n$ kommutative Ringe für jedes $i = 1, \dotsc, n$ sei $\aideal_i \subseteq R_i$ ein Ideal.
  \begin{enumerate}
    \item
      Zeigen Sie, dass $\aideal_1 \times \dotsb \times \aideal_n$ ein Ideal in $R_1 \times \dotsb \times R_n$ ist.
    \item
      Zeigen Sie, dass $(R_1 \times \dotsb R_n)/(\aideal_1 \times \dotsb \times \aideal_n) \cong (R_1/\aideal_1) \times \dotsb \times (R_n/\aideal_n)$ gilt.
    \item
      Folgern Sie, dass $\aideal_1 \times \dotsb \times \aideal_n$ genau dann prim ist, wenn es ein $1 \leq j \leq n$ gibt, so dass $\aideal_j \subseteq R_j$ prim ist, und $\aideal_i = R_i$ für alle $i \neq j$ gilt.
    \item
      Entscheiden Sie, ob die obige Aussage auch für maximale Ideale gilt.
  \end{enumerate}
\end{question}


\begin{solution}
  \begin{enumerate}
    \item
      Für jedes $i = 1, \dotsc, n$ sei $\pi_i \colon R_i \to R_i/\aideal_i$, $x \mapsto \overline{x}$ die kanonische Projektion.
      Es ist
      \[
        R_1 \times \dotsb \times R_n
        \xrightarrow{\pi \coloneqq \pi_1 \times \dotsb \times \pi_n}
        (R_1/\aideal_1) \times \dotsb \times (R_n/\aideal_n)
      \]
      ein Ringhomomorphismus mit $\ker \pi = (\ker \pi_1) \times \dotsb \times (\ker \pi_n) = \aideal_1 \times \dotsb \times \aideal_n$.
      Inbesondere ist deshalb $\aideal_1 \times \dotsb \times \aideal_n$ ein Ideal in $R_1 \times \dotsb \times R_n$.
    \item
      Da die $\pi_i$ surjektiv sind, ist es auch $\pi$.
      Da $\ker \pi = \aideal_1 \times \dotsb \times \aideal_n$ gilt, induziert $\pi$ also einen Isomorphismus
      \begin{align*}
                \overline{\pi}
        \colon  (R_1 \times \dotsb \times R_n)/(\aideal_1 \times \dotsb \times \aideal_n)
        &\to    (R_1/\aideal_1) \times \dotsb \times (R_n/\aideal_n),
        \\
                  \overline{(x_1, \dotsc, x_n)}
        &\mapsto  (\overline{x_1}, \dotsc, \overline{x_n}).
      \end{align*}
    \item
      Ist $\aideal_i \neq R_i$ und $\aideal_j \neq R_j$ für $i \neq j$, so sind in dem Produkt $(R_1/\aideal_1) \times \dotsb \times (R_n/\aideal_n)$ mindestens zwei Faktoren nicht trivial, und der Ring somit nicht nullteilerfrei.
      Es genügt daher, sich auf den Fall einzuschränken, dass $\aideal_i = R_i$ für alle $i = 1, \dotsc, n$ bis auf ein $1 \leq j \leq n$.
      Dann gilt
      \begin{gather*}
        \begin{aligned}
                  (R_1 \times \dotsb R_n)/(\aideal_1 \times \dotsb \times \aideal_n)
          &\cong  (R_1/\aideal_1) \times \dotsb \times (R_n/\aideal_n)
          \\
          &\cong  0 \times \dotsb \times 0 \times (R_j/\aideal_j) \times 0 \times \dotsb \times 0
           \cong  R_j/\aideal_j,
        \end{aligned}
      \intertext{und somit}
        \begin{aligned}
              &\, \text{$\aideal_1 \times \dotsb \times \aideal_n$ ist prim}
          \\
          \iff&\, \text{$(R_1 \times \dotsb R_n)/(\aideal_1 \times \dotsb \times \aideal_n)$ ist ein Integritätsbereich}
          \\
          \iff&\, \text{$R_j/\aideal_j$ ist ein Integritätsbereich}
          \iff \text{$\aideal_j$ ist prim}.
        \end{aligned}
      \end{gather*}
    \item
      Die Aussage gilt auch für maximale Ideale.
      Man muss nur in der obigen Argumentation \emph{prim} durch \emph{maximal} und \emph{Integritätsbereich} durch \emph{Körper} ersetzen.
  \end{enumerate}
\end{solution}


\begin{question}
  \label{questions: ideals in products are products of ideal}
  Es seien $R_1, \dotsc, R_n$ kommutative Ringe
  Zeigen Sie, dass jedes Ideal $\aideal \subseteq R_1 \times \dotsb \times R_n$ von der Form $\aideal = \aideal_1 \times \dotsb \times \aideal_n$ für eindeutige Ideale $\aideal_i \subseteq R_i$ ist.
\end{question}


\begin{solution}
  Die Eindeutigkeit ist klar, und es gilt nur die Existenz zu zeigen:
  Für jedes $i = 1, \dotsc, n$ sei $\pi_i \colon R_1 \times \dotsb \times R_n \to R_i$, $(x_1, \dotsc, x_n) \mapsto x_i$ die kanonische Projektion.
  Für jedes $i = 1, \dotsc, n$ sei außerdem $e_i = (0, \dotsc, 0, 1, 0, \dotsc, 0) \in R_1 \times \dotsb \times R_n$ das Element, dessen $i$-ter Eintrag $1$ ist, und dessen Einträge sonst alle $0$ sind.
  Für alle $i = 1, \dotsc, n$ sei $\aideal_i \coloneqq \pi_i(\aideal)$.
  
  Es gilt $\aideal \subseteq \aideal_1 \times \dotsb \times \aideal_n$, denn für jedes $(x_1, \dotsc, x_n) \in \aideal$ gilt $x_i = \pi(x) \in \aideal_i$ für alle $i = 1, \dotsc, n$ und somit $x \in \aideal_1 \times \dotsb \times \aideal_n$.
  
  Ist andererseits $x = (x_1, \dotsc, x_n) \in \aideal_1 \times \dotsb \times \aideal_n$, so ist $x_i \in \aideal_i$ für alle $i = 1, \dotsc, n$.
  Für alle $i = 1, \dotsc, n$ gibt es deshalb ein $y^{(i)} = (y^{(i)}_1, \dotsc, y^{(i)}_n) \in \aideal$ mit $\pi_i(y^{(i)}) = x_i$, also $y^{(i)}_i = x_i$.
  Es folgt, dass
  \begin{align*}
        x
    &=  (x_1, \dotsc, x_n)
     =  \sum_{i=1}^n \underbrace{(0, \dotsc, 0, x_i, 0, \dotsc, 0)}_{\text{$x_i$ an $i$-ter Stelle}}
    \\
    &=  \sum_{i=1}^n e_i \left( y^{(i)}_1, \dotsc, y^{(i)}_{i-1}, x_i, y^{(i)}_{i+1}, \dotsc, y^{(i)}_n \right)
     =  \sum_{i=1}^n e_i y^{(i)}
    \in \aideal.
  \end{align*}
\end{solution}


\begin{remark*}
  Übung~\ref{question: product of ideals is an ideal} und Übung~\ref{questions: ideals in products are products of ideal} ergeben zusammen eine Klassifikation der Primideale, bzw.\ maximalen Ideale in $R_1 \times \dotsb \times R_n$:
  Es handelt sich (in gewisser Weise) um die disjunkte Vereinigung der Primideale, bzw.\ maximalen Ideale der $R_i$.
\end{remark*}


% \begin{question}
%   Zeigen Sie, dass $\Integer[i] \cong \Integer[X]/(X^2 + 1)$.
% \end{question}


% \begin{question}
%   Es sei $R$ ein Integritätsbereich und $K$ ein Körper.
%   \begin{enumerate}
%     \item
%       Zeigen Sie, dass $Q(R[X]) \cong Q(R)(X)$.
%     \item
%       Folgern Sie, dass $K(X,Y) \cong K(X)(Y)$.
%   \end{enumerate}
% \end{question}


\begin{question}
  Es sei $f \colon R \to R'$ ein Ringhomomorphismus zwischen kommutativen Ringen $R$ und $R'$.
  Es sei $S \subseteq R$ eine multiplikative Menge.
  \begin{enumerate}
    \item
      Zeigen Sie, dass $S' \coloneqq f(S)$ eine multiplikative Menge in $R'$ ist.
    \item
      Zeigen Sie, dass es einen eindeutigen Ringhomomorphismus $\hat{f} \colon R \to R'$ gibt, so dass das folgende Diagramm kommutiert:
      \[
        \begin{tikzcd}[ampersand replacement=\&]
              R
              \arrow{r}{f}
              \arrow{d}
          \&  R'
              \arrow{d}
          \\
              R_S
              \arrow{r}{\hat{f}}
          \&  R'_{S'}
        \end{tikzcd}
      \]
      Hierbei sind die unbenannten vertikalen Pfeile jeweils die kanonischen Ringhomomorphismen.
  \end{enumerate}
\end{question}


\begin{solution}
  \begin{enumerate}
    \item
      Da $1 \in S$ ist $1 = f(1) \in f(S) = S'$.
      Für $s'_1, s'_2 \in S'$ gibt es $s_1, s_2 \in S$ mit $s'_1 = f(s_1)$ und $s'_2 = f(s_2)$, und damit ist auch $s'_1 s'_2 = f(s_1) f(s_2) = f(s_1 s_2) \in f(S) = S'$.
    \item
      Es seien $i \colon R \to R_S$, $r \mapsto r/1$ und $i' \colon R' \to R'_{S'}$, $r' \mapsto r'/1$ die kanonischen Ringhomomorphismen.
      Die Komposition $i' \circ f \colon R \mapsto R'_{S'}$ bildet $s \in S$ auf die Einheit $f(s)/1 \in R'_{S'}$
      ab.
      Nach der universellen Eigenschaft der Lokalisierung induziert $i' \circ f$ einen eindeutigen Ringhomomorphismus $\hat{f} \colon R_S \to R'_{S'}$ mit $\hat{f} i = i' f$, d.h.\ so dass das folgende Diagram kommutiert:
      \[
        \begin{tikzcd}[ampersand replacement = \&]
              R
              \arrow{r}{f}
              \arrow{d}{i}
          \&  R'
              \arrow{d}{i'}
          \\
              R_S
              \arrow{r}{\hat{f}}
          \&  R'_{S'}
        \end{tikzcd}
      \]
  \end{enumerate}
\end{solution}


\begin{question}
  Es seien $R$ und $R'$ zwei kommutative Ringe, und $S \subseteq R$ und $S' \subseteq R'$ multiplikative Mengen.
  Es seien $i \colon R \to R_S$ und $i' \colon R' \to R'_{S'}$ die kanonischen Ringhomomorphismen.
  \begin{enumerate}
    \item
      Zeigen Sie, dass $S \times S' \subseteq R \times R'$ eine multiplikative Menge ist.
  \end{enumerate}
  Es sei $j \colon R \times R' \to (R \times R')_{S \times S'}$ der kanonische Ringhomomorphismus.
  \begin{enumerate}[resume]
    \item
      Zeigen Sie, dass es eine eindeutigen Ringhomomorphismus $\varphi \colon (R \times R')_{S \times S'} \to R_S \times R'_{S'}$ gibt, so dass das folgende Diagram kommutiert:
      \[
        \begin{tikzcd}[ampersand replacement = \&]
              {}
          \&  R \times R'
              \arrow{dl}[swap]{j}
              \arrow{dr}{i \times i'}
              {}
          \\
              (R \times R')_{S \times S'}
              \arrow{rr}{\varphi}
          \&  {}
          \&  R_S \times R'_{S'}
        \end{tikzcd}
      \]
    \item
      Zeigen Sie, dass $\varphi$ ein Isomorphismus ist.
  \end{enumerate}
\end{question}


\begin{solution}
  \begin{enumerate}
    \item
      Es gelten $1 \in S$ und $1' \in S'$ und somit auch $1_{R \times R'} = (1_R, 1_{R'}) \in S \times S'$.
      
      Für $(s_1, s'_1), (s_2, s'_2) \in S \times S'$ gelten auch $s_1 s_1' \in S_1$ und $s_2 s_2' \in S_2$, und deshalb gilt $(s_1 s_1', s_2, s_2') \in S \times S'$.
      
    \item
      Für $(s, s') \in S \times S'$ gelten $i(s) = s/1 \in R_S^\times$ und $i'(s') = s'/1 \in R_{S'}^\times$ und somit $(i \times i')(s,s') = (s/1, s'/1) \in R_S^\times \times R_{S'}^\times = (R_S \times R_{S'})^\times$.
      Nach der universellen Eigenschaft der Lokalisierung $(R \times R')_{S \times S'}$ setzt sich $i \times i'$ eindeutig zu einem Ringhomomorphismus $\varphi \colon (R \times R')_{S \times S'} \to R_S \times R'_{S'}$ fort, so dass $\varphi \circ j = i \times i'$, d.h.\ so dass das gegebene Diagramm kommutiert.
      Konkret gilt $\varphi((r,r')/(s,s')) = (r/s, r'/s')$ für alle $(r,r')/(s,s') \in (R \times R')_{S \times S'}$.
      
    \item
      Für jedes $(r/s, r'/s') \in R_S \times R'_{S'}$ gilt $(r/s, r'/s') = \varphi((r,r')/(s,s')) \in \im \varphi$, also ist $\varphi$ surjektiv.
      Für $(r,r')/(s,s') \in \ker \varphi$ gilt $0 = \varphi((r,r')/(s,s')) = (r/s, r'/s')$.
      Es ist also $r/s = 0/1$, es gibt also $t \in S$ mit $tr = 0$.
      Analog gibt es auch $t' \in S'$ mit $t'r' = 0$.
      Dann ist $(t,t') \in S \times S'$ mit $(t,t')(r,r') = (tr,t'r') = 0$, weshalb $(r,r')/(s,s') = 0$ gilt.
      Also ist $\ker \varphi = 0$ und $\varphi$ somit auch injektiv.
  \end{enumerate}
\end{solution}



\begin{question}
  Es sei $R$ ein kommutativer Ring und $f \in R$.
  Zeigen Sie, dass $R_f \cong R[X]/(fX-1)$.
\end{question}


\begin{solution}
  Das Element $\overline{f} \in R[X]/(fX-1)$ ist eine Einheit mit $\overline{f}^{-1} = \overline{X}$ da
  \[
      \overline{f} \, \overline{X}
    = \overline{fX}
    = \overline{1}
    = 1.
  \]
  Nach der universellen Eigenschaft der Lokalisierung $R_f$ induziert der Ringhomomorphismus $R \to R[X] \to R[X]/(fX-1)$ einen Ringhomomorphismus $\varphi \colon R_f \to R[X]/(fX-1)$ mit
  \[
      \varphi\left( \frac{r}{f^k} \right)
    = \frac{\overline{r}}{\overline{f}^k}
    = \overline{r} \overline{X}^k
    = \overline{r X^k}.
  \]
  
  Andererseits induziert der kanonische Ringhomomorphismus $i \colon R \to R_f$, $r \mapsto r/1$ nach der universellen Eigenschaft des Polynomrings $R[X]$ einen eindeutigen Ringhomomorphismus $\tilde{\psi} \colon R[X] \to R_f$ mit $\tilde{\psi}|_R = i$ und $\tilde{\psi}(X) = 1/f$, und dieser ist gegeben durch
  \[
      \tilde{\psi}\left( \sum_i r_i X^i \right)
    = \sum_i \frac{r_i}{f^i}.
  \]
  Dann gilt insbesondere
  \[
      \tilde{\psi}(fX-1)
    = \tilde{\psi}(f) \tilde{\psi}(X) - \tilde{\psi}(1)
    = \frac{f}{1} \frac{1}{f} - \frac{1}{1}
    = 0.
  \]
  Also faktorisiert $\tilde{\psi}$ über einen eindeutigen Ringhomomorphismus $\psi \colon R[X]/(fX-1) \to R_f$ mit $\psi(\overline{p}) = \tilde{\psi}(p)$ für alle $p \in R[X]$, d.h.\ es ist
  \[
      \psi\left( \overline{ \sum_i r_i X^i } \right)
    = \sum_i \frac{r_i}{f^i}
    \qquad
    \text{für alle $\sum_i r_i X^i \in R[X]$}.
  \]
  
  Die beiden Ringhomomorphismen $\varphi$ und $\psi$ sind invers zueinander:
  Für alle $r/f^k \in R_f$ gilt
  \[
      \psi\left( \varphi\left( \frac{r}{f^k} \right) \right)
    = \psi\left( \overline{r X^k} \right)
    = \frac{r}{f^k},
  \]
  und für alle $\sum_i r_i X^i \in R[X]$ gilt
  \[
      \varphi\left( \psi\left( \overline{\sum_i r_i X^i} \right) \right)
    = \varphi\left( \sum_i \frac{r_i}{f^i} \right)
    = \sum_i \varphi\left( \frac{r_i}{f^i} \right)
    = \overline{\sum_i r_i X^i}.
  \]
  Also ist $\varphi$ ein Isomorphismus mit $\varphi^{-1} = \psi$.
\end{solution}


\begin{question}
  Bestimmen Sie die Einheitengruppe $\Integer[i]^\times$.
\end{question}


\begin{solution}
  Ein Element $z \in \Integer[i]$ ist genau dann eine Einheit in $\Integer[i]$, wenn $z \neq 0$ und $z^{-1} \in \Integer[i]$ (hier bezeichnet $z^{-1} = 1/z$ das Inverse von $z$ in $\Complex$).
  Für die Elemente $1, -1, i, -i \in \Integer[i]$ ist dies erfüllt.
  Ist $z \in \Integer[i]$ mit $z \neq 0$ und $z^{-1} \in \Integer[i]$, so ist
  \begin{equation}
    \label{eqn: units in the gaussian integers}
      1
    = |1|^2
    = |z z^{-1}|^2
    = |z|^2 |z^{-1}|.
  \end{equation}
  Für alle $w \in \Integer[i]$ mit $w = a + ib$ gilt $a, b \in \Integer$ und deshalb $|w|^2 = a^2 + b^2 \in \Integer$.
  In \eqref{eqn: units in the gaussian integers} gilt deshalb, dass $|z|^2, |z^{-1}|^2 \in \Integer$, und somit $|z|^2 \in \Integer^\times = \{1,-1\}$.
  Also gilt $|z|^2 = 1$.
  Ist $z = a + ib$ mit $a,b \in \Integer$ so ist also $a^2 + b^2 = 1$ und somit entweder $a = 0$ und $b = \pm 1$, oder $a = \pm 1$ und $b = 0$.
  Es ist also $z \in \{1, -1, i, -i\}$.
  Insgesamt zeigt dies, dass $\Integer[i]^\times = \{1, -1, i, -i\}$.
\end{solution}


\begin{question}
  Formulieren und beweisen Sie den Hilbertschen Basissatz.
\end{question}


\begin{question}
  Der Hilbertsche Basissatz besagt, dass für einen noetherschen Ring $R$ auch der Polynomring $R[X]$ noethersch ist.
  
  Zum Beweis des Satzes sei $I \subseteq R[X]$ ein Ideal.
  Für jedes $d \geq 0$ sei
  \[
              I_d
    \coloneqq \{a \in R \mid \text{es gibt $f \in I$ mit $\deg f = d$ und Leitkoeffizienten $a$}\} \cup \{0\}.
  \]
  
  \begin{claim*}
    Für jedes $d \geq 0$ ist $I_d \subseteq R$ ein Ideal
  \end{claim*}
  \begin{proof}
    Per Definition gilt $0 \in I_d$.
    Sind $a_1, a_2 \in I_d$ so gibt es Polynome $f_1, f_2 \in I$ vom Grad $d$, so dass $a_i$ der Leitkoeffizient von $f_i$ ist.
    Ist $a_1 = -a_2$, so gilt $a_1 + a_2\in I_d$;
    andernfalls löschen sich in der Summe $f_1 + f_2 \in I$ die beiden Leitkoeffizienten nicht aus, weshalb $f_1 + f_2$ ein Polynom von Grad $d$ mit Leitkoeffizienten $a_1 + a_2$ ist.
    In beiden Fällen gilt $a_1 + a_2 \in I_d$.
    Ist $r \in R$ und $a \in I_d$, so gibt es ein Polynom $f \in I$ vom Grad $d$ mit Leitkoeffizienten $a$.
    Ist $r \cdot a = 0$, so ist insbesondere $r \cdot a \in I_d$;
    ist $r \cdot a \neq 0$, so ist auch $r f \in I$ wieder vom Grad $d$ und hat $ra$ als Leitkoeffizienten, weshalb auch $ra \in I_d$.
  \end{proof}
  
  Ist $f \in I$ vom Grad $d$ mit Leitkoeffizietnen $a \in R$, so ist $X f \in I$ vom Grad $d+1$ mit Leitkoeffizienten $a$.
  Deshalb ist $I_d \subseteq I_{d+1}$ für alle $d \geq 0$.
  Da $R$ noethersch ist, stabiliert die aufsteigende Kette $0 \subseteq I_1 \subseteq I_2 \subseteq I_3 \subseteq \dotsb$; es gibt also ein $D \geq 0$ mit $I_D = I_{D+l}$ für alle $l \geq 0$.
  Die Ideale $I_0, \dotsc, I_D$ sind endlich erzeugt, da $R$ noethersch sind;
  Für jedes $d = 0, \dotsc, D$ sei $I_d = (a^{(d)}_1, \dotsc, a^{(d)}_{s_d})$.
  Für jedes $d = 0, \dotsc, D$ und $i = 1, \dotsc, s_d$ gibt es dann ein Polynom $f^{(d)}_i \in I$ vom Grad $d$ mit Leitkoeffizienten $a^{(d)}_i$;
  man merke für $d = 0$, dass $f^{(0)}_i = a^{(0)}_i$ für alle $i = 0, \dotsc, d$, da ein konstantes Polynom mit seiem Leitkoeffizienten übereinstimmt.
  Es sei $J \coloneqq (f^{(d)}_i \mid d = 0, \dotsc, D, i = 1, \dotsc, s_d) \subseteq I$.
  
  Wir zeigen, dass bereits $J = I$ gilt, dass also $f \in J$ für jedes $f \in I$.
  Wir zeigen dies per Induktion über den Grad von $f$:
  Für $\deg f = 0$ gilt bereits $f\in I_0$, und es gilt
  \[
              I_0
    =         (a^{(0)}_1, \dotsc, a^{(0)}_{s_0})_R
    \subseteq (a^{(0)}_1, \dotsc, a^{(0)}_{s_0})_{R[X]}
    =         (f^{(0)}_1, \dotsc, af^{(0)}_{s_0})_{R[X]}
    \subseteq J.
  \]
  Es sei nun $f \in I$ mit $d \coloneqq \deg f \geq 1$ und es gelte $g \in J$ für alle $f \in I$ mit $\deg g \leq d-1$.
  Es sei $a \in I_d$ der Leitkoeffizient von $f$.
  Wir unterscheiden zwischen zwei Fällen:
  \begin{itemize}
    \item
      Gilt $d \leq D$, so gilt $a \in I_d = (a^{(d)}_1, \dotsc, a^{(d)}_{s_d})$, also $a = \sum_{i=1}^{s_d} r_i a^{(d)}_i$ für passende $r_i \in R$.
      Das Polynom $g \coloneqq \sum_{i=1}^{s_d} r_i f^{(d)}_i \in J$ hat dann $a$ als Leitkoeffzienten und hat ebenfalls Grad $d$.
      Das Polynom $f - g \in I$ hat daher echt kleineren Grad als $f$, weshalb nach Induktionsvoraussetzung $f - g \in J$ gilt.
      Somit ist auch $f = (f - g) + g \in J$.
    \item
      Gilt $d \geq D$, so ist $a \in I_d = I_D$.
      Es gilt daher $a = \sum_{i=1}^{s_D} r_i a^{(D)}_i$ für passende $r_i \in R$.
      Das Polynom $g \coloneqq \sum_{i=1}^{s_D} r_i f^{(D)}_i \in J$ hat dann $a$ als Leitkoeffizienten und Grad $D \leq d$.
      Deshalb ist $X^{d-D} g \in J$ mit Leitkoeffizienten $a$ und Grad $d$.
      Das Polynom $f - X^{d-D} g$ hat daher echt kleineren Grad als $f$, weshalb nach Induktionsvoraussetzung $f - X^{d-D} g \in J$ gilt.
      Somit ist auch $f = (f - X^{d-D} g) + X^{d-D} g \in J$.
  \end{itemize}
  In beiden Fällen erhalten wir also, dass auch $f \in J$.
  
  Ingesamt zeigt dies, dass $I = J$ endlich erzeugt ist.
  Der Ring $R[X]$ ist also noethersch, da jedes seiner Ideale endlich erzeugt ist.
\end{question}


\begin{question}[subtitle = Ein Lemma von Gauß]
  Es sei $R$ ein faktorieller Ring und $f, g \in R[T]$ seien zwei primitive Polynome.
  Zeigen Sie, dass auch $fg$ primitiv ist.
\end{question}


\begin{question}
  Wir nehmen an, dass $fg$ nicht primitiv ist.
  Dann gibt es ein Primelement $p \in R$, dass alle Koeffizienten von $fg$ teilt.
  Für den von der kanonischen Projektion $R \to R/(p)$, $r \mapsto \overline{r}$ induzierten Ringhomomorphismus $\varphi \colon R[T] \to (R/(p))[T]$ gilt dann $0 = \varphi(fg) = \varphi(f) \varphi(g)$.
  Der Quotient $R/(p)$ ist ein Integritätsbereich, da $p$ prim ist, und $(R/(p))[T]$ somit ebenfalls.
  Also muss bereits $\varphi(f) = 0$ oder $\varphi(g) = 0$.
  Dann sind aber alle Koeffizienten von $f$ durch $p$ teilbar, oder alle Koeffizienten von $g$ durch $p$ teilbar, was der Primitivität von $f$ und $g$ widerspricht.
\end{question}


\begin{question}
  Es sei $K$ ein Körper und $R \coloneqq K[t^2, t^3] \subseteq K[t]$.
  \begin{enumerate}
    \item
      Zeigen Sie, dass $R$ noethersch ist.
    \item
      Folgern Sie, dass in $R$ eine Zerlegung in irreduzible Elemente existiert.
    \item
      Zeigen Sie, dass $R$ nicht faktoriell ist.
      (\emph{Hinweis}:
       Zeigen Sie zunächst, dass $t^2$ und $t^3$ irreduzibel sind.)
  \end{enumerate}
\end{question}


\begin{solution}
  \begin{enumerate}
    \item
      Nach dem Hilbertschen Nullstellensatz ist $K[X,Y]$ noethersch.
      Der Einsetzhomomorphismus $\varphi \colon K[X,Y] \to K[t^2, t^3]$ mit $\varphi(X) = t^2$ und $\varphi(Y) = t^3$ ist surjektiv, und somit $R = K[t^2, t^3] \cong K[X,Y]/\ker \varphi$ als Quotient eines noetherschen Rings ebenfalls noethersch.
    \item
      Als Unterring von $K[t]$ ist $R$ ein Integritätsbereich.
      Aus der Vorlesung ist bekannt, dass in noetherschen Integritätsbereichen eine Zerlegung in irreduzible Elemente existiert.
    \item
      Wir bemerken zunächst, dass $R = \{\sum_i f_i T^i \in K[t] \mid f_1 = 0\}$.
      Es enthält also $R$ keine Polynome vom Grad $1$.
      Jede nicht-triviale Zerlegung von $t^2$ oder $t^3$ in $K[t]$ enthält aber einen Faktor vom Grad $1$;
      folglich sind beide Elemente irreduzibel in $R$.
      Wir erhalten nun für $t^6 \in R$ mit $t^6 = t^2 \cdot t^2 \cdot t^2 = t^3 \cdot t^3$ zwei Zerlegungen in irreduzible Elemente, die nicht äquivalent im Sinne eines faktoriellen Rings sind (insbesondere kommen in beiden Zerlegungen unterschiedlich viele Faktoren vor).
      Folgich ist $R$ nicht faktoriell.
  \end{enumerate}
\end{solution}


\begin{question}
  Es sei $K$ ein Körper.
  Zeigen Sie, dass der Ring $\powerseries{K}{X}$ ein eindeutiges maximales Ideal besitzt.
\end{question}


\begin{solution}
  Wir geben zwei mögliche Beweise an:
  \begin{enumerate}
    \item
      Der Ring $\powerseries{K}{X}$ ein ein euklidisch mit der üblichen Gradabbildung $\Deg$ und somit ein Hauptidealring.
      Also ist jedes Ideal in $\powerseries{K}{X}$ von der Form $(f)$ für ein Element $f \in \powerseries{K}{X}$.
      Ist $f \in \powerseries{K}{X}$ mit $f \neq 0$, so ist $f = \sum_{i=n}^\infty f_i X^i$ mit $f_n \neq 0$ für ein $n \geq 0$.
      Dann ist
      \[
          f
        = \sum_{i=n}^\infty f_i X^i
        = X^n \cdot \sum_{j=0}^\infty f_{n+j} X^j
        = X^n \cdot g.
      \]
      für das Element $g \coloneqq \sum_{j=0}^\infty f_{n+j} X^j$.
      Es gilt $g_0 \neq 0$, also $g_0 \in K^\times$, und somit $g \in \powerseries{K}{X}^\times$.
      Also sind $f$ und $X^n$ assoziiert, und somit $(f) = (X^n)$.
      
      Damit ist gezeigt, dass $0$ und die Ideale $(X^n)$ für $n \geq 0$ die einzigen Ideale in $\powerseries{K}{X}$ sind.
      Falls $(X^n) = (X^m)$ mit $n \leq m$, so sind $X^n$ und $X^m$ assoziiert zueinander, und es gibt $g \in \powerseries{K}{X}^\times$ mit $X^n = g X^m$.
      Dann gilt $g_0 \neq 0$ und somit $\Deg g X^m = m$, weshalb $n = \Deg X^n = \Deg g X^m = m$.
      Die Ideale in $\powerseries{K}{X}$ bilden also eine echte absteigende Kette
      \[
                    (1)
        =           (X^0)
        \supsetneq  (X)
        \supsetneq  (X^2)
        \supsetneq  (X^3)
        \supsetneq  (X^4)
        \supsetneq \dotsb
        \supsetneq  0.
      \]
      Inbesondere ist $(X)$ das eindeutige maximale Ideal in $\powerseries{R}{X}$.
    \item
      Es sei $\mideal \coloneqq \{f \in \powerseries{K}{X} \mid f_0 = 0\}$.
      Die Abbildung $\varphi \colon \powerseries{K}{X} \to K$, $f \mapsto f_0$ ist ein Ringhomomorphismus mit $\ker \varphi = \mideal$, weshalb $\mideal$ ein Ideal in $\powerseries{K}{X}$ ist.
      Da
      \[
              K
        =     \im \varphi
        \cong \powerseries{K}{X}/\ker \varphi
        =     \powerseries{K}{X}/\mideal
      \]
      ein Körper ist, ist $\mideal$ bereits ein maximales Ideal.

      Gebe es ein maximales Ideal $\mideal' \subseteq \powerseries{K}{X}$ mit $\mideal' \neq \mideal$, so würde wegen der Maximalität von $\mideal'$ inbesondere $\mideal' \nsubseteq \mideal$ gelten.
      Dann gebe es $f \in \mideal'$ mit $f \notin \mideal$, also $f_0 \neq 0$ und somit $f \in K^\times$.
      Dann würde aber $f \in \powerseries{K}{X}^\times$ gelten, und somit $(1) = (f) \subseteq \mideal'$, was im Widerspruch dazu stünde, dass $\mideal'$ ein echtes Ideal in $\powerseries{K}{X}$ ist.
  \end{enumerate}
\end{solution}
