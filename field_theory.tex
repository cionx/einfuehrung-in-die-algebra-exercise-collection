\section{Körpertheorie}



\begin{question}[subtitle = Wahr oder Falsch?]
  \begin{enumerate}
    \item
      Ist $L/K$ eine endliche Körpererweiterung mit $[L : K] \neq 1$, so gilt auch $\Gal(L/K) \neq 1$.
    \item
      Es gilt $\sqrt[4]{2} \in \Rational(\sqrt[10]{6})$.
    \item
      Sind $M/L/K$ Körpererweiterungen, so dass $M/K$ normal ist, so ist auch $M/L$ normal.
    \item
      Sind $M/L/K$ Körpererweiterungen, so dass $M/L$ und $L/K$ normal sind, so ist auch $M/K$ normal.
    \item
      Es gilt $\Field_{8} \subseteq \Field_{32}$.
    \item
      Ist $L/K$ eine Körpererweiterung und $\overline{L} \supseteq L$ ein algebraischer Abschluss, so ist auch $\overline{L} \supseteq K$ auch ein algebraischer Abschluss.
    \item
      Ist $L/K$ ein Körperweiterung und $L$ der Zerfällungskörper eines Polynoms $f \in K[X]$ von Grad $n \coloneqq \deg f$, so gilt $[L : K] \mid n!$.
    \item
      Ist $L/K$ ein Körperweiterung und $L$ der Zerfällungskörper eines Polynoms $f \in K[X]$ von Grad $n \coloneqq \deg f$, so gilt $|{\Gal(L/K)}| \mid n!$.
    \item
      Es sei $L/K$ eine Körpererweiterung und $L_1, L_2 \subseteq L$ seien Unterkörper, so dass $L_1/K$ und $L_2/K$ Unterkörper sind.
      Bezeichnet $L_1 L2 \subseteq L$ den kleinsten Unterkörper, der $L_1$ und $L_2$ enthält, so ist auch $L_1 L_2/K algebraisch$.
  \end{enumerate}
\end{question}


\begin{solution}
  \begin{enumerate}
    \item
      Die Aussage ist falsch:
      Man betrachte etwa $K = \Rational$ und $L = \Rational(\sqrt[3]{2})$.
      Jeder Automorphismus $\sigma \in \Gal(\Rational(\sqrt[3]{2})/\Rational)$ muss die Nullstellen des Polynoms $f(X) \coloneqq X^3-2 \in \Rational[X]$ permutieren.
      Die Nullstellen von $f$ sind $\sqrt[3]{2}$, $\zeta \sqrt[3]{2}$, und $\zeta^2 \sqrt[3]{2}$, wobei $\zeta \in \Complex$, $\zeta \notin \Real$ eine primitive dritte Einheitswurzel ist (etwa $\zeta = e^{2 \pi i/3}$).
      Inbesondere gelten $\zeta \sqrt[3]{2}, \zeta^2 \sqrt[3]{2} \notin \Real$.
      Da $\Rational(\sqrt[3]{2}) \subseteq \Real$ gilt, ist deshalb $\sqrt[3]{2}$ die einzige Nullstelle von $f$ in $\Rational(\sqrt[3]{2})$;
      für jedes $\sigma \in \Gal(\Rational(\sqrt[3]{2})/\Rational)$ muss deshalb $\sigma(\sqrt[3]{2}) = \sqrt[3]{2}$ gelten, und somit bereits $\sigma = \id_{\Rational(\sqrt[3]{2})}$.
      
    \item
      Die Aussage ist falsch:
      Das Polynoms $f(X) = X^4 - 2 \in \Rational[X]$ hat $\sqrt[4]{2}$ als Nullstelle, und das Polynom $g(X) = X^{10} - 6 \in \Rational[X]$ hat $\sqrt[10]{6}$ als Nullstelle.
      Die beiden Polynome $f$ und $g$ sind normiert und nach Eisenstein mit dem Primelement $2 \in \Integer$ irreduzibel.
      Folglich ist $f$ das Minimalpolynom von $\sqrt[4]{2}$ und $g$ das Minimalpolynom von $\sqrt[10]{6}$, und somit $[\Rational(\sqrt[4]{2}) : \Rational] = \deg f = 4$ und $[\Rational(\sqrt[10]{6}) : \Rational] = \deg g = 10$.
      Wäre $\sqrt[4]{2} \in \Rational(\sqrt[10]{6})$, so wäre $[\Rational(\sqrt[4]{2}) : \Rational]$ ein Teiler von $[\Rational(\sqrt[10]{6} : \Rational)]$; da $4 \nmid 10$ ist dies nicht der Fall.
      
    \item
      Die Aussage ist wahr:
      Siehe Übung~\ref{question: restriction and transitivity of normality}.
      
    \item
      Die Aussage ist falsch:
      Siehe Übung~\ref{question: restriction and transitivity of normality}.
      
    \item
      Die Aussage ist falsch, denn sonst wäre
      \[
              3
        =     [\Field_{2^3} : \Field_2]
        =     [\Field_8 : \Field_2]
        \mid  [\Field_{32} : \Field_2]
        =     [\Field_{2^5} : \Field_2]
        =     5.
      \]
      
    \item
      Die Aussage ist falsch:
      Sie gilt genau dann, wenn $L/K$ algebraisch ist, siehe Übung~\ref{question: transitivity of algebraic closure}.
      Als Gegenbeispiel betrachte man somit beispielsweise $L = K(t)$.
      
    \item
      Die Aussage ist wahr:
      Sie wurde auf einem der Übungszettel gezeigt.
      
    \item
      Die Aussage ist wahr:
      Sind $\alpha_1, \dotsc, \alpha_d \in L$ mit $d \leq n$ die paarweise verschiedenen Nullstellen von $f$, so muss jedes $\sigma \in \Gal(L/K)$ die Nullstellen von $f$ permutieren.
      Also gibt es eine Einbettung $\varphi \colon \Gal(L/K) \to S_d$, $\sigma \mapsto \pi_\sigma$ wobei $\pi_\sigma \in S_d$ die eindeutige Permutation ist, so dass $\sigma(\alpha_i) = \alpha_{\pi_\sigma(i)}$ für alle $i = 1, \dotsc, d$.
      Da $\im \varphi \subseteq S_d$ eine Untergruppe ist, erhalten wir, dass
      \[
              |{\Gal(L/K)}|
        =     |{\im \varphi}|
        \mid  |S_d|
        =     d!
        \mid  n!.
      \]
    
    \item
      Die Aussage ist wahr:
      Da $L_1/K$ und $L_2/K$ algebraisch sind beide Erweiterungen von algebraischen ELementen erzeugt, also $L_1 = K(\alpha_i \mid i \in I)$ und $L_2 = K(\beta_j \mid j \in J)$ für algebraische Elemente $\alpha_i \in L_1$ und $\beta_j \in L_2$.
      Dann gilt $L_1 L_2 = K(\alpha_i, \beta_j \mid i \in I, j \in J)$, weshalb auch $L_1 L_2/K$ von algebraischen Elementen erzeugt wird, also algebraisch ist.
  \end{enumerate}
\end{solution}


\begin{question}
  Es sei $\alpha \in \Complex$ eine Nullstelle des Polynoms $X^3 - 6 X^2 + 9 X + 3 \in \Rational[X]$.
  \begin{enumerate}
    \item
      Bestimmen Sie eine $\Rational$-Basis von $\Rational(\alpha)$.
    \item
      Drücken Sie $\alpha^5$ und $3 \alpha^4 - 2 \alpha^3 + 1$ in der obigen Basis aus.
    \item
      Zeigen Sie, dass $\alpha + 2 \neq 0$ und drücken Sie $1/(\alpha + 2)$ in der obigen Basis aus.
  \end{enumerate}
\end{question}


\begin{solution}
  \begin{enumerate}
    \item
      Das Polynom $f(X) \coloneqq X^3 - 6 X^2 + 9 X + 3 \in \Rational[X]$ ist nach Eisenstein bezüglich des Primelements $3 \in \Integer$ irreduzibel.
      Also ist $f$ das Minimalpolynom von $\alpha$, und somit $[\Rational(\alpha) : \Rational] = \deg f = 3$.
      Deshalb ist $(1, \alpha, \alpha^2)$ eine $\Rational$-Basis von $\Rational(\alpha)$.
    
    \item
      Da $\alpha$ eine Nullstelle von $f$ ist, gilt $\alpha^3 = 6 \alpha^2 - 9 \alpha - 3$.
      Somit gilt auch $\alpha^5 = 6 \alpha^4 - 9 \alpha^3 - 3 \alpha^2$ und $\alpha^4 = 6 \alpha^3 - 9 \alpha^2 - 3 \alpha$.
      Durch sukzessives Einsetzen ergibt sich, dass
      \[
          \alpha^5
        = 6 \alpha^4 - 9 \alpha^3 - 3 \alpha^2
        = 27 \alpha^3 - 57 \alpha^2 - 18 \alpha
        = 105 \alpha^2 - 261 \alpha - 81.
      \]
      Alternativ ergibt sich auch durch Polynomdivision, dass
      \[
          \alpha^5
        = (\alpha^2 + 6 \alpha + 27)(\alpha^3 - 6 \alpha^2 + 9 \alpha + 3) + 105 \alpha^2 - 261 \alpha - 81,
      \]
      wobei der erste Summand verschwindet, da $\alpha^3 - 6 \alpha^2 + 9 \alpha + 3 = 0$ gilt.
      Durch sukszessives Einsetzen ergibt sich auch, dass
      \[
          3 \alpha^4 - 2 \alpha^3 + 1
        = 16 \alpha^3 - 27 \alpha^2 - 9 \alpha + 1
        = 69 \alpha^2 - 153 \alpha - 47.
      \]
      Alternativ ergibt sich mithilfe von Polynomdivision, dass
      \[
          3 \alpha^4 - 2 \alpha^3 + 1
        = (3 \alpha + 16)(\alpha^3 - 6 \alpha^2 + 9 \alpha + 3) + 69 \alpha^2 - 153 \alpha - 47,
      \]
      wobei der erste Summand verschwindet, da $\alpha^3 - 6 \alpha^2 + 9 \alpha + 3 = 0$ gilt.
    
    \item
      Es gibt mehrere Möglichkeiten um einzusehen, dass $\alpha + 2 \neq 0$.
      \begin{itemize}
        \item
          Dann wäre $\alpha = -2$, aber $-2$ ist keine Nullstelle von $f$.
        \item
          Dann wäre $\alpha = -2$, weshalb $f$ eine rationale Nullstelle hätte, was im Widerspruch zur Irreduziblität von $f$ steht.
        \item
          Dann wäre $2 \cdot 1 + 1 \cdot \alpha + 0 \cdot \alpha^2 = 0$ eine nicht-triviale Linearkombination der $0$, was der linearen Unabhängigkeit von $(1, \alpha, \alpha^2)$ über $\Rational$ widerspricht.
      \end{itemize}
      Mithilfe des euklidischen Algorithmus ergibt sich, dass
      \[
        -(\alpha^3 - 6 \alpha^2 + \alpha + 3) + (\alpha + 2)(\alpha^2 - 8 \alpha + 25) = 47,
      \]
      und somit, dass
      \[
          \frac{1}{\alpha + 2}
        = \frac{1}{47} \alpha^2 - \frac{8}{47} \alpha + \frac{25}{47}.  
      \]
  \end{enumerate}
\end{solution}


\begin{question}
  Es sei $L \coloneqq \Rational(\sqrt{2} + \sqrt{3})$.
  \begin{enumerate}
    \item
      Zeigen Sie, dass $\sqrt{2}, \sqrt{3} \in L$ und folgern Sie, dass $L = \Rational(\sqrt{2}, \sqrt{3})$.
    \item
      Bestimmen Sie den Grad der Erweiterung $L/\Rational$.
    \item
      Zeigen Sie, dass $L/\Rational$ galoisch ist.
    \item
      Bestimmen Sie $\Gal(L/\Rational)$, und entscheiden Sie, ob $\Gal(L/\Rational)$ abelsch ist.
  \end{enumerate}
\end{question}


\begin{solution}
  \begin{enumerate}
    \item
      Es gilt $(\sqrt{2} + \sqrt{3})^3 = 11 \sqrt{2} + 9 \sqrt{3}$ und deshalb
      \[
            \sqrt{2}
        =   \frac{1}{2}\left( (\sqrt{2} + \sqrt{3})^2 - 9(\sqrt{2} + \sqrt{3}) \right)
        \in L.
      \]
      Somit gilt auch $\sqrt{3} = (\sqrt{2} + \sqrt{3}) - \sqrt{3} \in L$.
      Dass $L \subseteq \Rational(\sqrt{2}, \sqrt{3})$ ist klar, und dass $\Rational(\sqrt{2}, \sqrt{3}) \subseteq L$ gilt, folgt aus $\Rational \subseteq L$ und $\sqrt{2}, \sqrt{3} \in L$.
      
    \item
      Wir betrachten die Zwischenerweiterung $\Rational \subseteq \Rational(\sqrt{2}) \subseteq L$.
      
      Das Minimalpolynom von $\sqrt{2}$ über $\Rational$ ist $f(X) = X^2 - 2 \in \Rational[X]$, denn $f$ ist normiert, nach Eisenstein irreduzibel, und hat $\sqrt{2}$ ist als Nullstelle.
      Deshalb gilt $[\Rational(\sqrt{2}) : \Rational] = 2$.
      
      Da $\sqrt{3}$ eine Nullstelle des Polynoms $g(X) = X^2 - 3 \in \Rational(\sqrt{2})[X]$ ist, gilt die Abschätzung $[L : \Rational(\sqrt{2})] \leq 2$.
      Wäre $[L : \Rational(\sqrt{2})] < 2$, also $[L : \Rational(\sqrt{2})] = 1$ und somit $L = \Rational(\sqrt{2})$, so gebe es $a, b \in \Rational$ mit $\sqrt{3} = a + b \sqrt{2}$ (denn $\{1, \sqrt{2}\}$ ist eine $\Rational$-Basis von $\Rational(\sqrt{2})$, da $[\Rational(\sqrt{2}) : \Rational] = 2$).
      Deshalb würde dann
      \[
          3
        = \sqrt{3}^2
        = (a + b \sqrt{2})^2
        = a^2 + 2 b^2 + ab \sqrt{3}
      \]
      gelten.
      Es müsste $a \neq 0$ gelten, denn sonst wäre $\sqrt{3/2} = b \in \Rational$, und es müsste auch $b \neq 0$ gelten, denn sonst wäre $\sqrt{3} = a \in \Rational$.
      Also wäre bereits $\sqrt{3} = (3 - a^2 - 2b^2)/(ab) \in \Rational$, was aber nicht gilt.
      
      Es muss also auch $[L : \Rational(\sqrt{2})] = 2$ gelten, und somit insgesamt
      \[
          [L : \Rational]
        = [L : \Rational(\sqrt{2})] [\Rational(\sqrt{2}) : \Rational]
        = 2 \cdot 2
        = 4.
      \]
      
    \item
      Es gilt $\Rational(\sqrt{2}, \sqrt{3}) = \Rational(\sqrt{2}, \sqrt{3}, -\sqrt{2}, -\sqrt{3})$, also wird $L/\Rational$ von den Nullstellen des Polynoms $f(X) = (X^2 - 2)(X^2 - 3) \in \Rational[X]$ erzeugt.
      Somit ist $L$ der Zerfällungskörper von $f$ über $\Rational$.
      Da die Nullstellen von $f$ paarweise verschieden sind, ist $f$ separabel.
      Also ist $L$ als Zerfällungskörper des separablen Polynoms $f$ bereits galoisch.
    
    \item
      Da $L/\Rational$ galoisch ist, wissen wir, dass $|{\Gal(L / \Rational)}| = [L : \Rational] = 4$.
      Außerdem muss jedes $\sigma \in \Gal(L : \Rational)$ die Nullstellen der rationalen Polynome $X^2 - 2, X^2 - 3 \in \Rational[X]$ permutieren;
      es muss also $\sigma(\sqrt{2}) = \pm \sqrt{2}$ und $\sigma(\sqrt{3}) = \pm \sqrt{3}$.
      Da $L$ von $\sqrt{2}$ und $\sqrt{3}$ erzeugt wird, ist $\sigma$ durch die beiden Werte $\sigma(\sqrt{2})$ und $\sigma(\sqrt{3})$ auch schon eindeutig bestimmt.
      
      Zusammen mit $|{\Gal(L / \Rational)}| = 4$ erhalten wir hieraus, dass die vier Automorphismen $\sigma_1, \sigma_2, \sigma_3, \sigma_4 \colon \Gal(L / \Rational)$ durch
      \begin{gather*}
        \sigma_1 \colon
        \left\{
          \begin{array}{ccr}
            \sqrt{2} & \mapsto  & \sqrt{2}, \\
            \sqrt{3} & \mapsto  & \sqrt{3},
          \end{array}
        \right.
        \quad
        \sigma_2 \colon
        \left\{
          \begin{array}{ccr}
            \sqrt{2} & \mapsto  & -\sqrt{2},  \\
            \sqrt{3} & \mapsto  &  \sqrt{3},
          \end{array}
        \right.
        \\
        \sigma_3 \colon
        \left\{
          \begin{array}{ccr}
            \sqrt{2} & \mapsto  &  \sqrt{2},  \\
            \sqrt{3} & \mapsto  & -\sqrt{3},
          \end{array}
        \right.
        \quad
        \sigma_4 \colon
        \left\{
          \begin{array}{ccr}
            \sqrt{2} & \mapsto  & -\sqrt{2},  \\
            \sqrt{3} & \mapsto  & -\sqrt{3},
          \end{array}
        \right.
      \end{gather*}
      gegeben sind.
      Insbesondere erhalten wir, dass $\Gal(L / \Rational) \cong \Integer/2 \times \Integer/2$,
      weshalb $\Gal(L / \Rational)$ abelsch ist.
  \end{enumerate}
\end{solution}



\begin{question}
  Es sei $L \coloneqq \Rational(\sqrt[4]{2}, i)$.
  \begin{enumerate}
    \item
      Bestimmen Sie den Grad der Erweiterun $L/\Rational$.
    \item
      Zeigen Sie, dass $L/\Rational$ galoisch ist.
    \item
      Bestimmen Sie $\Gal(L / \Rational)$.
    \item
      Entscheiden Sie, ob $\Gal(L / \Rational)$ abelsch ist.
  \end{enumerate}
\end{question}


\begin{solution}
  \begin{enumerate}
    \item
      Wir betrachten den Zwischenkörper $\Rational \subseteq \Rational(\sqrt[4]{2}) \subseteq L$.
      
      Das Polynom $f(X) = X^4 - 2 \in \Rational[X]$ hat $\sqrt[4]{2}$ als Nullstelle, ist normiert, und nach Eisenstein mit $2 \in \Integer$ irreduzibel.
      Also ist $f$ das Minimalpolynom von $\sqrt[4]{2}$ über $\Rational$ und somit $[\Rational(\sqrt[4]{2}) : \Rational] = 4$.
      
      Da $i$ eine Nullstelle des Polynoms $X^2 + 1 \in \Rational(\sqrt[4]{2})[X]$ ist, gilt $[L : \Rational(\sqrt[4]{2})] \leq 2$.
      Wäre $[L : \Rational(\sqrt[4]{2})] = 1$, also $L = \Rational(\sqrt[4]{2})$, so wäre $i \in \Rational(\sqrt[4]{2})$.
      Dies ist aber nicht der Fall, da $\Rational(\sqrt[4]{2}) \subseteq \Real$.
      also ist $[L : \Rational(\sqrt[4]{2})] = 2$.
      
      Somit gilt $[L : \Rational] = [L : \Rational(\sqrt[4]{2})] [\Rational(\sqrt[4]{2}) : \Rational] = 4$.
      
    \item
      Wir betrachten erneut das Polynom $f(X) = X^4 - 2 \in \Rational[X]$.
      Die Nullstellen dieses Polynoms sind $\sqrt[4]{2}$, $i\sqrt[4]{2}$, $-\sqrt[4]{2}$ und $-i\sqrt[4]{2}$.
      Da $i = (i\sqrt[4]{2})/\sqrt[4]{2}$ gilt, folgt, dass
      \[
          L
        = \Rational(\sqrt[4]{2}, i)
        = \Rational(\sqrt[4]{2}, i\sqrt[4]{2}, -\sqrt[4]{2}, -i\sqrt[4]{2}).
      \]
      Also wird $L$ von den Nullstellen von $f$ erzeugt, ist also ein Zerfällungskörper von $f$.
      Da die Nullstellen von $f$ paarweise verschieden sind, ist $f$ separabel.
      Also ist $L$ Zerfällungskörper eines separablen Polynoms, und somit galoisch.
    
    \item
      Da $L/\Rational$ Galoisch ist gilt $|{\Gal(L / \Rational)}| = [L : \Rational] = 8$.
      Da $\sigma \in \Gal(L / \Rational)$ die Nullstellen der beiden Polynome $X^4 - 2, X^2 + 1 \in \Rational[X]$ jeweils permutieren muss, gelten $\sigma(\sqrt[4]{2}) \in \{ \sqrt[4]{2}, i\sqrt[4]{2}, -\sqrt[4]{2}, -i\sqrt[4]{2} \}$ und $\sigma(i) = \pm 1$.
      Da $L$ von $\sqrt[4]{2}$ und $i$ erzeugt wird ist $\sigma$ durch die beiden Werte $\sigma(\sqrt[4]{2})$ und $\sigma(i)$ auch schon eindeutig bestimmt.
      
      Zusammen mit $|{\Gal(L / \Rational)}| = 8$ erhalten wir, dass die Automorphismen $\sigma_{r,s} \in \Gal(L / \Rational)$ mit $r = 1, 2, 3, 4$ und $s = 1, 2$ durch
      \begin{align*}
        \sigma_{1,1}
        &\colon
        \left\{
          \begin{array}{rcl}
            \sqrt[4]{2} & \mapsto & \sqrt[4]{2},  \\
            i           & \mapsto & i,
          \end{array}
        \right.
        &
        \sigma_{1,2}
        &\colon
        \left\{
          \begin{array}{rcl}
            \sqrt[4]{2} & \mapsto & \phantom{-}\sqrt[4]{2},  \\
            i           & \mapsto &           -i,
          \end{array}
        \right.
        \\
        \sigma_{2,1}
        &\colon
        \left\{
          \begin{array}{rcl}
            \sqrt[4]{2} & \mapsto & i\sqrt[4]{2},  \\
            i           & \mapsto & i,
          \end{array}
        \right.
        &
        \sigma_{2,2}
        &\colon
        \left\{
          \begin{array}{rcl}
            \sqrt[4]{2} & \mapsto & \phantom{-}i\sqrt[4]{2},  \\
            i           & \mapsto &           -i,
          \end{array}
        \right.
        \\
        \sigma_{3,1}
        &\colon
        \left\{
          \begin{array}{rcl}
            \sqrt[4]{2} & \mapsto &           -\sqrt[4]{2},  \\
            i           & \mapsto & \phantom{-}i,
          \end{array}
        \right.
        &
        \sigma_{3,2}
        &\colon
        \left\{
          \begin{array}{rcl}
            \sqrt[4]{2} & \mapsto & -\sqrt[4]{2},  \\
            i           & \mapsto & -i,
          \end{array}
        \right.
        \\
        \sigma_{4,1}
        &\colon
        \left\{
          \begin{array}{rcl}
            \sqrt[4]{2} & \mapsto &           -i\sqrt[4]{2},  \\
            i           & \mapsto & \phantom{-}i,
          \end{array}
        \right.
        &
        \sigma_{4,2}
        &\colon
        \left\{
          \begin{array}{rcl}
            \sqrt[4]{2} & \mapsto & -i\sqrt[4]{2},  \\
            i           & \mapsto & -i,
          \end{array}
        \right.
      \end{align*}
      gegeben sind.
      
    \item
      Die Gruppe $\Gal(L / \Rational)$ ist nicht abelsch, denn
      \begin{align*}
            \left( \sigma_{(2,1)} \circ \sigma_{(1,2)} \right)\left( \sqrt[2]{4} \right)
        &=  \sigma_{(2,1)}\left( \sigma_{(1,2)}\left( \sqrt[2]{4} \right) \right)
         =  \sigma_{(2,1)}\left( \sqrt[2]{4} \right)
         =  i\sqrt[2]{4}
      \shortintertext{sowie}
            \left( \sigma_{(1,2)} \circ \sigma_{(2,1)} \right)\left( \sqrt[2]{4} \right)
        &=  \sigma_{(1,2)}\left( \sigma_{(2,1)}\left( \sqrt[2]{4} \right) \right)
         =  \sigma_{(1,2)}\left( i\sqrt[2]{4} \right)
         =  -i\sqrt[2]{4}.
      \end{align*}
    (Die Gruppe $\Gal(L / \Rational)$ ist isomorph zu der Diedergruppe $D_8$, die Symmetriegruppe eines Quadrats)
  \end{enumerate}
\end{solution}


\begin{question}
  Es sei $\zeta \in \Complex$ eine dritte primitive Einheitswurzel (etwa $\zeta = e^{2 \pi i/3}$) und $L \coloneqq \Rational(\sqrt[3]{3}, \zeta)$.
  \begin{enumerate}
    \item
      Zeigen Sie, dass $L/\Rational$ Galoisch ist.
    \item
      Bestimmen Sie den Grad $[L : \Rational]$.
    \item
      Bestimmen Sie die Gruppe $\Gal(L/\Rational)$ und entscheiden Sie, ob sie abelsch ist.
  \end{enumerate}
\end{question}


\begin{solution}
  \begin{enumerate}
    \item
      Es gilt $L = \Rational(\sqrt[3]{3}, \zeta \sqrt[3]{3}, \zeta^2 \sqrt[3]{3})$ da $\zeta = (\zeta \sqrt[3]{3}) / \sqrt[3]{3})$.
      Also wird $L$ über $\Rational$ von den Nullstellen des Polynoms $f(X) \coloneqq X^3 - 3 \in \Rational[X]$ erzeugt, ist also ein Zerfällungskörper von $f$.
      Die Nullstellen von $f$ sind paarweise verschieden, also ist $f$ separabel.
      Als Zerfällungskörper eines separablen Polynoms ist $L/\Rational$ galoisch.
      
    \item
      Wir betrachten den Zwischenkörper $\Rational \subseteq \Rational(\sqrt[3]{3}) \subseteq L$.
      Nach Eisenstein bezüglich $3 \in \Integer$ ist $f$ irreduzibel.
      Also ist $f$ das Minimalpolynom von $\sqrt[3]{3}$, also $[\Rational(\sqrt[3]{3}) : \Rational] = \deg f = 3$.
      Außerdem gilt $\zeta \notin \Real \supseteq L$.
      Es gilt nun (mindestens) zwei Argumentationen:
      \begin{itemize}
        \item
          Aus $\zeta \notin \Rational(\sqrt[3]{3})$ ergibt sich, dass $[L : \Rational(\sqrt[3]{3})] \geq 2$.
          Zusammen mit $[\Rational(\sqrt[3]{3}) : \Rational] = 3$ erhalten wir, dass $[L : \Rational] \geq 6$ gilt.
          Andererseits ist $L$ ein Zerfällungskörper eines kubischen Polynoms und somit $[L : \Rational] \leq 3! = 6$.
          Zusammen erhalten wir, dass $[L : \Rational] = 2$.
        \item
          Es gilt $L = \Rational(\sqrt[3]{3}, \zeta) = \Rational(\sqrt[3]{3}, \zeta \sqrt[3]{3}) = \Rational(\sqrt[3]{3})(\zeta \sqrt[3]{3})$.
          Das Polynom
          \[
                      g(X)
            \coloneqq (X - \zeta \sqrt[3]{3})(X - \zeta^2 \sqrt[3]{3})
            =         f(X)/(X-\sqrt[3]{3})
            \in       \Rational(\sqrt[3]{3})[X]
          \]
          ist irreduzibel, da wegen $\zeta \notin \Rational(\sqrt[3]{3})$ auch $\pm \zeta \sqrt[3]{3} \notin \Rational(\sqrt[3]{3})$.
          Somit ist $g$ das Minimalpolynom von $\zeta \sqrt[3]{3}$ über $\Rational(\sqrt[3]{3})$ und deshalb $[L : \Rational(\sqrt[3]{3})] = 2$.
          Zusammen mit $[\Rational(\sqrt[3]{3}) : \Rational] = 3$ erhalten wir, dass $[L : \Rational] = 6$ gilt.
      \end{itemize}
          
    \item
      Es ist $L$ der Zerfällungskörper von $f$ und $[L : \Rational] = 6 = 3! = (\deg f)!$.
      Aus der Vorlesung ist bekannt, dass deshalb bereits $\Gal(L/\Rational) \cong S_3$ gilt.
      Aus $\Gal(L/\Rational) \cong S_3$ erhalten wir inbesondere, dass $\Gal(L/\Rational)$ nicht abelsch ist.
      
      Von Hand lässt sich die Aussage wie folgt nachrechnen:
      Jedes $\sigma \in \Gal(L/\Rational)$ muss die Nullstellen von $f$, also $z_1 = \sqrt[3]{3}$, $z_2 = \zeta \sqrt[3]{3}$ und $z_3 = \zeta^2 \sqrt[3]{3}$, permutieren.
      Da $L$ bereits von diesen Nullstellen erzeugt wird, ergibt sich eine Einbettung $\varphi \colon \Gal(L/\Rational) \to S_3$, wobei $\pi = \varphi(\sigma)$ die eindeutige Permutation mit $\sigma(z_i) = z_{\pi(i)}$ für alle $i = 1, 2, 3$ ist.
      Da $L/\Rational$ galoisch ist, gilt dabei $|{\Gal(L/\Rational)}| = [L : \Rational] = 6 = 3! = |S_3|$, weshalb $\varphi$ bereits ein Isomorphismus ist.
  \end{enumerate}
\end{solution}


\begin{question}
  Es sei $f(X) \coloneqq X^3 - 2 X^2 - X + 1 \in \Rational[X]$ und $\alpha \in \Complex$ eine Nullstelle von $f$.
  \begin{enumerate}
    \item
      Bestimmen Sie den Grad von $\Rational(\alpha) / \Rational$.
    \item
      Zeigen Sie, dass auch $\alpha(\alpha-2)$ eine Nullstelle von $f$ ist.
    \item
      Folgern Sie, dass $\Rational(\alpha)/\Rational$ galoisch ist.
    \item
      Bestimmen Sie $\Gal(\Rational(\alpha)/\Rational)$ bis auf Isomorphie.
  \end{enumerate}
\end{question}


\begin{solution}
  \begin{enumerate}
    \item
      Durch Reduzieren bezüglich des Primelements $2 \in \Integer$ erhält man das kubische Polynom $\tilde{f}(X) = X^3 + X + 1 \in \Field_2[X]$.
      Da $\tilde{f}(0) = \tilde{f}(1) = 1$ hat $\tilde{f}$ keine Nullstellen.
      Da $\tilde{f}$ kubisch ist, ist $\tilde{f}$ somit bereits irreduzibel.
      Also ist auch $f$ schon irreduzibel.
      Folglich ist $f$ bereits das Minimalpolynom von $\alpha$.
      Somit gilt $[\Rational(\alpha) : \Rational] = \deg f = 3$.
      
    \item
      Es gilt
      \begin{align*}
        \label{equation: polynomial of degree 6}
            f(\alpha(\alpha-2))
        &=  \alpha^3 (\alpha-2)^3 - 2 \alpha^2 (\alpha-2)^2 - \alpha(\alpha-2) + 1
        \\
        &=  \alpha^6 - 6 \alpha^5 + 10 \alpha^4 - 9 \alpha^2 + 2 \alpha + 1.
      \end{align*}
      Aus $0 = f(\alpha) = \alpha^3 - 2 \alpha^2 - \alpha + 1$ erhalten wir, dass $\alpha^3 = 2 \alpha^2 + \alpha - 1$.
      Somit gelten auch $\alpha^4 = 2 \alpha^3 + \alpha^2 - \alpha$, $\alpha^5 = 2 \alpha^4 + \alpha^3 - \alpha^2$ und $\alpha^6 = 2 \alpha^5 + \alpha^4 - \alpha^3$.
      Einsetzen von $\alpha^6 = 2 \alpha^5 + \alpha^4 - \alpha^3$ liefert
      \[
          \alpha^6 - 6 \alpha^5 + 10 \alpha^4 - 9 \alpha^2 + 2 \alpha + 1
        = -4 \alpha^5 + 11 \alpha^4 - \alpha^3 - 9 \alpha^2 + 2 \alpha + 1.
      \]
      Einsetzen von $\alpha^5 = 2 \alpha^4 + \alpha^3 - \alpha^2$ liefert
      \[
          -4 \alpha^5 + 11 \alpha^4 - \alpha^3 - 9 \alpha^2 + 2 \alpha + 1
        = 3 \alpha^4 - 5 \alpha^3 - 5 \alpha^2 + 2 \alpha + 1.
      \]
      Einsetzen von $\alpha^4 = 2 \alpha^3 + \alpha^2 - \alpha$ liefert schließlich
      \[
          3 \alpha^4 - 5 \alpha^3 - 5 \alpha^2 + 2 \alpha + 1
        = \alpha^3 - 2 \alpha^2 - \alpha + 1
        = 0.
      \]
      Ingesamt gilt also $f(\alpha (\alpha - 2)) = \dotsb = 0$.
      Alternativ ergibt sich mithilfe von Polynomdivision, dass
      \[
          \alpha^6t - 6 \alpha^5 + 10 \alpha^4 - 9 \alpha^2 + 2 \alpha + 1
        = (\alpha^3 - 4 \alpha^2 + 3 \alpha + 1)\underbrace{(\alpha^3 - 2 \alpha^2 - \alpha + 1)}_{=0}
        = 0.
      \]
      
    \item
      Da $\ringchar \Rational = 0$ ist $f \in \Rational[X]$ als irreduzibles Polynom bereits separabel.
      Inbesondere sind die Nullstellen $\alpha$ und $\alpha (\alpha - 2)$ verschieden.
      Also hat $f$ in $\Rational(\alpha)$ zwei verschiedene Nullstellen; da $f$ kubisch ist, zerfällt $f$ deshalb über $\Rational(\alpha)$ bereits in Linearfaktoren.
      Es ist also $\Rational(\alpha)$ der Zerfällungskörper des separablen Polynoms $f$, und die Erweiterung $\Rational(\alpha)/\Rational$ somit galoisch.
%       
    \item
      Da $\Rational(\alpha)/\Rational$ galoisch ist, gilt $|\Gal(\Rational(\alpha)/\Rational)| = [\Rational(\alpha) : \Rational] = 3$.
      Da $\Integer/3$ bis auf Isomorphie die einzige dreielementige Gruppe ist, gilt also $\Gal(\Rational(\alpha)/\Rational) \cong \Integer/3$.
  \end{enumerate}
\end{solution}


\begin{question}
  Es sei $p > 0$ prim und $f(X) \coloneqq X^p - 2 \in \Rational[X]$.
  Es sei $\zeta \in \Complex$ eine primitive $p$-te Einheitswurzel (etwa $\zeta = e^{2 \pi i/p}$).
  Es sei $L \coloneqq \Rational(\sqrt[p]{2}, \zeta)$.
  \begin{enumerate}
    \item
      Zeigen Sie, dass $L$ ein Zerfällungskörper von $f$ ist.
    \item
      Folgern Sie, dass $L/\Rational$ galoisch ist.
    \item
      Zeigen Sie, dass $[\Rational(\zeta) : \Rational] = p-1$.
      \\
      (\emph{Hinweis}:
       Man betrachte Übung~\ref{question: irreducibility of polynomials}.)
    \item
      Folgern Sie, dass $[L : \Rational] = p(p-1)$.
      \\
      (\emph{Hinweis}:
       Zeigen Sie, dass $p, p-1 \mid [L : \Rational]$.
    \item
      Bestimmen Sie $\Gal(L / \Rational)$.
    \item
      Entscheiden Sie, ob $\Gal(L / \Rational)$ abelsch ist.
  \end{enumerate}
\end{question}


\begin{question}
  Es sei $K$ ein Körper und $L/K$ eine endliche Galoiserweiterung.
  Es sei $f \in K[X]$ und es seien $\alpha_1, \dotsc, \alpha_n \in L$ die paarweise verschiedenen Nullstellen von $f$.
  Zeigen Sie für das Polynom $g(X) \coloneqq \prod_{i=1}^n (X - \alpha_i) \in L[X]$, dass bereits $g \in K[X]$ gilt.
  \\
  (\emph{Hinweis}:
   Überlegen Sie sich, dass die Koeffizienten von $g$ invariant unter der Galoisgruppe $\Gal(L/K)$ sind.)
\end{question}


\begin{solution}
  Es sei $G \coloneqq \Gal(L/K)$.
  Jedes $\sigma \in G$ induziert einen Automorphismus $\varphi_\sigma \colon L[X] \to L[X]$, $\sum_i a_i X^i \mapsto \sum_i \sigma(a_i) X^i$.
  Jedes $\sigma \in G$ muss die Nullstellen von $f$ permutieren;
  es gibt also eine Permutation $\pi_\sigma \in S_n$ mit $\sigma(\alpha_i) = \alpha_{\pi_\sigma(i)}$ für alle $i = 1, \dotsc, n$.
  Für jedes $\sigma \in G$ gilt deshalb
  \begin{align*}
        \varphi_\sigma( g(X) )
    &=  \varphi_\sigma\left( \prod_{i=1}^n X - \alpha_i \right)
     =  \prod_{i=1}^n ( X - \sigma(\alpha_i) )
    \\
    &=  \prod_{i=1}^n ( X- \alpha_{\pi_\sigma(i)} )
     =  \prod_{j=1}^n ( X - \alpha_j )
     =  g(X).
  \end{align*}
  Also ist $g$ invariant unter allen $\varphi_\sigma$ mit $\sigma \in G$.
  Da die $\varphi_\sigma$ koeffizientenweise agieren, erhalten wir, dass alle Koeffizienten von $g$ invariant unter allen $\sigma \in G$ sind.
  Die Koeffizienten von $G$ liegen also im Fixkörper $L^G$;
  somit gilt $g \in L^G[X]$.
  Da $L/K$ galoisch ist, gilt dabei bereits $L^G = K$.
\end{solution}



\begin{question}
  Es sei $\sigma \in \Gal(\Real/\Rational)$.
  \begin{enumerate}
    \item
      Zeigen Sie, dass für alle $x \in \Real$ genau dann $x \geq 0$, wenn $\sigma(x) \geq 0$.
    \item
      Folgern Sie, dass $\sigma$ streng monoton steigend ist.
    \item
      Folgern Sie, dass $\sigma$ stetig ist.
    \item
      Folgern Sie, dass $\sigma = \id_\Real$.
  \end{enumerate}
  Das zeigt, dass $\Gal(\Real/\Rational) = 1$.
\end{question}


\begin{solution}
  \begin{enumerate}
    \item
      Eine reelle Zahl ist genau dann nicht-negativ, wenn sie eine Quadratzahl ist;
      diese Eigenschaft ist invariant unter Körperautomorphismen.
      
    \item
      Für alle $x, y \in \Real$ gilt
      \[
              x \geq y
        \iff  x - y \geq 0
        \iff  \sigma(x) - \sigma(y) \geq 0
        \iff  \sigma(x) \geq \sigma(y).
      \]
      Somit ist $\sigma$ monoton steigend;
      dass $\sigma$ bereits \emph{streng} monoton steigend ist ergibt sich aus der Injektivität von $\sigma$.
      
    \item
      Für alle $x, y \in \Real$ mit $x < y$ gilt nach dem vorherigen Aussagenteil, dass genau dann $x < z < y$, wenn 
      $\sigma(x) < \sigma(z) < \sigma(y)$; also bildet $\sigma$ offene Intervalle auf offene Intervalle ab.
      Da eine jede offene Menge $U \subseteq \real$ eine Vereinigung offener Intervalle ist, folgt daraus, dass auch $\sigma(U) \subseteq \real$ offen ist.
      Wendet man dieses Resultat auf $\sigma^{-1} \in \Gal(\Real/\Rational)$ an, so ergibt sich, dass für jede offene Teilmenge $U \subseteq \Real$ auch $\sigma^{-1}(U)$ offen ist.
    
    \item
      Da $\Rational \subseteq \Real$ dicht liegt, folgt aus $\sigma|_{\Rational} = \id_\Rational$ und der Stetigkeit von $\sigma$, dass bereits $\sigma = \id_\Real$ gilt.
  \end{enumerate}
\end{solution}


\begin{question}
  Es sei $p$ eine Primzahl und $f(X) \coloneqq X^p - 2 \in \Rational[X]$.
  Es sei $\zeta \in \Complex$ eine primitive $p$-te Einheitswurzel (etwa $\zeta = e^{2 \pi i/p}$).
  Es sei $L \coloneqq \Rational(\sqrt[p]{2}, \zeta)$.
  \begin{enumerate}
    \item
      Zeigen Sie, dass $L$ ein Zerfällungskörper von $f$ ist.
    \item
      Folgern Sie, dass $L/\Rational$ galoisch ist.
    \item
      Zeigen Sie, dass $[\Rational(\zeta) : \Rational] = p-1$.
      \\
      (\emph{Hinweis}:
       Man betrachte Übung~\ref{question: irreducibility of polynomials}.)
    \item
      Folgern Sie, dass $[L : \Rational] = p(p-1)$.
      \\
      (\emph{Hinweis}:
       Zeigen Sie, dass $p, p-1 \mid [L : \Rational]$.
    \item
      Bestimmen Sie $\Gal(L / \Rational)$.
    \item
      Entscheiden Sie in Abhängigkeit von $p$, ob $\Gal(L / \Rational)$ abelsch ist.
    \item
      Entscheiden Sie in Abhängigkeit von $p$, ob $\Gal(L / \Rational(\sqrt[p]{2}))$ normal in $\Gal(L/\Rational)$ ist.
    \item
      Entscheiden Sie in Abhängigkeit von $p$, ob $\Gal(L / \Rational(\zeta))$ normal in $\Gal(L/\Rational)$ ist.
  \end{enumerate}
\end{question}


\begin{solution}
  \begin{enumerate}
    \item
      Über $\Complex$ zerfällt $f$ in Linearfaktoren, und die Nullstellen sind $\sqrt[6]{2}, \zeta \sqrt[6]{2}, \dotsc, \zeta^{p-1} \sqrt[6]{2}$.
      Es gilt $L = \Rational(\sqrt[6]{2}, \zeta) = \Rational(\sqrt[6]{2}, \zeta \sqrt[6]{2}, \dotsc, \zeta^{p-1} \sqrt[6]{2}$ da $\zeta = (\zeta \sqrt[6]{2})/\sqrt[6]{2}$.
      Somit wird $L$ von den Nullstellen von $f$ erzeugt, ist also ein Zerfällungskörper vor $f$.
      
    \item
      Das Polynom $f$ ist separabel, da alle Nullstellen paarweise verschieden sind.
      Somit ist $L$ Zerfällungskörper des separablen Polynoms $f \in \Rational[X]$ galoisch über $\Rational$.
      
    \item
      Als $p$-te Einheitswurzel ist $\zeta$ eine Nullstelle des Polynoms $X^p - 1 \in \Rational[X]$.
      Dabei gilt $X^p - 1 = (X-1)(X^{p-1} + X^{p-2} + \dotsb + X + 1)$, und da $\zeta \neq 1$ gilt, erhalten wir, dass $\zeta$ bereits eine Nullstelle von $g(X)  = X^{p-1} + X^{p-2} + \dotsb + X + 1 \in \Rational[X]$ ist.
      Durch Eisenstein bezüglich der Primzahl $p$ ergibt sich, dass das Polynom $g(X+1)$ irreduzibel ist; somit ist auch $g$ irreduzibel (man siehe Übung~\ref{question: irreducibility of polynomials} und die zugehörigen Lösungen für eine detailiertere Rechnung).
      Also ist $g$ bereits das Minimalpolynom von $\zeta$, und somit $[\Rational(\zeta) : \Rational] = \deg g = p-1$.
      
    \item
      Das Minimalpolynom von $\sqrt[p]{2}$ über $\Rational$ ist $X^p - 2 \in \Rational[X]$, da dieses Polynom nach Eisenstein irreduzibel ist.
      Deshalb gilt $[\Rational(\sqrt[p]{2}) : \Rational] = \deg (X^p - 2) = p$.
      Außerdem gilt $[L : \Rational(\sqrt[p]{2})] \leq p-1$, da $L = \Rational(\sqrt[p]{2})(\zeta)$ gilt und $\zeta$ eine Nullstelle von $g(X) \in \Rational(\sqrt[p]{2})[X]$ ist.
      Somit gilt insgesamt
      \[
              [L : \Rational]
        =     [L : \Rational(\sqrt[p]{2})] [\Rational(\sqrt[p]{2}) : \Rational]
        \leq  p(p-1).
      \]
      Andererseits gilt $\Rational(\sqrt[p]{2}) \subseteq L$ und somit $p = [\Rational(\sqrt[p]{2}) : \Rational] \mid [L : \Rational]$, sowie analog auch $\Rational(\zeta) \subseteq L$ und somit $p-1 = [\Rational(\zeta) : \Rational] \mid [L : \Rational]$.
      Da $p$ prim ist sind $p$ und $p-1$ teilerfremd; also gilt bereits $p(p-1) \mid [L : \Rational]$ und somit auch $p(p-1) \leq [L : \Rational]$.
      
    \item
      Da $L/\Rational$ galoisch ist, wissen wir bereits, dass $|{\Gal(L/\Rational)}| = [L : \Rational] = p(p-1)$ gilt.
      Jedes $\sigma \in \Gal(L/\Rational)$ permutiert die Nullstellen von $f$, weshalb $\sigma(\sqrt[p]{2}) = \zeta^k \sqrt[p]{2}$ für ein eindeutiges $k \in \{0, \dotsc, p-1\}$ gilt.
      Außerdem muss $\sigma$ die Nullstellen von $g$, also die $p$-ten Einheitswurzeln, die verschieden von $1$ sind, permutieren, weshalb $\sigma(\zeta) = \zeta^\ell$ für ein eindeutiges $\ell \in \{1, \dotsc, p-1\}$ gilt.
      Da $\sigma$ durch die beiden Werte $\sigma(\sqrt[p]{2})$ und $\sigma(\zeta)$ bereits eindeutig bestimmt ist, erhalten wir zusammen mit $|{\Gal(L/\Rational)}| = p(p-1)$, dass $\Gal(L/\Rational) = \{\sigma_{k,\ell} \mid k = 1, \dotsc, p, \ell = 1, \dotsc, p-1\}$ gilt, wobei $\sigma_{k,\ell}$ durch
      \[
        \sigma_{k,\ell}
        \colon
        \left\{
          \begin{array}{rcl}
            \sqrt[p]{2} & \mapsto & \zeta^{k-1} \sqrt[p]{2},
            \\
            \zeta       & \mapsto & \zeta^\ell,
          \end{array}
        \right.
      \]
      eindeutig bestimmt ist.
    
    \item
      Für $p = 2$ gilt $|{\Gal(L/\Rational)}| = 2$ und $\Gal(L / \Rational) \cong \Integer/2$ ist abelsch.
      Für $p \neq 3$ Gruppe ist nicht abelsch, denn dann gilt $\sigma_{2,1} \sigma_{1,2} \neq \sigma_{1,2} \sigma_{2,1}$ da
      \[
              \sigma_{2,1}( \sigma_{1,2}( \sqrt[p]{2} ) )
        =     \sigma_{2,1}( \sqrt[p]{2} )
        =     \zeta \sqrt[p]{2}
        \neq  \zeta^2 \sqrt[p]{2}
        =     \sigma_{1,2}( \zeta \sqrt[p]{2} )
        =     \sigma_{1,2}( \sigma_{2,1}( \sqrt[p]{2} ) ).
      \]
      
    \item
      Für $p = 2$ ist $\Gal(L/\Rational)$ abelsch und somit jede Untergruppe normal.
      Wir betrachten daher im Folgenden nur den Fall $p \neq 2$.
      Die Untergruppe
      \[
                  H
        \coloneqq \Gal(L / \Rational(\sqrt[p]{2}) )
        =         \{ \sigma_{1,1}, \dotsc, \sigma_{1,p-1} \}
      \]
      ist dann nicht normal in $\Gal(L / \Rational)$.
      Es gibt (mindestens) zwei Möglichkeiten dies einzusehen:
      \begin{itemize}
        \item
          Es gilt $\sigma_{2,1}^{-1} = \sigma_{p,1}$ denn es gelten
          \begin{align*}
                \sigma_{2,1}( \sigma_{p,1}( \sqrt[p]{2} ) )
             =  \sigma_{2,1}( \zeta^{p-1} \sqrt[p]{2} )
            &=  \sigma_{2,1}( \zeta )^{p-1}  \sigma_{2,1}( \sqrt[p]{2} )
            \\
            &=  \zeta^{p-1} \zeta \sqrt[p]{2}
             =  \zeta^p \sqrt[p]{2}
             =  \sqrt[p]{2}
          \end{align*}
          sowie $\sigma_{2,1}(\sigma_{p,1}(\zeta))= \sigma_{2,1}(\zeta) = \zeta$, und somit $\sigma_{2,1} \sigma_{p,1} = \id_L$.
          Für $\sigma_{1,2} \in H$ gilt somit $\sigma_{2,1} \sigma_{1,2} \sigma_{2,1}^{-1} = \sigma_{2,1} \sigma_{1,2} \sigma_{p,1} = \sigma_{p,2} \notin H$, denn
          \begin{gather*}
            \begin{aligned}
                  \sigma_{2,1}( \sigma_{1,2}( \sigma_{p,1}( \sqrt[p]{2} ) ) )
              &=  \sigma_{2,1}( \sigma_{1,2}( \zeta^{p-1} \sqrt[p]{2} ) )
               =  \sigma_{2,1}( \sigma_{1,2}(\zeta)^{p-1} \sigma_{1,2}(\sqrt[p]{2}) )
              \\
              &=  \sigma_{2,1}( ((\zeta)^2)^{p-1} \sqrt[p]{2} )
               =  \sigma_{2,1}( \zeta^{2p-2} \sqrt[p]{2} )
              \\
              &=  \sigma_{2,1}(\zeta)^{2p-2} \sigma_{2,1}(\sqrt[p]{2})
               =  \zeta^{2p-2} \zeta \sqrt[p]{2}
              \\
              &=  \zeta^{2p-1} \sqrt[p]{2}
               =  \zeta^{p-1} \sqrt[p]{2}
            \end{aligned}
          \shortintertext{sowie}
              \sigma_{2,1}( \sigma_{1,2}( \sigma_{p,1}( \zeta ) ) )
            = \sigma_{2,1}( \sigma_{1,2}( \zeta ) )
            = \sigma_{2,1}( \zeta^2 )
            = \sigma_{2,1}(\zeta)^2
            = \zeta^2.
          \end{gather*}
        \item
          Wäre $H$ normal in $G$, so wäre nach dem Hauptsatz der Galoistheorie die Erweiterung $\Gal(\Rational(\sqrt[p]{2}) / \Rational)$ galoisch.
          Die Erweiterung ist aber nicht normal, denn $f$ ist irreduzibel in $\Rational[X]$ und hat eine Nullstelle in $\Rational(\sqrt[p]{2})$, zerfällt aber in $\Rational(\sqrt[p]{2})[X]$ nicht in Linearfaktoren, denn
          \begin{itemize}
            \item
              die Nullstellen $\zeta \sqrt[p]{2}, \dotsc, \zeta{p-1} \sqrt[p]{2}$ von $f$ liegen nicht in $\Real$ und somit auch nicht in $\Rational(\sqrt[p]{2})$,
          \end{itemize}
          bzw.\
          \begin{itemize}[resume]
            \item
              $L$ ist ein Zerfällungskörper von $f$, und es gilt $\Rational(\sqrt[p]{2}) \subsetneq L$.
          \end{itemize}
      \end{itemize}
      
    \item
       Die Gruppe
      \[
                  N
        \coloneqq \Gal(L / \Rational(\zeta) )
        =         \{ \sigma_{1,1}, \dotsc, \sigma_{p,1} \}
      \]
      ist normal in $\Gal(L / \Rational)$.
      Es gibt (mindestens) zwei Möglichkeiten dies einzusehen:
      \begin{itemize}
        \item
          Man bemerke, dass für alle $\sigma_{k, \ell}, \sigma_{k', \ell'} \in \Gal(L/\Rational)$ die Gleichheit
          \[
              \sigma_{k, \ell}( \sigma_{k', \ell'}( \zeta ) )
            = \sigma_{k, \ell}( \zeta )^{\ell'}
            = ((\zeta)^{\ell})^{\ell'}
            = \zeta^{\ell \ell'}
            = \zeta^{\ell' \ell}
            = \dotsb
            = \sigma_{k', \ell'}( \sigma_{k, \ell}( \zeta ) )
          \]
          gilt.
          Obwohl $\Gal(L/\Rational)$ für $p \neq 2$ nicht abelsch ist, lassen sich die Elemente beim Auswerten an $\zeta$ deshalb dennoch vertauschen.
          Für $\sigma \in \Gal(L/\Rational)$ und $\sigma' \in N$ gilt deshalb auch $\sigma \sigma' \sigma^{-1} \in N$, denn es gilt
          \[
              \sigma( \sigma'( \sigma^{-1}( \zeta ) ) )
            = \sigma( \sigma^{-1}( \sigma'( \zeta ) ) )
            = \sigma'( \zeta )
            = \zeta.
          \]
          
        \item
          Nach dem Hauptsatz der Galoistheorie ist $N$ genau dann Normal in $\Gal(L/\Rational)$, wenn $\Rational(\zeta)/\Rational$ galoisch ist.
          Hierfür bemerke man, dass $\Rational(\zeta)$ alle Potenzen $\zeta^k$ enthält, also alle $p$-ten Einheitswurzeln (denn $\zeta$ ist eine \emph{primitive} $p$-te Einheitswurzel).
          Also ist $\Rational(\zeta)$ der Zerfällungskörper des Kreisteilungspolynoms $X^p - 1 \in \Rational[X]$; dieses ist separabel, da die $p$-ten Einheitswurzeln paarweise verschieden sind.
          Somit ist $\Rational(\zeta)$ der Zerfällungskörper eines separablen Polynoms aus $\Rational[X]$, und $\Rational(\zeta)/\Rational$ somit galoisch.
      \end{itemize}
  \end{enumerate}
  
  \begin{remark*}
    Für die Untergruppen $N, H \subseteq \Gal(L/\Rational)$ ist also $N$ normal und $H \cap N = 1$.
    Es gilt außerdem $NH = G$, da $\sigma_{k,\ell} = \sigma_{k,1} \sigma_{1,\ell}$ für alle $k, \ell$ gilt, wobei $\sigma_{k,1} \in N$ und $\sigma_{1,\ell} \in N$ gilt.
    Insgesamt erhält man hierdurch, dass $\Gal(L/\Rational) = N \rtimes H$.
    Zusammen mit $N \cong \Integer/p$ und $H \cong (\Integer/p)^\times \cong \Integer/(p-1)$ erhält man somit, dass $\Gal(L/\Rational) \cong (\Integer/p) \rtimes (\Integer/(p-1)$.
    
    Für $p = 2$ ergibt sich somit, dass $\Gal(L/\Rational) \cong \Integer/2$, und für $p = 3$ ergibt sich, dass $\Gal(L/\Rational) \cong (\Integer/3) \rtimes (\Integer/2) \cong S_3$.
  \end{remark*}  
\end{solution}


\begin{question}
  \label{question: characterization of fields via its ideals}
  Zeigen Sie, dass für einen kommutativen Ring $K$ die folgenden Bedingungen äquivalent sind:
  \begin{enumerate}
    \item
      \label{enum: K is a field}
      $K$ ist ein Körper.
    \item
      \label{enum: K has exactly two ideals}
      $K$ hat genau zwei Ideale.
    \item
      \label{enum: The zero ideal is maximal}
      Das Nullideal in $K$ ist maximal.
  \end{enumerate}
\end{question}


\begin{solution}
  (\ref{enum: K is a field} $\implies$ \ref{enum: K has exactly two ideals})
  Da $K$ ein Körper ist gilt $0 \neq K$, also hat $K$ mindestens zwei Ideale.
  Ist $I \subseteq K$ ein Ideal mit $I \neq 0$, so gibt es ein $x \in I$ mit $x \neq 0$.
  Dann ist $x$ eine Einheit in $K$, somit $K = (x) \subseteq I$ und deshalb $I = K$.
  Also sind $0$ und $K$ die einzigen Ideale in $K$.
  
  (\ref{enum: K has exactly two ideals} $\implies$ \ref{enum: The zero ideal is maximal})
  Es muss $0 \neq K$, denn sonst wäre $0$ das einzige Ideal in $K$.
  Also sind $0$ und $K$ die einzigen beiden Ideale in $K$.
  Ist $I \subseteq K$ ein Ideal mit $0 \subsetneq I$, so muss bereits $I = K$.
  Also ist $0$ ein maximales Ideal.
  
  (\ref{enum: The zero ideal is maximal} $\implies$ \ref{enum: K is a field})
  Da $0 \subseteq K$ maximal ist, ergibt sich, dass $K \cong K/0$ ein Körper ist.
\end{solution}


\begin{question}
  Es sei $K$ ein algebraisch abgeschlossener Körper.
  Zeigen Sie, dass $K$ unendlich ist.
\end{question}


\begin{solution}
  Wäre $K$ endlich, so wäre
  \[
              p(T)
    \coloneqq 1 + \prod_{\lambda \in K} (T - \lambda)
    \in       K[T]
  \]
  ein Polynom positiven Grades ohne Nullstellen (denn $p(x) = 1$ für alle $x \in K$).
  Dies stünde im Widerspruch zur algebraischen Abgeschlossenheit von $K$.
\end{solution}


\begin{question}
  Es sei $K$ ein Körper und $p \in K[T]$ ein Polynom mit $\deg p \in \{2, 3\}$.
  Zeigen Sie, dass $p$ genau dann irreduzibel ist, wenn $p$ keine Nullstelle hat.
\end{question}


\begin{solution}
  Wäre $p$ reduzibel, so gebe $q_1, q_2 \in K[T]$ mit $p = q_1 q_2$ und $\deg q_1, \deg q_2 \geq 1$.
  Es müsste dann $\deg p = \deg q_1 + \deg q_2$ und somit $\deg q_1 = 1$ oder $\deg q_2 = 1$.
  Also besäße $p$ einen Teiler vom Grad $1$;
  dieser wäre bis auf Normierung ein Linearfaktor, weshalb $p$ ein Nullstelle hätte.
\end{solution}


\begin{question}
  Es seien $p, q \in K[T]$ zwei normierte irreduzible Polynome mit $p \neq q$.
  Zeigen Sie, dass $p$ und $q$ in $\overline{K}$ keine gemeinsamen Nullstellen haben.
\end{question}


\begin{solution}
  Gebe es eine gemeinsame Nullstelle $\alpha \in \overline{K}$ von $p$ und $q$, so wären $p$ und $q$ beide das Minimalpolynom von $\alpha$ über $K$, und somit $p = q$.
\end{solution}


\begin{question}
  Es sei $K(\alpha)/K$ eine endliche, zyklische Körpererweiterung von ungeraden Grad.
  Zeigen Sie, dass $K(\alpha) = K(\alpha^2)$.
\end{question}


\begin{solution}
  Da $K(\alpha^2) \subseteq K(\alpha)$ gilt, genügt es zu zeigen, dass $\alpha^2 \in K(\alpha)$.
  Wir nehmen an, dass $\alpha^2 \notin K(\alpha)$.
  Dann ist das normierte quadratische Polynom $P(T) \coloneqq T^2 - \alpha^2 \in K(\alpha^2)[T]$ irreduzibel mit $P(\alpha) = 0$, und deshalb das Minimalpolynom von $\alpha$ über $K(\alpha^2)$.
  Es ist also $[K(\alpha) : K(\alpha^2)] = 2$.
  Damit gilt
  \[
      [K(\alpha) : K]
    = [K(\alpha) : K(\alpha^2)] [K(\alpha^2) : K]
    = 2 [K(\alpha^2) : K],
  \]
  was im Widerspruch dazu steht, dass $[K(\alpha) : K]$ ungerade ist.
\end{solution}


\begin{question}
  \label{question: algebraically closed fields have no nontrivial algebraic extensions}
  Es sei $K$ ein algebraisch abgeschlossener Körper und $L/K$ eine algebraische Körpererweiterung.
  Zeigen Sie, dass bereits $L = K$ gilt.
\end{question}


\begin{solution}
  Es sei $\alpha \in L$.
  Da $L/K$ algebraisch ist, gibt es ein normiertes Polynom $P \in K[T]$ mit $P \neq 0$ und $P(\alpha) = 0$.
  Da $K$ algebraisch abgeschlossen ist zerfällt $P$ in Linearfaktoren, also $P(T) = (T - a_1) \dotsm (T - a_n)$ mit $a_1, \dotsc, a_n \in K$ und $n = \deg P$.
  Da
  \[
      0
    = P(\alpha)
    = (\alpha - a_1) \dotsm (\alpha - a_n)
  \]
  muss bereits $\alpha = a_i$ für ein $1 \leq i \leq n$, und somit $\alpha \in K$.
\end{solution}


\begin{question}
  \label{question: transitivity of algebraic closure}
  Es sei $L/K$ eine Körpererweiterung und $\overline{L}/L$ ein algebraischer Abschluss von $L$.
  \begin{enumerate}
    \item
      Zeigen Sie, dass $\overline{L}/K$ genau dann ein algebraischer Abschluss ist, wenn $L/K$ algebraisch ist.
    \item
      Zeigen Sie, dass es einen Unterkörper $\overline{K} \subseteq L$ gibt, so dass $\overline{K} \supseteq K$ gilt und $\overline{K}/K$ ein algebraischer Abschluss ist.
    \item
      Entscheiden Sie, ob der obige Körper $\overline{K}$ eindeutig ist.
  \end{enumerate}
\end{question}


\begin{solution}
  \begin{enumerate}
    \item
      Wegen der algebraischen Abgeschlossenheit von $\overline{L}$ ist $\overline{L}/K$ genau dann ein algebraischer Abschluss, wenn $\overline{L}/K$ algebraisch ist.
      Da $\overline{L}/L$ algebraisch ist folgt aus der Transitivität von Algebraizität, dass dies genau dann gilt, wenn $L/K$ algebraisch ist.
      
    \item
      Es sei $\overline{K} \coloneqq \{x \in \overline{L} \mid \text{$x$ is algebraisch über $K$}\}$.
      Dann ist $K \subseteq \overline{K} \subseteq \overline{L}$ ein Zwischenkörper, so dass $\overline{K}/K$ algebraisch ist.
      Jedes nicht-konstante Polynom $f \in \overline{K}[X] \subseteq \overline{L}[X]$ hat eine Nullstelle $x \in \overline{L}$, da $\overline{L}$ algebraisch abgeschlossen ist.
      Dann ist $x$ algebraisch über $\overline{K}$, und wegen der Transitivität von Algebraizität damit auch algebraisch über $K$;
      somit gilt bereits $x \in \overline{K}$.
      Das zeigt, dass $\overline{K}$ algebraisch abgeschlossen ist.
      
    \item
      Es sei $K \subseteq \overline{K}' \subseteq \overline{L}$ ein weiterer Zwischenkörper, so dass $\overline{K}'/K$ ein algebraischer Abschluss ist.
      Dann ist $\overline{K}'/K$ algebraisch und somit $\overline{K}' \subseteq \overline{K}$.
      Wir haben also einen Erweiterungsturm $\overline{K}/\overline{K}'/K$.
      Da $\overline{K}/K$ algebraisch ist, ist es auch $\overline{K}/\overline{K}'$;
      da $\overline{K}'$ algebraisch abgeschlossen ist, gilt deshalb bereits $\overline{K} = \overline{K}'$ (siehe Übung~\ref{question: algebraically closed fields have no nontrivial algebraic extensions}).
  \end{enumerate}
\end{solution}


\begin{question}
  \label{equation: finite extensions are algebraic}
  Zeigen Sie, dass endliche Körpererweiterungen algebraisch sind.
\end{question}


\begin{solution}
  Es sei $L/K$ eine endliche Körpererweiterung und $x \in L$.
  Für den $K$-Untervektorraum $\generate{ \{x^n \mid n \in \Natural\} }_K \subseteq L$ gilt
  \[
          \dim_K \generate{ \{x^n \mid n \in \Natural \} }_K
    \leq  \dim_K L
    =     [L : K]
    <     \infty,
  \]
  weshalb die Potenzen $x^n$ mit $n \in \Natural$ linear abhängig über $K$ sind.
  Also gibt es eine nichttriviale Linearkombination
  \[
    a_n x^n + \dotsb + a_1 x + a_0 = 0
  \]
  mit $n \geq 1$ und $a_n, \dotsc, a_0 \in K$ mit $a_n \neq 0$.
  Für das Polynom
  \[
              P(T)
    \coloneqq a_n T^n + \dotsb + a_1 T + a_0
    \in       K[T]
  \]
  gilt also $P(x) = 0$, weshalb $x$ algebraisch über $K$ ist.
\end{solution}


\begin{question}
  Es seien $M / L / K$ Körpererweiterungen, so dass $M/L$ und $L/K$ algebraisch sind.
  Zeigen Sie, dass auch $M/K$ algebraisch ist.
\end{question}


\begin{question}
  Es sei $x \in M$.
  Da $M/L$ algebraisch ist gibt es ein Polynom $p(T) \in L(T)$ mit $p \neq 0$ und $p(x) = 0$.
  Es seien $a_0, \dotsc, a_n \in L$ die Koeffizienten von $p$.
  Da $L/K$ algebraisch ist sind $a_0, \dotsc, a_n$ algebraisch über $K$.
  Für $L' \coloneqq K(a_0, \dotsc, a_n)$ wird die Erweiterung $L'/K$ von endlich vielen algebraischen Elementen erzeugt und ist deshalb endlich.
  Es gilt bereits $p(T) \in L'(T)$, weshalb $x$ algebraisch über $L'$ ist.
  Insbesondere ist deshalb $L'(x)/L'$ endlich.
  Ingesamt gilt also $[K(x) : K] \leq [L'(x) : K] = [L'(x) : L'][L' : K] < \infty$.
  Die Erweiterung $K(x)/K$ ist also endlich, und damit algebraisch (siehe Übung~\ref{equation: finite extensions are algebraic}).
  Inbesondere ist $x$ algebraisch über $K$.
\end{question}


\begin{question}
  Es sei $L/K$ eine Körpererweiterung und es seien $\alpha, \beta \in L$.
  Zeigen Sie, dass $\alpha$ und $\beta$ genau dann beide algebraisch über $K$ sind, wenn $\alpha + \beta$ und $\alpha \beta$ beide algebraisch über $K$ sind.
\end{question}


\begin{remark*}
  Da $\pi$ und $e$ transzenent (über $\Rational$) sind, muss von den beiden Zahlen $\pi + e$ und $\pi e$ mindestens eine transzendent sein.
  Es ist nicht bekannt, welches von ihnen es ist.
\end{remark*}


\begin{solution}
  Sind $\alpha$ und $\beta$ algebraisch über $K$, so ist $K(\alpha, \beta) / K$ eine algebraische Körpererweiterung.
  Da $\alpha + \beta, \alpha \beta \in K(\alpha, \beta)$ sind $\alpha + \beta$ und $\alpha \beta$ dann algebraisch über $K$.
  
  Es seien nun $\alpha + \beta$ und $\alpha \beta$ algebraisch über $K$.
  Dann ist $K(\alpha + \beta, \alpha \beta)/K$ eine algebraische Erweiterung.
  Auch die Erweiterung $K(\alpha, \beta)/K(\alpha + \beta, \alpha \beta)$ ist algebraisch, da $\alpha$ und $\beta$ Nullstellen des Polynoms
  \[
              P(T)
    \coloneqq (T - \alpha)(T - \beta)
    =         T^2 - (\alpha + \beta)T + \alpha \beta
    \in       K(\alpha + \beta, \alpha \beta)[T]
  \]
  sind.
  Wegen der Transitivität von Algebraizität folgt, dass auch $K(\alpha,\beta)/K$ algebraisch ist, also $\alpha$ und $\beta$ algebraisch über $K$ sind.
\end{solution}


\begin{question}
  Es sei $L/K$ eine Körpererweiterung, so dass $p \coloneqq [L : K]$ endlich und prim ist.
  Zeigen Sie, dass $L/K$ ein zyklische Erweiterung ist, und bestimmen Sie alle $\alpha \in L$ mit $L = K(\alpha)$.
\end{question}


\begin{solution}
  Für alle $\alpha \in K$ ist $K(\alpha) = K$.
  Ist $\alpha \in L$ mit $\alpha \notin K$, so ist $K(\alpha)/K$ eine echte Körperweiterung, weshalb $[K(\alpha) : K] \neq 1$ gilt.
  Aus
  \[
      p
    = [L : K]
    = [L : K(\alpha)] \underbrace{[K(\alpha) : K]}_{\neq 1}
  \]
  folgt, da $p$ prim ist, dass $[L : K(\alpha)] = 1$ (und $[K(\alpha) : K] = p$), und somit $K(\alpha) = L$.
  Also ist $L$ eine zyklische Körpererweiterung, und die möglichen Elemente sind genau die $\alpha \in L$, für die $\alpha \notin K$.
\end{solution}


\begin{question}
  Es sei $L/K$ eine endliche Körpererweiterung mit $[L : K] = 2^k$ für ein $k \geq 0$.
  Es sei $P \in K[T]$ ein kubisches Polynom, das eine Nullstelle in $L$ hat.
  Zeigen Sie, dass $f$ bereits eine Nullstelle in $K$ hat.
\end{question}


\begin{solution}
  Wir können o.B.d.A.\ davon ausgehen, dass $P$ normiert ist.
  Es sei $\alpha \in L$ eine Nullstelle von $P$.
  Hätte $P$ keine Nullstelle in $K$, so wäre $P$ irreduzibel in $K[T]$, da $P$ kubisch ist.
  Damit wäre dann $P$ das Minimalpolynom von $\alpha$ über $K$, und somit $[K(\alpha) : K] = \deg P = 3$.
  Dann wäre aber
  \[
          3
    =     [K(\alpha) : K]
    \mid  [L : K(\alpha)] [K(\alpha) : K]
    =     [L : K]
    =     2^k,
  \]
  was nicht gilt.
\end{solution}


\begin{question}
  Es sei $K$ ein Körper und $f \in K[T]$ ein irreduzibles Polynom.
  \begin{enumerate}
    \item
      Zeigen Sie, dass $f$ im Fall $\ringchar K = 0$ separabel ist.
    \item
      Zeigen Sie durch Angabe eines Beispiels, dass $f$ im Fall $\ringchar K > 0$ nicht notwendigerweise separabel ist.
  \end{enumerate}
\end{question}


\begin{enumerate}
  \item
    Wegen der Irreduziblität von $f$ gilt $\deg f \geq 1$.
    Wegen $\ringchar K = 0$ folgt, dass $f' \neq 0$.
    Da aber $\deg f' = \deg f - 1 < \deg f$ gilt, folgt aus der Irreduziblität von $f$, dass $f$ und $f'$ teilerfremd sind.
    Also ist $f$ separabel.
  \item
    Ist $p \coloneqq \ringchar K > 0$, so ist das Polynom $f(X) \coloneqq X^p - t \in \Field_p(t)[X]$ nach Eisenstein irreduzibel.
    Es gilt aber $f' = 0$, weshalb $f$ und $f'$ nicht teilerfremd, und $f$ somit nicht separabel ist.
\end{enumerate}


\begin{question}
  Es sei $K$ ein Körper, $L/K$ eine endliche Körpererweiterung und es seien $K \subseteq L_1, L_2 \subseteq L$ zwei Zwischenkörper.
  \begin{enumerate}
    \item
      Zeigen Sie für alle $\alpha, \beta \in L$, dass $[K(\alpha, \beta) : K(\alpha)] \leq [K(\beta) : K]$ gilt.
    \item
      Es sei $L_1 L_2 \subseteq L$ der kleinste Unterkörper, der $L_1$ und $L_2$ enthält.
      Zeigen Sie, dass $[L_1 L_2 : L_2] \leq [L_1 : K]$.
    \item
      Zeigen Sie, dass $[L_1 L_2 : K] = [L_1 : K] [L_2 : K]$ falls $[L_1 : K]$ und $[L_2 : K]$ teilerfremd sind.
  \end{enumerate}
\end{question}


\begin{solution}
  \begin{enumerate}
    \item
      Da $L/K$ endlich ist, gilt dies auch für $[K(\beta) : K]$.
      Ist $f(X) \in K[X]$ das Minimalpolynom von $\beta$, so gilt $[K(\beta) : K] = \deg f$.
      Dann ist auch $f(X) \in K(\alpha)[X]$ mit $f(\beta) = 0$;
      für das Minimalpolynom $g(X) \in K(\alpha)[X]$ von $\beta$ gilt daher $\deg g \leq \deg f$.
      Dabei gilt $\deg g = [K(\alpha, \beta) : K(\alpha)]$ und die Aussage folgt.
    
    \item
      Da $L/K$ endlich erzeugt ist, sind es auch $L_1/K$ und $L_2/K$.
      Inbesondere sind die beiden Erweiterungen sind endlich erzeugt, also $L_1 = K(\alpha_1, \dotsc, \alpha_n)$ und $L_2 = K(\beta_1, \dotsc, \beta_m)$.
      Inbesondere gilt damit auch $L_1 L_2 = K(\alpha_1, \dotsc, \alpha_n, \beta_1, \dotsc, \beta_m)$.
      Aus dem vorherigen Aussagenteil erhalten wir, dass
      \begin{align*}
              [L_1 : K]
       &=     [K(\alpha_1, \dotsc, \alpha_n) : K]
       \\
       &\geq  [K(\alpha_1, \dotsc, \alpha_n, \beta_1) : K(\beta_1)]
       \\
       &\geq  [K(\alpha_1, \dotsc, \alpha_n, \beta_1, \beta_2) : K(\beta_1, \beta_2)]
       \\
       &\geq  \dotsb
       \\
       &\geq  [K(\alpha_1, \dotsc, \alpha_n, \beta_1, \dotsc, \beta_m) : K(\beta_1, \dotsc, \beta_m)]
        =  [L_1 L_2 : L_2].
      \end{align*}
    
    \item
      Aus $K \subseteq L_1, L_2 \subseteq L_1 L_2$ folgt wegen der Multiplikativität des Grades, dass $[L_1 : K]$ und $[L_2 : K]$ den Grad $[L_1 L_2 : K]$ teilen.
      Wegen der Teilerfremdheit folgt, dass auch $[L_1 : K] [L_2 : K]$ ein Teiler von $[L_1 L_2 : K]$ ist.
      Andererseits gilt nach dem vorherigen Aussagenteil, dass
      \[
              [L_1 L_2 : K]
        =     [L_1 L_2 : L_2] [L_2 : K]
        \leq  [L_1 : K] [L_2 : K].
      \]
      Ingesamt folgt deshalb $[L_1 L_2 : K] = [L_1 : K][L_2 : K]$.
  \end{enumerate}
\end{solution}


\begin{question}[subtitle = Quadratische Körpererweiterungen]
  \label{question: quadratic field extensions}
  Es sei $L/K$ eine Körpererweiterung vom Grad $2$.
  Es gelte zunächst $\ringchar K \neq 2$.
  \begin{enumerate}
    \item
      Zeigen Sie, dass $L = K(\alpha)$ für ein $\alpha \in L$ mit $\alpha \notin K$ und $\alpha^2 \in K$ gilt.
      ($L$ entsteht also durch Hinzuadjungieren einer Quadratwurzel.)
    \item
      Folgern Sie, dass $L/K$ galoisch ist.
  \end{enumerate}
  Es gelte nun $\ringchar K = 2$.
  \begin{enumerate}[resume]
    \item
      Zeigen Sie, dass $L$ nicht notwendigerweise durch Hinzuadjungieren einer Quadratwurzel entsteht.
    \item
      Zeigen Sie, dass $L/K$ nicht notwendigerweise galoisch ist.
  \end{enumerate}
\end{question}


\begin{solution}
  \begin{enumerate}
    \item
      Ist $\alpha \in L$ mit $\alpha \neq K$, so gilt
      \[
              2
        =     [L : K)
        \geq  [K(\alpha) : K]
        \geq  2,
      \]
      also $[K(\alpha) : K] = 2$ und somit $L = K(\alpha)$.
      Ist $f \in K[X]$ das Minimalpolynom von $\alpha$, so gilt deshalb $\deg f = [K(\alpha : K] = 2$ und $L = K(\alpha) \cong K[X](f)$.
      Wir können daher o.B.d.A.\ davon ausgehen, dass $L = K[X]/(f)$ für ein normiertes, irreduzibles quadratisches Polyom $f \in K[X]$ gilt.
      
      Da $\ringchar K \neq 2$ gilt, können wir
      \[
          f(X)
        = X^2 + aX + b
        = \left( X + \frac{a}{2} \right)^2 - \left( \frac{a^2}{4} - b\right)
      \]
      schreiben.
      Der Automorphismus $K[X] \to K[X]$, $p(X) \mapsto p(X - a/2)$ induziert deshalb einen Isomorphismus $K[X]/(f) \to K[X]/(X^2 - (a^2/4 - b))$.
      Wir können also auch o.B.d.A.\ davon ausgehen, dass $f$ von der Form $f(X) = X^2 - c$ ist, wobei $c \in K$ kein Quadrat ist.
      
      Wir haben nun also $L = K[X]/(X^2 - c)$, wobei $c \in K$ kein Quadrat ist.
      Nun leistet $\alpha = \overline{X}$ das Gewünschte.
    
    \item
      Es sei $\alpha \in L$ mit $\alpha \notin K$ und $\alpha^2 \in K$.
      Dann ist $f(X) \coloneqq X^2 - \alpha^2 \in K[X]$ das Minimalpolynom, denn es ist normiert, hat $\alpha$ als Nullstelle, und ist irreduzibel (denn es ist quadratisch und hat keine Nullstelle in $K$).
      Es gibt nun (mindestens) zwei mögliche Argumentationsweisen:
      \begin{itemize}
        \item
          Der Automorphismus $K[X] \to K[X]$, $p(X) \mapsto p(-X)$ induziert einen $K$-linearen Automorphismus $K[X]/(f) \to K[X]/(f)$, $a + b \overline{X} \mapsto a - b \overline{X}$.
          Unter dem Isomorphismus $K[X]/(f) \to L = K(\alpha)$, $a + b \overline{X} \mapsto a + b \alpha$ entspricht dies dem $K$-linearen Automorphismus $\tau \colon L \to L$, $a + b \alpha \mapsto a - b \alpha$.
          Es ergibt sich nun (mindestens) zwei Weisen um einzusehen, dass $|{\Gal(L/K)}| = 2 = [L : K]$ gilt:
          \begin{itemize}
            \item
              Jedes $\sigma \in \Gal(L/K)$ muss die Nullstellen von $f$ permutieren, und somit muss $\sigma(\alpha) = \pm \alpha$ gelten.
              Da $L = K(\alpha)$ gilt, ist deshalb bereits $\sigma = \id$ (falls $\sigma(\alpha) = \alpha$) oder $\sigma = \tau$ (falls $\sigma(\alpha) = - \alpha$).
              Es gilt also bereits $\Gal(L/K) = \{\id_L, \tau\}$.
            \item
              Es gilt $2 \leq |{\Gal(L/K)}| \leq [L : K] = 2$.
          \end{itemize}
        \item
          Da $L = K(\alpha) = K(\alpha, -\alpha)$ gilt, ist $L$ ein Zerfällungskörper von $f$ über $K$.
          Die beiden Nullstellen von $f$, $\alpha$ und $-\alpha$, sind verschieden, da $\alpha \neq 0$ (denn $\alpha \notin K$) und $\ringchar K \neq 2$ gelten.
          Also ist $f$ separabel.
          Als Zerfällungskörper eines separablen Polynoms ist $L/K$ galoisch.
      \end{itemize}
      
    \item
      Wir betrachten den Fall, dass $K$ ein endlicher Körper ist.
      Dann ist der Frobeniushomomorphismus $\sigma \colon K \to K$, $x \mapsto x^2$ ein Automorphismus, also insbesondere surjektiv.
      Ist $\alpha \in L$ mit $\alpha^2 \in K$, so gibt es deshalb ein $x \in K$ mit $x^2 = \alpha^2$, also $0 = x^2 - \alpha^2 = (x - \alpha)^2$.
      Somit gilt dann bereits $\alpha = x \in K$.
      
    \item
      Wir betrachten den Fall $K = \Field_2(t)$.
      Das Polynom $f(X) = X^2 - t \in \Field_2(t)[X]$ ist nach Eisenstein bezüglich des Primelements $t \in \Field_2[t]$ irreduzibel.
      Es gilt aber $f' = 0$, weshalb $f$ nicht separabel ist.
      Für $L \coloneqq K[X]/(f)$ ist dann $\overline{X} \in L$ nicht separabel, da $f$ das Minimalpolynom von $\overline{X}$ ist.
      Somit ist $L/K$ auch nicht galoisch.
  \end{enumerate}
\end{solution}


\begin{question}[subtitle = Einschränkung und Transitivität von Normalität]
  \label{question: restriction and transitivity of normality}
  Es seien $M/L/K$ algebraische Körpererweiterungen.
  \begin{enumerate}
    \item
      Es sei $\overline{K}$ ein algebraischer Abschluss von $K$.
      Zeigen Sie, dass $\overline{K}/K$ normal ist.
    \item
      Zeigen Sie, dass $M/L$ normal ist, falls $M/K$ normal ist.
    \item
      Zeigen Sie, dass $L/M$ nicht notwendigerweise normal ist, falls $M/K$ normal ist.
    \item
      Zeigen Sie, dass $M/K$ nicht notwendigerweise normal ist, wenn $M/L$ und $L/K$ normal sind.
  \end{enumerate}
\end{question}


\begin{solution}
  \begin{enumerate}
    \item
      Jedes Polynom $f \in K[X]$ zerfällt über $\overline{K}$ bereits in Linearfaktoren;
      inbesondere also jedes irreduzibel Polynom, das eine Nullstelle in $\overline{K}$ hat.
      
    \item
      Nach Annahme gibt es eine Familie $(f_i)_{i \in I}$ von Polynomen $f_i \in K[X]$, so dass $M$ der Zerfällungskörper der $f_i$ über $K$ ist.
      Dann ist $M$ auch der Zerfällungskörper der $f_i$ über $L$, und somit $M/L$ normal.
      
    \item
      Es sei $L/K$ eine algebraische, nicht normale Körpererweiterung, und $\overline{L}$ ein algebraischer Abschluss von $L$.
      Dann ist $\overline{L}$ algebraisch abgeschlossen und $\overline{L}/K$ algebraisch, also $\overline{L}$ auch von $K$ ein algebraischer Abschluss.
      Nach dem ersten Aussagenteil ist somit $\overline{L}/L/K$ ein Gegenbeispiel.
      
    \item
      Wir betrachten die Körpererweiterungen $\Rational(\sqrt[4]{2}) / \Rational(\sqrt{2}) / \Rational$.
      
      Es gilt $[\Rational(\sqrt[4]{2}) : \Rational] = 4$, denn das Minimalpolynom von $\sqrt[4]{2}$ über $\Rational$ ist $X^4 - 2$ (die Irreduziblität ergibt sich mit Eisenstein), und es gilt $[\Rational(\sqrt[4]{2}) : \Rational] = 2$, denn das Minimalpolynom von $\sqrt{2}$ über $\Rational$ ist $X^2 - 2$ (die Irreduziblität ergibt sich ebenfalls mit Eisenstein).
      Somit gilt $[\Rational(\sqrt[4]{2}) : \Rational(\sqrt{2})] = 2$.
      
      Die beiden Erweiterungen $\Rational(\sqrt[4]{2}) / \Rational(\sqrt{2})$ und $\Rational(\sqrt{2}) / \Rational$ sind also beide vom Grad $2$;
      da $\ringchar \Rational \neq 2$ gilt, sind sie somit beide galoisch (siehe Übung~\ref{question: quadratic field extensions}) und damit insbesondere normal.
      Die Erweiterung $\Rational(\sqrt[4]{2}) / \Rational$ ist allerdings nicht normal, denn das Polynom $f(X) \coloneqq X^4 - 2 \in \Rational[X]$ hat zwar eine Nullstelle in $\Rational(\sqrt[4]{2})$, zerfällt dort aber nicht in Linearfaktoren, da die beiden Nullstellen $\pm i \sqrt[4]{2}$ nicht in $\Rational(\sqrt[4]{2})$ liegen (denn $\Rational(\sqrt[4]{2}) \subseteq \Real$).
  \end{enumerate}
\end{solution}


\begin{question}
  Zeigen Sie, dass eine Körpererweiterung $L/K$ genau dann algebraisch ist, wenn jeder Zwischenring $K \subseteq R \subseteq L$ bereits ein Körper ist.
\end{question}


\begin{solution}
  Es sei $L/K$ algebraisch und $K \subseteq R \subseteq L$ ein Zwischenring.
  Für $\alpha \in R$ ist dann $\alpha$ algebraisch über $K$, und somit $K(\alpha) = K[\alpha]$.
  Da $R$ ein Ring ist, der $\alpha$ und $R$ enthält, gilt $K[\alpha] \subseteq R$.
  Somit ist $K(\alpha) = K[\alpha] \subseteq R$.
  Ist $\alpha \neq 0$, so ist inbesondere $\alpha^{-1} \in K(\alpha) \subseteq R$.
  Das zeigt, dass jedes Element $\alpha \in R$ mit $\alpha \neq 0$ in $R$ invertierbar ist.
  Somit ist $R$ ein Körper.
  (Die Kommutativität von $R$ ist klar, es sich um einen Unterring von $L$ handelt, und $L$ als Körper kommutativ ist.)
  
  Es sei nun $L/K$ nicht algebraisch.
  Dann gibt es ein Element $\alpha \in L$, das transzendent über $K$ ist.
  Der Zwischenring $K \subseteq K[\alpha] \subseteq L$ ist dann kein Körper:
  Für den Polynomring $K[T]$ ist der Einsetzhomorphismus $K[T] \to K[\alpha]$, $P(T) \to P(\alpha)$ surjektiv, und wegen der Transzendenz von $\alpha$ auch injektiv, und somit ein Isomorphismus.
  Der Polynomring $K[T]$, und somit auch $K[\alpha]$, ist aber kein Körper.
\end{solution}


\begin{question}
  Es seien $L/E/K$ endliche Körpererweiterungen, so dass $L/K$ galoisch ist.
  Zeigen Sie, dass $E/K$ genau dann galoisch ist, wenn $\sigma(E) = E$ für alle $\sigma \in \Gal(L/K)$ gilt.
\end{question}


\begin{solution}
  Wir geben zwei mögliche Lösungen an:
  \begin{itemize}
    \item
      Nach dem Hauptsatz der Galoistheorie ist $E/K$ genau dann Galoisch, wenn die Untergruppe $\Gal(L/E) \subseteq \Gal(L/K)$ normal ist.
      
      Es gelte zunächst $\sigma(E) = E$ für alle $\sigma \in \Gal(L/K)$, und es seien $\sigma \in \Gal(L/K)$ und $\tau \in \Gal(L/E)$.
      Für jedes $e \in E$ gilt nach Annahme $\sigma^{-1}(e) \in E$, also $\tau(\sigma^{-1}(e)) = \sigma^{-1}(e)$ und somit $\sigma(\tau(\sigma^{-1}(e))) = \sigma(\sigma^{-1}(e)) = e$.
      Also gilt $\sigma \tau \sigma^{-1} \in \Gal(L/E)$.
      Das zeigt, dass $\Gal(L/E)$ normal in $\Gal(L/E)$ ist.
      
      Es sei nun $\Gal(L/E)$ normal in $\Gal(L/K)$, und es sei $\sigma \in \Gal(L/K)$.
      Nach Annahme gilt $\Gal(L/E) \sigma = \sigma \Gal(L/E)$, weshalb es für jedes $\tau \in \Gal(L/E)$ ein $\tau' \in \Gal(L/E)$ mit $\tau \sigma = \sigma \tau'$ gibt.
      Für jedes $e \in E$ gilt deshalb, dass
      \[
        \tau(\sigma(e)) = \sigma(\tau'(e)) = \sigma(e)
        \qquad
        \text{für alle $\tau \in \Gal(L/E)$}
      \]
      gilt.
      Also gilt $\sigma(e) \in L^{\Gal(L/E)}$ für alle $e \in E$;
      nach dem Hauptsatz der Galoistheorie gilt $L^{\Gal(L/E)} = E$, weshalb $\sigma(E) \subseteq E$.
      
      Für alle $\sigma \in \Gal(L/K)$ gilt also $\sigma(E) \subseteq E$.
      Da dies auch für $\sigma^{-1}$ gilt, erhalten wir mit $\sigma^{-1}(E) \subseteq E$, dass $E = \sigma(\sigma^{-1}(E)) \subseteq \sigma(E) \subseteq E$ und somit $E = \sigma(E)$ gilt.
      
    \item
      Die Erweiterung $L/K$ ist separabel und normal, da sie Galoisch ist.
      Damit ist auch $L/E$ separabel;
      es bleibt also nur zu zeigen, dass $L/E$ genau dann normal ist, wenn $\sigma(E) = E$ für alle $\sigma \in \Gal(L/K)$ gilt.
      
      Da $L/K$ algebraisch ist gibt es einen algebraischen Abschluss $\overline{K}$ von $K$, der $L$ enthält.
      
      Es sei zunächst $E/K$ normal und $\sigma \in \Gal(L/K)$.
      Es seien $\iota_E \colon E \to \overline{K}$, $x \mapsto x$ und $\iota_L \colon L \to \overline{K}$, $x \mapsto x$ die kanonische Inklusionen.
      Es sei außerdem $\iota_{LE} \colon E \to L$, $x \mapsto x$ die kanonische Inklusion;
      es gilt $\iota_E = \iota_L \iota_{LE}$.
      
      Es sind $\iota_E \colon E \to \overline{K}$ und $\iota_L \circ \sigma \circ \iota_{LE} \colon E \to \overline{K}$ zwei $K$-lineare Körperhomomorphismen.
      Da $E/K$ normal ist haben sie das gleiche Bild; es gilt also
      \[
          E
        = \iota_E(E)
        = \iota_L(\sigma(\iota_{LE}(E)))
        = \sigma(E).
      \]
      
      Es gelte nun $\sigma(E) = E$ für alle $\sigma \in \Gal(L/K)$.
      Es sei $\varphi \colon E \to \overline{K}$ ein $K$-linearer Körperhomomorphismus.
      Da $L/E$ algebraisch ist, lässt sich $\varphi$ zu einem Körperhomomorphismus $\psi \colon L \to \overline{K}$ fortsetzen;
      da $\psi|_K = \varphi|_K = \id_K$ gilt, ist auch $\psi$ $K$-linear.
      Da $L/K$ normal ist und $\overline{K}$ ein algebraischer Abschluss von $K$ ist, gilt $\psi(L) = L$ (denn auch $\iota \colon L \to \overline{K}$, $x \mapsto x$ ist eine $K$-lineare Einbettung, und wegen der Normalität von $L/K$ gilt bereits $\im \psi = \im \iota = L$).
      Also schränkt sich $\psi$ zu einem $K$-linearen Automorphismus $\sigma \colon L \to L$, $x \mapsto \psi(x)$ ein.
      Nach Annahme gilt somit
      \[
          E
        = \sigma(E)
        = \psi(E)
        = \varphi(E)
        = \im \varphi.
      \]
      Das Bild von $\varphi$ hängt also nicht von $\varphi$ selbst ab;
      dies zeigt die Normalität von $E/K$.
  \end{itemize}
\end{solution}


