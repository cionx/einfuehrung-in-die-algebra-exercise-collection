\section{Körpertheorie}


\begin{question}\label{question: characterization of fields via its ideals}
  Zeigen Sie, dass für einen kommutativen Ring $K$ die folgenden Bedingungen äquivalent sind:
  \begin{enumerate}
    \item
      \label{enum: K is a field}
      $K$ ist ein Körper.
    \item
      \label{enum: K has exactly two ideals}
      $K$ hat genau zwei Ideale.
    \item
      \label{enum: The zero ideal is maximal}
      Das Nullideal in $K$ ist maximal.
  \end{enumerate}
\end{question}


\begin{solution}
  (\ref{enum: K is a field} $\implies$ \ref{enum: K has exactly two ideals})
  Da $K$ ein Körper ist gilt $0 \neq K$, also hat $K$ mindestens zwei Ideale.
  Ist $I \subseteq K$ ein Ideal mit $I \neq 0$, so gibt es ein $x \in I$ mit $x \neq 0$.
  Dann ist $x$ eine Einheit in $K$, somit $K = (x) \subseteq I$ und deshalb $I = K$.
  Also sind $0$ und $K$ die einzigen Ideale in $K$.
  
  (\ref{enum: K has exactly two ideals} $\implies$ \ref{enum: The zero ideal is maximal})
  Es muss $0 \neq K$, denn sonst wäre $0$ das einzige Ideal in $K$.
  Also sind $0$ und $K$ die einzigen beiden Ideale in $K$.
  Ist $I \subseteq K$ ein Ideal mit $0 \subsetneq I$, so muss bereits $I = K$.
  Also ist $0$ ein maximales Ideal.
  
  (\ref{enum: The zero ideal is maximal} $\implies$ \ref{enum: K is a field})
  Da $0 \subseteq K$ maximal ist, ergibt sich, dass $K \cong K/0$ ein Körper ist.
\end{solution}


\begin{question}
  Es sei $K$ ein algebraisch abgeschlossener Körper.
  Zeigen Sie, dass $K$ unendlich ist.
\end{question}


\begin{solution}
  Wäre $K$ endlich, so wäre
  \[
              p(T)
    \coloneqq 1 + \prod_{\lambda \in K} (T - \lambda)
    \in       K[T]
  \]
  ein Polynom positiven Grades ohne Nullstellen (denn $p(x) = 1$ für alle $x \in K$).
  Dies stünde im Widerspruch zur algebraischen Abgeschlossenheit von $K$.
\end{solution}


\begin{question}
  Es sei $K$ ein Körper und $p \in K[T]$ ein Polynom mit $\deg p \in \{2, 3\}$.
  Zeigen Sie, dass $p$ genau dann irreduzibel ist, wenn $p$ keine Nullstelle hat.
\end{question}


% TODO: Adding a solution.


\begin{question}
  Es seien $p, q \in K[T]$ zwei normierte irreduzible Polynome mit $p \neq q$.
  Zeigen Sie, dass $p$ und $q$ in $\overline{K}$ keine gemeinsamen Nullstellen haben.
\end{question}


\begin{solution}
  Gebe es eine gemeinsame Nullstelle $\alpha \in \overline{K}$ von $p$ und $q$, so wären $p$ und $q$ beide das Minimalpolynom von $\alpha$ über $K$, und somit $p = q$.
\end{solution}


\begin{question}
  Es sei $K(\alpha)/K$ eine endliche, zyklische Körpererweiterung von ungeraden Grad.
  Zeigen Sie, dass $K(\alpha) = K(\alpha^2)$.
\end{question}


\begin{solution}
  Da $K(\alpha^2) \subseteq K(\alpha)$ gilt, genügt es zu zeigen, dass $\alpha^2 \in K(\alpha)$.
  Wir nehmen an, dass $\alpha^2 \notin K(\alpha)$.
  Dann ist das normierte quadratische Polynom $P(T) \coloneqq T^2 - \alpha^2 \in K(\alpha^2)[T]$ irreduzibel mit $P(\alpha) = 0$, und deshalb das Minimalpolynom von $\alpha$ über $K(\alpha^2)$.
  Es ist also $[K(\alpha) : K(\alpha^2)] = 2$.
  Damit gilt
  \[
      [K(\alpha) : K]
    = [K(\alpha) : K(\alpha^2)] [K(\alpha^2) : K]
    = 2 [K(\alpha^2) : K],
  \]
  was im Widerspruch dazu steht, dass $[K(\alpha) : K]$ ungerade ist.
\end{solution}


\begin{question}
  Es sei $K$ ein algebraisch abgeschlossener Körper und $L/K$ eine algebraische Körpererweiterung.
  Zeigen Sie, dass bereits $L = K$ gilt.
\end{question}


\begin{solution}
  Es sei $\alpha \in L$.
  Da $L/K$ algebraisch ist, gibt es ein normiertes Polynom $P \in K[T]$ mit $P \neq 0$ und $P(\alpha) = 0$.
  Da $K$ algebraisch abgeschlossen ist zerfällt $P$ in Linearfaktoren, also $P(T) = (T - a_1) \dotsm (T - a_n)$ mit $a_1, \dotsc, a_n \in K$ und $n = \deg P$.
  Da
  \[
      0
    = P(\alpha)
    = (\alpha - a_1) \dotsm (\alpha - a_n)
  \]
  muss bereits $\alpha = a_i$ für ein $1 \leq i \leq n$, und somit $\alpha \in K$.
\end{solution}


\begin{question}
  Zeigen Sie, dass endliche Körpererweiterungen algebraisch sind.
\end{question}


\begin{solution}
  Es sei $L/K$ eine endliche Körpererweiterung und $x \in L$.
  Für den $K$-Untervektorraum $\generate{ \{x^n \mid n \in \Natural\} }_K \subseteq L$ gilt
  \[
          \dim_K \generate{ \{x^n \mid n \in \Natural \} }_K
    \leq  \dim_K L
    =     [L : K]
    <     \infty,
  \]
  weshalb die Potenzen $x^n$ mit $n \in \Natural$ linear abhängig über $K$ sind.
  Also gibt es eine nichttriviale Linearkombination
  \[
    a_n x^n + \dotsb + a_1 x + a_0 = 0
  \]
  mit $n \geq 1$ und $a_n, \dotsc, a_0 \in K$ mit $a_n \neq 0$.
  Für das Polynom
  \[
              P(T)
    \coloneqq a_n T^n + \dotsb + a_1 T + a_0
    \in       K[T]
  \]
  gilt also $P(x) = 0$, weshalb $x$ algebraisch über $K$ ist.
\end{solution}


\begin{question}
  Es seien $M / L / K$ Körpererweiterungen, so dass $M/L$ und $L/K$ algebraisch sind.
  Zeigen Sie, dass auch $M/K$ algebraisch ist.
\end{question}


% TODO: Adding a solution.


\begin{question}
  Es sei $L/K$ eine Körpererweiterung und es seien $\alpha, \beta \in L$.
  Zeigen Sie, dass $\alpha$ und $\beta$ genau dann beide algebraisch über $K$ sind, wenn $\alpha + \beta$ und $\alpha \beta$ beide algebraisch über $K$ sind.
\end{question}


\begin{remark*}
  Da $\pi$ und $e$ transzenent (über $\Rational$) sind, muss von den beiden Zahlen $\pi + e$ und $\pi e$ mindestens eine transzendent sein.
  Es ist nicht bekannt, welches von ihnen es ist.
\end{remark*}


\begin{solution}
  Sind $\alpha$ und $\beta$ algebraisch über $K$, so ist $K(\alpha, \beta) / K$ eine algebraische Körpererweiterung.
  Da $\alpha + \beta, \alpha \beta \in K(\alpha, \beta)$ sind $\alpha + \beta$ und $\alpha \beta$ dann algebraisch über $K$.
  
  Es seien nun $\alpha + \beta$ und $\alpha \beta$ algebraisch über $K$.
  Dann ist $K(\alpha + \beta, \alpha \beta)/K$ eine algebraische Erweiterung.
  Auch die Erweiterung $K(\alpha, \beta)/K(\alpha + \beta, \alpha \beta)$ ist algebraisch, da $\alpha$ und $\beta$ Nullstellen des Polynoms
  \[
              P(T)
    \coloneqq (T - \alpha)(T - \beta)
    =         T^2 - (\alpha + \beta)T + \alpha \beta
    \in       K(\alpha + \beta, \alpha \beta)[T]
  \]
  sind.
  Wegen der Transitivität von Algebraizität folgt, dass auch $K(\alpha,\beta)/K$ algebraisch ist, also $\alpha$ und $\beta$ algebraisch über $K$ sind.
\end{solution}


\begin{question}
  Es sei $L/K$ eine Körpererweiterung, so dass $p \coloneqq [L : K]$ endlich und prim ist.
  Zeigen Sie, dass $L/K$ ein zyklische Erweiterung ist, und bestimmen Sie alle $\alpha \in L$ mit $L = K(\alpha)$.
\end{question}


\begin{solution}
  Für alle $\alpha \in K$ ist $K(\alpha) = K$.
  Ist $\alpha \in L$ mit $\alpha \notin K$, so ist $K(\alpha)/K$ eine echte Körperweiterung, weshalb $[K(\alpha) : K] \neq 1$ gilt.
  Aus
  \[
      p
    = [L : K]
    = [L : K(\alpha)] \underbrace{[K(\alpha) : K]}_{\neq 1}
  \]
  folgt, da $p$ prim ist, dass $[L : K(\alpha)] = 1$ (und $[K(\alpha) : K] = p$), und somit $K(\alpha) = L$.
  Also ist $L$ eine zyklische Körpererweiterung, und die möglichen Elemente sind genau die $\alpha \in L$, für die $\alpha \notin K$.
\end{solution}


\begin{question}
  Es sei $L/K$ eine endliche Körpererweiterung mit $[L : K] = 2^k$ für ein $k \geq 0$.
  Es sei $P \in K[T]$ ein kubisches Polynom, das eine Nullstelle in $L$ hat.
  Zeigen Sie, dass $f$ bereits eine Nullstelle in $K$ hat.
\end{question}


\begin{solution}
  Wir können o.B.d.A.\ davon ausgehen, dass $P$ normiert ist.
  Es sei $\alpha \in L$ eine Nullstelle von $P$.
  Hätte $P$ keine Nullstelle in $K$, so wäre $P$ irreduzibel in $K[T]$, da $P$ kubisch ist.
  Damit wäre dann $P$ das Minimalpolynom von $\alpha$ über $K$, und somit $[K(\alpha) : K] = \deg P = 3$.
  Dann wäre aber
  \[
          3
    =     [K(\alpha) : K]
    \mid  [L : K(\alpha)] [K(\alpha) : K]
    =     [L : K]
    =     2^k,
  \]
  was nicht gilt.
\end{solution}


\begin{question}
  Es sei $K$ ein Körper und $f \in K[T]$ ein irreduzibles Polynom.
  \begin{enumerate}
    \item
      Zeigen Sie, dass $f$ im Fall $\ringchar K = 0$ separabel ist.
    \item
      Zeigen Sie durch Angabe eines Beispiels, dass $f$ im Fall $\ringchar K > 0$ nicht notwendigerweise separabel ist.
  \end{enumerate}
\end{question}


% TODO: Adding a solution.


\begin{question}
  Zeigen Sie, dass eine Körpererweiterung $L/K$ genau dann algebraisch ist, wenn jeder Zwischenring $K \subseteq R \subseteq L$ bereits ein Körper ist.
\end{question}


\begin{solution}
  Es sei $L/K$ algebraisch und $K \subseteq R \subseteq L$ ein Zwischenring.
  Für $\alpha \in R$ ist dann $\alpha$ algebraisch über $K$, und somit $K(\alpha) = K[\alpha]$.
  Da $R$ ein Ring ist, der $\alpha$ und $R$ enthält, gilt $K[\alpha] \subseteq R$.
  Somit ist $K(\alpha) = K[\alpha] \subseteq R$.
  Ist $\alpha \neq 0$, so ist inbesondere $\alpha^{-1} \in K(\alpha) \subseteq R$.
  Das zeigt, dass jedes Element $\alpha \in R$ mit $\alpha \neq 0$ in $R$ invertierbar ist.
  Somit ist $R$ ein Körper.
  (Die Kommutativität von $R$ ist klar, es sich um einen Unterring von $L$ handelt, und $L$ als Körper kommutativ ist.)
  
  Es sei nun $L/K$ nicht algebraisch.
  Dann gibt es ein Element $\alpha \in L$, das transzendent über $K$ ist.
  Der Zwischenring $K \subseteq K[\alpha] \subseteq L$ ist dann kein Körper:
  Für den Polynomring $K[T]$ ist der Einsetzhomorphismus $K[T] \to K[\alpha]$, $P(T) \to P(\alpha)$ surjektiv, und wegen der Transzendenz von $\alpha$ auch injektiv, und somit ein Isomorphismus.
  Der Polynomring $K[T]$, und somit auch $K[\alpha]$, ist aber kein Körper.
\end{solution}




