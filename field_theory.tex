\section{Körpertheorie}


% TODO:
% Calculations:
%   Galois groups
%   primitive elements in field extensions
% Theorems:


\begin{question}
  Es sei $L \coloneqq \Rational(\sqrt{2} + \sqrt{3})$.
  \begin{enumerate}
    \item
      Zeigen Sie, dass $\sqrt{2}, \sqrt{3} \in L$ und folgern Sie, dass $L = \Rational(\sqrt{2}, \sqrt{3})$.
    \item
      Bestimmen Sie den Grad der Erweiterung $L/\Rational$.
    \item
      Zeigen Sie, dass $L/\Rational$ galoisch ist.
    \item
      Bestimmen Sie $\Gal(L/\Rational)$, und entscheiden Sie, ob $\Gal(L/\Rational)$ abelsch ist.
  \end{enumerate}
\end{question}


\begin{solution}
  \begin{enumerate}
    \item
      Es gilt $(\sqrt{2} + \sqrt{3})^3 = 11 \sqrt{2} + 9 \sqrt{3}$ und deshalb
      \[
            \sqrt{2}
        =   \frac{1}{2}\left( (\sqrt{2} + \sqrt{3})^2 - 9(\sqrt{2} + \sqrt{3}) \right)
        \in L.
      \]
      Somit gilt auch $\sqrt{3} = (\sqrt{2} + \sqrt{3}) - \sqrt{3} \in L$.
      Dass $L \subseteq \Rational(\sqrt{2}, \sqrt{3})$ ist klar, und dass $\Rational(\sqrt{2}, \sqrt{3}) \subseteq L$ gilt, folgt aus $\Rational \subseteq L$ und $\sqrt{2}, \sqrt{3} \in L$.
      
    \item
      Wir betrachten die Zwischenerweiterung $\Rational \subseteq \Rational(\sqrt{2}) \subseteq L$.
      
      Das Minimalpolynom von $\sqrt{2}$ über $\Rational$ ist $f(X) = X^2 - 2 \in \Rational[X]$, denn $f$ ist normiert, nach Eisenstein irreduzibel, und hat $\sqrt{2}$ ist als Nullstelle.
      Deshalb gilt $[\Rational(\sqrt{2}) : \Rational] = 2$.
      
      Da $\sqrt{3}$ eine Nullstelle des Polynoms $g(X) = X^2 - 3 \in \Rational(\sqrt{2})[X]$ ist, gilt die Abschätzung $[L : \Rational(\sqrt{2})] \leq 2$.
      Wäre $[L : \Rational(\sqrt{2})] < 2$, also $[L : \Rational(\sqrt{2})] = 1$ und somit $L = \Rational(\sqrt{2})$, so gebe es $a, b \in \Rational$ mit $\sqrt{3} = a + b \sqrt{2}$ (denn $\{1, \sqrt{2}\}$ ist eine $\Rational$-Basis von $\Rational(\sqrt{2})$, da $[\Rational(\sqrt{2}) : \Rational] = 2$).
      Deshalb würde dann
      \[
          3
        = \sqrt{3}^2
        = (a + b \sqrt{2})^2
        = a^2 + 2 b^2 + ab \sqrt{3}
      \]
      gelten.
      Es müsste $a \neq 0$ gelten, denn sonst wäre $\sqrt{3/2} = b \in \Rational$, und es müsste auch $b \neq 0$ gelten, denn sonst wäre $\sqrt{3} = a \in \Rational$.
      Also wäre bereits $\sqrt{3} = (3 - a^2 - 2b^2)/(ab) \in \Rational$, was aber nicht gilt.
      
      Es muss also auch $[L : \Rational(\sqrt{2})] = 2$ gelten, und somit insgesamt
      \[
          [L : \Rational]
        = [L : \Rational(\sqrt{2})] [\Rational(\sqrt{2}) : \Rational]
        = 2 \cdot 2
        = 4.
      \]
      
    \item
      Es gilt $\Rational(\sqrt{2}, \sqrt{3}) = \Rational(\sqrt{2}, \sqrt{3}, -\sqrt{2}, -\sqrt{3})$, also wird $L/\Rational$ von den Nullstellen des Polynoms $f(X) = (X^2 - 2)(X^2 - 3) \in \Rational[X]$ erzeugt.
      Somit ist $L$ der Zerfällungskörper von $f$ über $\Rational$.
      Da die Nullstellen von $f$ paarweise verschieden sind, ist $f$ separabel.
      Also ist $L$ als Zerfällungskörper des separablen Polynoms $f$ bereits galoisch.
    
    \item
      Da $L/\Rational$ galoisch ist, wissen wir, dass $|{\Gal(L / \Rational)}| = [L : \Rational] = 4$.
      Außerdem muss jedes $\sigma \in \Gal(L : \Rational)$ die Nullstellen der rationalen Polynome $X^2 - 2, X^2 - 3 \in \Rational[X]$ permutieren;
      es muss also $\sigma(\sqrt{2}) = \pm \sqrt{2}$ und $\sigma(\sqrt{3}) = \pm \sqrt{3}$.
      Da $L$ von $\sqrt{2}$ und $\sqrt{3}$ erzeugt wird, ist $\sigma$ durch die beiden Werte $\sigma(\sqrt{2})$ und $\sigma(\sqrt{3})$ auch schon eindeutig bestimmt.
      
      Zusammen mit $|{\Gal(L / \Rational)}| = 4$ erhalten wir hieraus, dass die vier Automorphismen $\sigma_1, \sigma_2, \sigma_3, \sigma_4 \colon \Gal(L / \Rational)$ durch
      \begin{gather*}
        \sigma_1 \colon
        \left\{
          \begin{array}{ccr}
            \sqrt{2} & \mapsto  & \sqrt{2}, \\
            \sqrt{3} & \mapsto  & \sqrt{3},
          \end{array}
        \right.
        \quad
        \sigma_2 \colon
        \left\{
          \begin{array}{ccr}
            \sqrt{2} & \mapsto  & -\sqrt{2},  \\
            \sqrt{3} & \mapsto  &  \sqrt{3},
          \end{array}
        \right.
        \\
        \sigma_3 \colon
        \left\{
          \begin{array}{ccr}
            \sqrt{2} & \mapsto  &  \sqrt{2},  \\
            \sqrt{3} & \mapsto  & -\sqrt{3},
          \end{array}
        \right.
        \quad
        \sigma_4 \colon
        \left\{
          \begin{array}{ccr}
            \sqrt{2} & \mapsto  & -\sqrt{2},  \\
            \sqrt{3} & \mapsto  & -\sqrt{3},
          \end{array}
        \right.
      \end{gather*}
      gegeben sind.
      Insbesondere erhalten wir, dass $\Gal(L / \Rational) \cong \Integer/2 \times \Integer/2$,
      weshalb $\Gal(L / \Rational)$ abelsch ist.
  \end{enumerate}
\end{solution}


\begin{question}
  Es sei $\sigma \in \Gal(\Real/\Rational)$.
  \begin{enumerate}
    \item
      Zeigen Sie, dass für alle $x \in \Real$ genau dann $x \geq 0$, wenn $\sigma(x) \geq 0$.
    \item
      Folgern Sie, dass $\sigma$ streng monoton steigend ist.
    \item
      Folgern Sie, dass $\sigma$ stetig ist.
    \item
      Folgern Sie, dass $\sigma = \id_\Real$.
  \end{enumerate}
  Das zeigt, dass $\Gal(\Real/\Rational) = 1$.
\end{question}


\begin{solution}
  \begin{enumerate}
    \item
      Eine reelle Zahl ist genau dann nicht-negativ, wenn sie eine Quadratzahl ist;
      diese Eigenschaft ist invariant unter Körperautomorphismen.
      
    \item
      Für alle $x, y \in \Real$ gilt
      \[
              x \geq y
        \iff  x - y \geq 0
        \iff  \sigma(x) - \sigma(y) \geq 0
        \iff  \sigma(x) \geq \sigma(y).
      \]
      Somit ist $\sigma$ monoton steigend;
      dass $\sigma$ bereits \emph{streng} monoton steigend ist ergibt sich aus der Injektivität von $\sigma$.
      
    \item
      Für alle $x, y \in \Real$ mit $x < y$ gilt nach dem vorherigen Aussagenteil, dass genau dann $x < z < y$, wenn 
      $\sigma(x) < \sigma(z) < \sigma(y)$; also bildet $\sigma$ offene Intervalle auf offene Intervalle ab.
      Da eine jede offene Menge $U \subseteq \real$ eine Vereinigung offener Intervalle ist, folgt daraus, dass auch $\sigma(U) \subseteq \real$ offen ist.
      Wendet man dieses Resultat auf $\sigma^{-1} \in \Gal(\Real/\Rational)$ an, so ergibt sich, dass für jede offene Teilmenge $U \subseteq \Real$ auch $\sigma^{-1}(U)$ offen ist.
    
    \item
      Da $\Rational \subseteq \Real$ dicht liegt, folgt aus $\sigma|_{\Rational} = \id_\Rational$ und der Stetigkeit von $\sigma$, dass bereits $\sigma = \id_\Real$ gilt.
  \end{enumerate}
\end{solution}


\begin{question}
  Es sei $f(X) \coloneqq X^3 - 2 X^2 - X + 1 \in \Rational[X]$ und $\alpha \in \Complex$ eine Nullstelle von $f$.
  \begin{enumerate}
    \item
      Bestimmen Sie den Grad von $\Rational(\alpha) / \Rational$.
    \item
      Zeigen Sie, dass auch $\alpha(\alpha-2)$ eine Nullstelle von $f$ ist.
    \item
      Folgern Sie, dass $\Rational(\alpha)/\Rational$ galoisch ist.
    \item
      Bestimmen Sie $\Gal(\Rational(\alpha)/\Rational)$ bis auf Isomorphie.
  \end{enumerate}
\end{question}


\begin{solution}
  \begin{enumerate}
    \item
      Durch Reduzieren bezüglich des Primelements $2 \in \Integer$ erhält man das kubische Polynom $\tilde{f}(X) = X^3 + X + 1 \in \Field_2[X]$.
      Da $\tilde{f}(0) = \tilde{f}(1) = 1$ hat $\tilde{f}$ keine Nullstellen.
      Da $\tilde{f}$ kubisch ist, ist $\tilde{f}$ somit bereits irreduzibel.
      Also ist auch $f$ schon irreduzibel.
      Folglich ist $f$ bereits das Minimalpolynom von $\alpha$.
      Somit gilt $[\Rational(\alpha) : \Rational] = \deg f = 3$.
      
    \item
      Es gilt
      \begin{align*}
        \label{equation: polynomial of degree 6}
            f(\alpha(\alpha-2))
        &=  \alpha^3 (\alpha-2)^3 - 2 \alpha^2 (\alpha-2)^2 - \alpha(\alpha-2) + 1
        \\
        &=  \alpha^6 - 6 \alpha^5 + 10 \alpha^4 - 9 \alpha^2 + 2 \alpha + 1.
      \end{align*}
      Aus $0 = f(\alpha) = \alpha^3 - 2 \alpha^2 - \alpha + 1$ erhalten wir, dass $\alpha^3 = 2 \alpha^2 + \alpha - 1$.
      Somit gelten auch $\alpha^4 = 2 \alpha^3 + \alpha^2 - \alpha$, $\alpha^5 = 2 \alpha^4 + \alpha^3 - \alpha^2$ und $\alpha^6 = 2 \alpha^5 + \alpha^4 - \alpha^3$.
      Einsetzen von $\alpha^6 = 2 \alpha^5 + \alpha^4 - \alpha^3$ liefert
      \[
          \alpha^6 - 6 \alpha^5 + 10 \alpha^4 - 9 \alpha^2 + 2 \alpha + 1
        = -4 \alpha^5 + 11 \alpha^4 - \alpha^3 - 9 \alpha^2 + 2 \alpha + 1.
      \]
      Einsetzen von $\alpha^5 = 2 \alpha^4 + \alpha^3 - \alpha^2$ liefert
      \[
          -4 \alpha^5 + 11 \alpha^4 - \alpha^3 - 9 \alpha^2 + 2 \alpha + 1
        = 3 \alpha^4 - 5 \alpha^3 - 5 \alpha^2 + 2 \alpha + 1.
      \]
      Einsetzen von $\alpha^4 = 2 \alpha^3 + \alpha^2 - \alpha$ liefert schließlich
      \[
          3 \alpha^4 - 5 \alpha^3 - 5 \alpha^2 + 2 \alpha + 1
        = \alpha^3 - 2 \alpha^2 - \alpha + 1
        = 0.
      \]
      Ingesamt gilt also $f(\alpha (\alpha - 2)) = \dotsb = 0$.
%       
    \item
      Da $\ringchar \Rational = 0$ ist $f \in \Rational[X]$ als irreduzibles Polynom bereits separabel.
      Inbesondere sind die Nullstellen $\alpha$ und $\alpha (\alpha - 2)$ verschieden.
      Also hat $f$ in $\Rational(\alpha)$ zwei verschiedene Nullstellen; da $f$ kubisch ist, zerfällt $f$ deshalb über $\Rational(\alpha)$ bereits in Linearfaktoren.
      Es ist also $\Rational(\alpha)$ der Zerfällungskörper des separablen Polynoms $f$, und die Erweiterung $\Rational(\alpha)/\Rational$ somit galoisch.
%       
    \item
      Da $\Rational(\alpha)/\Rational$ galoisch ist, gilt $|\Gal(\Rational(\alpha)/\Rational)| = [\Rational(\alpha) : \Rational] = 3$.
      Da $\Integer/3$ bis auf Isomorphie die einzige dreielementige Gruppe ist, gilt also $\Gal(\Rational(\alpha)/\Rational) \cong \Integer/3$.
  \end{enumerate}
\end{solution}


\begin{question}
  \label{question: characterization of fields via its ideals}
  Zeigen Sie, dass für einen kommutativen Ring $K$ die folgenden Bedingungen äquivalent sind:
  \begin{enumerate}
    \item
      \label{enum: K is a field}
      $K$ ist ein Körper.
    \item
      \label{enum: K has exactly two ideals}
      $K$ hat genau zwei Ideale.
    \item
      \label{enum: The zero ideal is maximal}
      Das Nullideal in $K$ ist maximal.
  \end{enumerate}
\end{question}


\begin{solution}
  (\ref{enum: K is a field} $\implies$ \ref{enum: K has exactly two ideals})
  Da $K$ ein Körper ist gilt $0 \neq K$, also hat $K$ mindestens zwei Ideale.
  Ist $I \subseteq K$ ein Ideal mit $I \neq 0$, so gibt es ein $x \in I$ mit $x \neq 0$.
  Dann ist $x$ eine Einheit in $K$, somit $K = (x) \subseteq I$ und deshalb $I = K$.
  Also sind $0$ und $K$ die einzigen Ideale in $K$.
  
  (\ref{enum: K has exactly two ideals} $\implies$ \ref{enum: The zero ideal is maximal})
  Es muss $0 \neq K$, denn sonst wäre $0$ das einzige Ideal in $K$.
  Also sind $0$ und $K$ die einzigen beiden Ideale in $K$.
  Ist $I \subseteq K$ ein Ideal mit $0 \subsetneq I$, so muss bereits $I = K$.
  Also ist $0$ ein maximales Ideal.
  
  (\ref{enum: The zero ideal is maximal} $\implies$ \ref{enum: K is a field})
  Da $0 \subseteq K$ maximal ist, ergibt sich, dass $K \cong K/0$ ein Körper ist.
\end{solution}


\begin{question}
  Es sei $K$ ein algebraisch abgeschlossener Körper.
  Zeigen Sie, dass $K$ unendlich ist.
\end{question}


\begin{solution}
  Wäre $K$ endlich, so wäre
  \[
              p(T)
    \coloneqq 1 + \prod_{\lambda \in K} (T - \lambda)
    \in       K[T]
  \]
  ein Polynom positiven Grades ohne Nullstellen (denn $p(x) = 1$ für alle $x \in K$).
  Dies stünde im Widerspruch zur algebraischen Abgeschlossenheit von $K$.
\end{solution}


\begin{question}
  Es sei $K$ ein Körper und $p \in K[T]$ ein Polynom mit $\deg p \in \{2, 3\}$.
  Zeigen Sie, dass $p$ genau dann irreduzibel ist, wenn $p$ keine Nullstelle hat.
\end{question}


\begin{solution}
  Wäre $p$ reduzibel, so gebe $q_1, q_2 \in K[T]$ mit $p = q_1 q_2$ und $\deg q_1, \deg q_2 \geq 1$.
  Es müsste dann $\deg p = \deg q_1 + \deg q_2$ und somit $\deg q_1 = 1$ oder $\deg q_2 = 1$.
  Also besäße $p$ einen Teiler vom Grad $1$;
  dieser wäre bis auf Normierung ein Linearfaktor, weshalb $p$ ein Nullstelle hätte.
\end{solution}


\begin{question}
  Es seien $p, q \in K[T]$ zwei normierte irreduzible Polynome mit $p \neq q$.
  Zeigen Sie, dass $p$ und $q$ in $\overline{K}$ keine gemeinsamen Nullstellen haben.
\end{question}


\begin{solution}
  Gebe es eine gemeinsame Nullstelle $\alpha \in \overline{K}$ von $p$ und $q$, so wären $p$ und $q$ beide das Minimalpolynom von $\alpha$ über $K$, und somit $p = q$.
\end{solution}


\begin{question}
  Es sei $K(\alpha)/K$ eine endliche, zyklische Körpererweiterung von ungeraden Grad.
  Zeigen Sie, dass $K(\alpha) = K(\alpha^2)$.
\end{question}


\begin{solution}
  Da $K(\alpha^2) \subseteq K(\alpha)$ gilt, genügt es zu zeigen, dass $\alpha^2 \in K(\alpha)$.
  Wir nehmen an, dass $\alpha^2 \notin K(\alpha)$.
  Dann ist das normierte quadratische Polynom $P(T) \coloneqq T^2 - \alpha^2 \in K(\alpha^2)[T]$ irreduzibel mit $P(\alpha) = 0$, und deshalb das Minimalpolynom von $\alpha$ über $K(\alpha^2)$.
  Es ist also $[K(\alpha) : K(\alpha^2)] = 2$.
  Damit gilt
  \[
      [K(\alpha) : K]
    = [K(\alpha) : K(\alpha^2)] [K(\alpha^2) : K]
    = 2 [K(\alpha^2) : K],
  \]
  was im Widerspruch dazu steht, dass $[K(\alpha) : K]$ ungerade ist.
\end{solution}


\begin{question}
  Es sei $K$ ein algebraisch abgeschlossener Körper und $L/K$ eine algebraische Körpererweiterung.
  Zeigen Sie, dass bereits $L = K$ gilt.
\end{question}


\begin{solution}
  Es sei $\alpha \in L$.
  Da $L/K$ algebraisch ist, gibt es ein normiertes Polynom $P \in K[T]$ mit $P \neq 0$ und $P(\alpha) = 0$.
  Da $K$ algebraisch abgeschlossen ist zerfällt $P$ in Linearfaktoren, also $P(T) = (T - a_1) \dotsm (T - a_n)$ mit $a_1, \dotsc, a_n \in K$ und $n = \deg P$.
  Da
  \[
      0
    = P(\alpha)
    = (\alpha - a_1) \dotsm (\alpha - a_n)
  \]
  muss bereits $\alpha = a_i$ für ein $1 \leq i \leq n$, und somit $\alpha \in K$.
\end{solution}


\begin{question}
  \label{equation: finite extensions are algebraic}
  Zeigen Sie, dass endliche Körpererweiterungen algebraisch sind.
\end{question}


\begin{solution}
  Es sei $L/K$ eine endliche Körpererweiterung und $x \in L$.
  Für den $K$-Untervektorraum $\generate{ \{x^n \mid n \in \Natural\} }_K \subseteq L$ gilt
  \[
          \dim_K \generate{ \{x^n \mid n \in \Natural \} }_K
    \leq  \dim_K L
    =     [L : K]
    <     \infty,
  \]
  weshalb die Potenzen $x^n$ mit $n \in \Natural$ linear abhängig über $K$ sind.
  Also gibt es eine nichttriviale Linearkombination
  \[
    a_n x^n + \dotsb + a_1 x + a_0 = 0
  \]
  mit $n \geq 1$ und $a_n, \dotsc, a_0 \in K$ mit $a_n \neq 0$.
  Für das Polynom
  \[
              P(T)
    \coloneqq a_n T^n + \dotsb + a_1 T + a_0
    \in       K[T]
  \]
  gilt also $P(x) = 0$, weshalb $x$ algebraisch über $K$ ist.
\end{solution}


\begin{question}
  Es seien $M / L / K$ Körpererweiterungen, so dass $M/L$ und $L/K$ algebraisch sind.
  Zeigen Sie, dass auch $M/K$ algebraisch ist.
\end{question}


\begin{question}
  Es sei $x \in M$.
  Da $M/L$ algebraisch ist gibt es ein Polynom $p(T) \in L(T)$ mit $p \neq 0$ und $p(x) = 0$.
  Es seien $a_0, \dotsc, a_n \in L$ die Koeffizienten von $p$.
  Da $L/K$ algebraisch ist sind $a_0, \dotsc, a_n$ algebraisch über $K$.
  Für $L' \coloneqq K(a_0, \dotsc, a_n)$ wird die Erweiterung $L'/K$ von endlich vielen algebraischen Elementen erzeugt und ist deshalb endlich.
  Es gilt bereits $p(T) \in L'(T)$, weshalb $x$ algebraisch über $L'$ ist.
  Insbesondere ist deshalb $L'(x)/L'$ endlich.
  Ingesamt gilt also $[K(x) : K] \leq [L'(x) : K] = [L'(x) : L'][L' : K] < \infty$.
  Die Erweiterung $K(x)/K$ ist also endlich, und damit algebraisch (siehe Übung~\ref{equation: finite extensions are algebraic}).
  Inbesondere ist $x$ algebraisch über $K$.
\end{question}


\begin{question}
  Es sei $L/K$ eine Körpererweiterung und es seien $\alpha, \beta \in L$.
  Zeigen Sie, dass $\alpha$ und $\beta$ genau dann beide algebraisch über $K$ sind, wenn $\alpha + \beta$ und $\alpha \beta$ beide algebraisch über $K$ sind.
\end{question}


\begin{remark*}
  Da $\pi$ und $e$ transzenent (über $\Rational$) sind, muss von den beiden Zahlen $\pi + e$ und $\pi e$ mindestens eine transzendent sein.
  Es ist nicht bekannt, welches von ihnen es ist.
\end{remark*}


\begin{solution}
  Sind $\alpha$ und $\beta$ algebraisch über $K$, so ist $K(\alpha, \beta) / K$ eine algebraische Körpererweiterung.
  Da $\alpha + \beta, \alpha \beta \in K(\alpha, \beta)$ sind $\alpha + \beta$ und $\alpha \beta$ dann algebraisch über $K$.
  
  Es seien nun $\alpha + \beta$ und $\alpha \beta$ algebraisch über $K$.
  Dann ist $K(\alpha + \beta, \alpha \beta)/K$ eine algebraische Erweiterung.
  Auch die Erweiterung $K(\alpha, \beta)/K(\alpha + \beta, \alpha \beta)$ ist algebraisch, da $\alpha$ und $\beta$ Nullstellen des Polynoms
  \[
              P(T)
    \coloneqq (T - \alpha)(T - \beta)
    =         T^2 - (\alpha + \beta)T + \alpha \beta
    \in       K(\alpha + \beta, \alpha \beta)[T]
  \]
  sind.
  Wegen der Transitivität von Algebraizität folgt, dass auch $K(\alpha,\beta)/K$ algebraisch ist, also $\alpha$ und $\beta$ algebraisch über $K$ sind.
\end{solution}


\begin{question}
  Es sei $L/K$ eine Körpererweiterung, so dass $p \coloneqq [L : K]$ endlich und prim ist.
  Zeigen Sie, dass $L/K$ ein zyklische Erweiterung ist, und bestimmen Sie alle $\alpha \in L$ mit $L = K(\alpha)$.
\end{question}


\begin{solution}
  Für alle $\alpha \in K$ ist $K(\alpha) = K$.
  Ist $\alpha \in L$ mit $\alpha \notin K$, so ist $K(\alpha)/K$ eine echte Körperweiterung, weshalb $[K(\alpha) : K] \neq 1$ gilt.
  Aus
  \[
      p
    = [L : K]
    = [L : K(\alpha)] \underbrace{[K(\alpha) : K]}_{\neq 1}
  \]
  folgt, da $p$ prim ist, dass $[L : K(\alpha)] = 1$ (und $[K(\alpha) : K] = p$), und somit $K(\alpha) = L$.
  Also ist $L$ eine zyklische Körpererweiterung, und die möglichen Elemente sind genau die $\alpha \in L$, für die $\alpha \notin K$.
\end{solution}


\begin{question}
  Es sei $L/K$ eine endliche Körpererweiterung mit $[L : K] = 2^k$ für ein $k \geq 0$.
  Es sei $P \in K[T]$ ein kubisches Polynom, das eine Nullstelle in $L$ hat.
  Zeigen Sie, dass $f$ bereits eine Nullstelle in $K$ hat.
\end{question}


\begin{solution}
  Wir können o.B.d.A.\ davon ausgehen, dass $P$ normiert ist.
  Es sei $\alpha \in L$ eine Nullstelle von $P$.
  Hätte $P$ keine Nullstelle in $K$, so wäre $P$ irreduzibel in $K[T]$, da $P$ kubisch ist.
  Damit wäre dann $P$ das Minimalpolynom von $\alpha$ über $K$, und somit $[K(\alpha) : K] = \deg P = 3$.
  Dann wäre aber
  \[
          3
    =     [K(\alpha) : K]
    \mid  [L : K(\alpha)] [K(\alpha) : K]
    =     [L : K]
    =     2^k,
  \]
  was nicht gilt.
\end{solution}


\begin{question}
  Es sei $K$ ein Körper und $f \in K[T]$ ein irreduzibles Polynom.
  \begin{enumerate}
    \item
      Zeigen Sie, dass $f$ im Fall $\ringchar K = 0$ separabel ist.
    \item
      Zeigen Sie durch Angabe eines Beispiels, dass $f$ im Fall $\ringchar K > 0$ nicht notwendigerweise separabel ist.
  \end{enumerate}
\end{question}


\begin{enumerate}
  \item
    Wegen der Irreduziblität von $f$ gilt $\deg f \geq 1$.
    Wegen $\ringchar K = 0$ folgt, dass $f' \neq 0$.
    Da aber $\deg f' = \deg f - 1 < \deg f$ gilt, folgt aus der Irreduziblität von $f$, dass $f$ und $f'$ teilerfremd sind.
    Also ist $f$ separabel.
  \item
    Ist $p \coloneqq \ringchar K > 0$, so ist das Polynom $f(X) \coloneqq X^p - t \in \Field_p(t)[X]$ nach Eisenstein irreduzibel.
    Es gilt aber $f' = 0$, weshalb $f$ und $f'$ nicht teilerfremd, und $f$ somit nicht separabel ist.
\end{enumerate}


\begin{question}
  Zeigen Sie, dass eine Körpererweiterung $L/K$ genau dann algebraisch ist, wenn jeder Zwischenring $K \subseteq R \subseteq L$ bereits ein Körper ist.
\end{question}


\begin{solution}
  Es sei $L/K$ algebraisch und $K \subseteq R \subseteq L$ ein Zwischenring.
  Für $\alpha \in R$ ist dann $\alpha$ algebraisch über $K$, und somit $K(\alpha) = K[\alpha]$.
  Da $R$ ein Ring ist, der $\alpha$ und $R$ enthält, gilt $K[\alpha] \subseteq R$.
  Somit ist $K(\alpha) = K[\alpha] \subseteq R$.
  Ist $\alpha \neq 0$, so ist inbesondere $\alpha^{-1} \in K(\alpha) \subseteq R$.
  Das zeigt, dass jedes Element $\alpha \in R$ mit $\alpha \neq 0$ in $R$ invertierbar ist.
  Somit ist $R$ ein Körper.
  (Die Kommutativität von $R$ ist klar, es sich um einen Unterring von $L$ handelt, und $L$ als Körper kommutativ ist.)
  
  Es sei nun $L/K$ nicht algebraisch.
  Dann gibt es ein Element $\alpha \in L$, das transzendent über $K$ ist.
  Der Zwischenring $K \subseteq K[\alpha] \subseteq L$ ist dann kein Körper:
  Für den Polynomring $K[T]$ ist der Einsetzhomorphismus $K[T] \to K[\alpha]$, $P(T) \to P(\alpha)$ surjektiv, und wegen der Transzendenz von $\alpha$ auch injektiv, und somit ein Isomorphismus.
  Der Polynomring $K[T]$, und somit auch $K[\alpha]$, ist aber kein Körper.
\end{solution}




